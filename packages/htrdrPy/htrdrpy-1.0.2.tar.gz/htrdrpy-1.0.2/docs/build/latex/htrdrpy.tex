%% Generated by Sphinx.
\def\sphinxdocclass{report}
\documentclass[a4paper,10pt,english]{sphinxmanual}
\ifdefined\pdfpxdimen
   \let\sphinxpxdimen\pdfpxdimen\else\newdimen\sphinxpxdimen
\fi \sphinxpxdimen=.75bp\relax
\ifdefined\pdfimageresolution
    \pdfimageresolution= \numexpr \dimexpr1in\relax/\sphinxpxdimen\relax
\fi
%% let collapsible pdf bookmarks panel have high depth per default
\PassOptionsToPackage{bookmarksdepth=5}{hyperref}

\PassOptionsToPackage{booktabs}{sphinx}
\PassOptionsToPackage{colorrows}{sphinx}

\PassOptionsToPackage{warn}{textcomp}
\usepackage[utf8]{inputenc}
\ifdefined\DeclareUnicodeCharacter
% support both utf8 and utf8x syntaxes
  \ifdefined\DeclareUnicodeCharacterAsOptional
    \def\sphinxDUC#1{\DeclareUnicodeCharacter{"#1}}
  \else
    \let\sphinxDUC\DeclareUnicodeCharacter
  \fi
  \sphinxDUC{00A0}{\nobreakspace}
  \sphinxDUC{2500}{\sphinxunichar{2500}}
  \sphinxDUC{2502}{\sphinxunichar{2502}}
  \sphinxDUC{2514}{\sphinxunichar{2514}}
  \sphinxDUC{251C}{\sphinxunichar{251C}}
  \sphinxDUC{2572}{\textbackslash}
\fi
\usepackage{cmap}
\usepackage[T1]{fontenc}
\usepackage{amsmath,amssymb,amstext}
\usepackage{babel}



\usepackage{tgtermes}
\usepackage{tgheros}
\renewcommand{\ttdefault}{txtt}



\usepackage[Bjarne]{fncychap}
\usepackage{sphinx}

\fvset{fontsize=auto}
\usepackage{geometry}


% Include hyperref last.
\usepackage{hyperref}
% Fix anchor placement for figures with captions.
\usepackage{hypcap}% it must be loaded after hyperref.
% Set up styles of URL: it should be placed after hyperref.
\urlstyle{same}


\usepackage{sphinxmessages}
\setcounter{tocdepth}{5}
\setcounter{secnumdepth}{5}

\usepackage[utf8]{inputenc}
\usepackage[T1]{fontenc}
\usepackage{textcomp}
\usepackage{enumitem}
\setlistdepth{9}
\renewlist{itemize}{itemize}{9}
\setlist[itemize]{label=\textbullet}
\setlist[itemize,1]{label=\textbullet}
\setlist[itemize,2]{label=\textbullet}
\setlist[itemize,3]{label=\textbullet}
\setlist[itemize,4]{label=\textbullet}
\setlist[itemize,5]{label=\textbullet}
\setlist[itemize,6]{label=\textbullet}
\setlist[itemize,7]{label=\textbullet}
\setlist[itemize,8]{label=\textbullet}
\setlist[itemize,9]{label=\textbullet}

\let\oldsphinxtableofcontents\sphinxtableofcontents
\renewcommand{\sphinxtableofcontents}{%
    \setcounter{tocdepth}{6}%
    \oldsphinxtableofcontents%
}
    

\title{htrdrPy}
\date{Jan 09, 2026}
\release{0.0.1}
\author{Anthony Arfaux}
\newcommand{\sphinxlogo}{\vbox{}}
\renewcommand{\releasename}{Release}
\makeindex
\begin{document}

\ifdefined\shorthandoff
  \ifnum\catcode`\=\string=\active\shorthandoff{=}\fi
  \ifnum\catcode`\"=\active\shorthandoff{"}\fi
\fi

\pagestyle{empty}
\sphinxmaketitle
\pagestyle{plain}
\sphinxtableofcontents
\pagestyle{normal}
\phantomsection\label{\detokenize{index::doc}}



\chapter{Introduction}
\label{\detokenize{index:introduction}}
\sphinxAtStartPar
htrdrPy is a wrapper for htrdr\sphinxhyphen{}planets (\sphinxurl{https://www.meso-star.com/projects/htrdr/htrdr.html}).
htrdr\sphinxhyphen{}planets is not included in the wrapper, it must be install separately.
htrdrPy is meant to simplfy the use, by managing the input data, scripts,
observation geometry and treatement of the results.


\chapter{Installation}
\label{\detokenize{index:installation}}
\sphinxAtStartPar
htrdrPy can be installed from the Python Package Index with the following
command:

\begin{sphinxVerbatim}[commandchars=\\\{\}]
\PYGZdl{} pip install htrdrPy
\end{sphinxVerbatim}


\chapter{Summary}
\label{\detokenize{index:summary}}
\sphinxstepscope


\section{Examples}
\label{\detokenize{Examples:examples}}\label{\detokenize{Examples::doc}}
\sphinxAtStartPar
This is an example of use of the htrdrPy package

\sphinxstepscope


\section{htrdrPy documentation}
\label{\detokenize{htrdrPy:htrdrpy-documentation}}\label{\detokenize{htrdrPy::doc}}
\sphinxstepscope


\subsection{htrdrPy.data module}
\label{\detokenize{htrdrPy.data:module-htrdrPy.data}}\label{\detokenize{htrdrPy.data:htrdrpy-data-module}}\label{\detokenize{htrdrPy.data::doc}}\index{module@\spxentry{module}!htrdrPy.data@\spxentry{htrdrPy.data}}\index{htrdrPy.data@\spxentry{htrdrPy.data}!module@\spxentry{module}}\index{Data (class in htrdrPy.data)@\spxentry{Data}\spxextra{class in htrdrPy.data}}

\begin{fulllineitems}
\phantomsection\label{\detokenize{htrdrPy.data:htrdrPy.data.Data}}
\pysigstartsignatures
\pysiglinewithargsret
{\sphinxbfcode{\sphinxupquote{\DUrole{k}{class}\DUrole{w}{ }}}\sphinxcode{\sphinxupquote{htrdrPy.data.}}\sphinxbfcode{\sphinxupquote{Data}}}
{\sphinxparam{\DUrole{n}{radius}}\sphinxparamcomma \sphinxparam{\DUrole{n}{nTheta}\DUrole{o}{=}\DUrole{default_value}{None}}\sphinxparamcomma \sphinxparam{\DUrole{n}{nPhi}\DUrole{o}{=}\DUrole{default_value}{None}}\sphinxparamcomma \sphinxparam{\DUrole{n}{mass}\DUrole{o}{=}\DUrole{default_value}{None}}\sphinxparamcomma \sphinxparam{\DUrole{n}{gravity}\DUrole{o}{=}\DUrole{default_value}{None}}\sphinxparamcomma \sphinxparam{\DUrole{n}{name}\DUrole{o}{=}\DUrole{default_value}{\textquotesingle{}\textquotesingle{}}}}
{}
\pysigstopsignatures
\sphinxAtStartPar
Bases: \sphinxcode{\sphinxupquote{object}}

\sphinxAtStartPar
\sphinxcode{\sphinxupquote{htrdrPy.Data}} is a class aiming to handle the optical and physical properties
of the system and to create the input files for htrdr.
\subsubsection*{Examples}

\sphinxAtStartPar
The first step is the creation of a instance of \sphinxcode{\sphinxupquote{htrdrPy.Data}}:

\begin{sphinxVerbatim}[commandchars=\\\{\}]
\PYG{g+gp}{\PYGZgt{}\PYGZgt{}\PYGZgt{} }\PYG{n}{d} \PYG{o}{=} \PYG{n}{htrdrPy}\PYG{o}{.}\PYG{n}{Data}\PYG{p}{(}\PYG{n}{radius}\PYG{o}{=}\PYG{l+m+mf}{1e6}\PYG{p}{,} \PYG{n}{nTheta}\PYG{o}{=}\PYG{l+m+mi}{30}\PYG{p}{,} \PYG{n}{nPhi}\PYG{o}{=}\PYG{l+m+mi}{50}\PYG{p}{,} \PYG{n}{name}\PYG{o}{=}\PYG{l+s+s2}{\PYGZdq{}}\PYG{l+s+s2}{Planet}\PYG{l+s+s2}{\PYGZdq{}}\PYG{p}{)}
\end{sphinxVerbatim}

\sphinxAtStartPar
The next step is to provide the physical and radiative properties of the
atmosphere and ground. Different methods exist depending on the case
considered. In the following, we consider a 1D set of data forming an
horizontally homogeneous planet.

\begin{sphinxVerbatim}[commandchars=\\\{\}]
\PYG{g+gp}{\PYGZgt{}\PYGZgt{}\PYGZgt{} }\PYG{n}{nLevel} \PYG{o}{=} \PYG{l+m+mi}{50}
\PYG{g+gp}{\PYGZgt{}\PYGZgt{}\PYGZgt{} }\PYG{n}{nCoeff} \PYG{o}{=} \PYG{l+m+mi}{4}
\PYG{g+gp}{\PYGZgt{}\PYGZgt{}\PYGZgt{} }\PYG{n}{nWavelengths} \PYG{o}{=} \PYG{l+m+mi}{20}
\PYG{g+gp}{\PYGZgt{}\PYGZgt{}\PYGZgt{} }\PYG{n}{weights} \PYG{o}{=} \PYG{n}{np}\PYG{o}{.}\PYG{n}{array}\PYG{p}{(}\PYG{n}{nWavelengths} \PYG{o}{*} \PYG{p}{[}\PYG{l+m+mf}{0.2}\PYG{p}{,} \PYG{l+m+mf}{0.3}\PYG{p}{,} \PYG{l+m+mf}{0.3}\PYG{p}{,} \PYG{l+m+mf}{0.2}\PYG{p}{]}\PYG{p}{)}\PYG{o}{.}\PYG{n}{reshape}\PYG{p}{(}\PYG{n}{nWavelengths}\PYG{p}{,} \PYG{n}{nCoeff}\PYG{p}{)}
\PYG{g+gp}{\PYGZgt{}\PYGZgt{}\PYGZgt{} }\PYG{n}{altitudes} \PYG{o}{=} \PYG{n}{np}\PYG{o}{.}\PYG{n}{linspace}\PYG{p}{(}\PYG{l+m+mi}{0}\PYG{p}{,} \PYG{l+m+mf}{5e5}\PYG{p}{,} \PYG{n}{nLevel}\PYG{p}{)}
\PYG{g+gp}{\PYGZgt{}\PYGZgt{}\PYGZgt{} }\PYG{n}{temperatures} \PYG{o}{=} \PYG{n}{np}\PYG{o}{.}\PYG{n}{linspace}\PYG{p}{(}\PYG{l+m+mi}{300}\PYG{p}{,} \PYG{l+m+mi}{500}\PYG{p}{,} \PYG{n}{nLevel}\PYG{p}{)}
\PYG{g+gp}{\PYGZgt{}\PYGZgt{}\PYGZgt{} }\PYG{n}{scatt} \PYG{o}{=} \PYG{n}{np}\PYG{o}{.}\PYG{n}{linspace}\PYG{p}{(}\PYG{l+m+mf}{1e\PYGZhy{}8}\PYG{p}{,} \PYG{l+m+mf}{1e\PYGZhy{}2}\PYG{p}{,}
\PYG{g+gp}{... }    \PYG{n}{nLevel}\PYG{o}{*}\PYG{n}{nCoeff}\PYG{o}{*}\PYG{n}{nWavelengths}\PYG{p}{)}\PYG{o}{.}\PYG{n}{reshape}\PYG{p}{(}\PYG{p}{(}\PYG{n}{nWavelengths}\PYG{p}{,} \PYG{n}{nLevel}\PYG{p}{,} \PYG{n}{nCoeff}\PYG{p}{)}\PYG{p}{)}
\PYG{g+gp}{\PYGZgt{}\PYGZgt{}\PYGZgt{} }\PYG{n}{absor} \PYG{o}{=} \PYG{n}{np}\PYG{o}{.}\PYG{n}{linspace}\PYG{p}{(}\PYG{l+m+mf}{1e\PYGZhy{}5}\PYG{p}{,} \PYG{l+m+mf}{1e\PYGZhy{}1}\PYG{p}{,}
\PYG{g+gp}{... }    \PYG{n}{nLevel}\PYG{o}{*}\PYG{n}{nCoeff}\PYG{o}{*}\PYG{n}{nWavelengths}\PYG{p}{)}\PYG{o}{.}\PYG{n}{reshape}\PYG{p}{(}\PYG{p}{(}\PYG{n}{nWavelengths}\PYG{p}{,} \PYG{n}{nLevel}\PYG{p}{,} \PYG{n}{nCoeff}\PYG{p}{)}\PYG{p}{)}
\PYG{g+gp}{\PYGZgt{}\PYGZgt{}\PYGZgt{} }\PYG{n}{asymm} \PYG{o}{=} \PYG{n}{np}\PYG{o}{.}\PYG{n}{linspace}\PYG{p}{(}\PYG{l+m+mi}{0}\PYG{p}{,} \PYG{l+m+mi}{1}\PYG{p}{,}
\PYG{g+gp}{... }    \PYG{n}{nLevel}\PYG{o}{*}\PYG{n}{nCoeff}\PYG{o}{*}\PYG{n}{nWavelengths}\PYG{p}{)}\PYG{o}{.}\PYG{n}{reshape}\PYG{p}{(}\PYG{p}{(}\PYG{n}{nWavelengths}\PYG{p}{,} \PYG{n}{nLevel}\PYG{p}{,} \PYG{n}{nCoeff}\PYG{p}{)}\PYG{p}{)}
\PYG{g+gp}{\PYGZgt{}\PYGZgt{}\PYGZgt{} }\PYG{n}{wavelengths} \PYG{o}{=} \PYG{n}{np}\PYG{o}{.}\PYG{n}{linspace}\PYG{p}{(}\PYG{l+m+mf}{2e7}\PYG{p}{,} \PYG{l+m+mf}{9e7}\PYG{p}{,} \PYG{n}{nWavelengths}\PYG{p}{)}
\PYG{g+gp}{\PYGZgt{}\PYGZgt{}\PYGZgt{} }\PYG{n}{bandsLow} \PYG{o}{=} \PYG{n}{np}\PYG{o}{.}\PYG{n}{zeros}\PYG{p}{(}\PYG{n}{nWavelengths}\PYG{p}{)}
\PYG{g+gp}{\PYGZgt{}\PYGZgt{}\PYGZgt{} }\PYG{n}{bandsUp} \PYG{o}{=} \PYG{n}{np}\PYG{o}{.}\PYG{n}{zeros}\PYG{p}{(}\PYG{n}{nWavelengths}\PYG{p}{)}
\PYG{g+gp}{\PYGZgt{}\PYGZgt{}\PYGZgt{} }\PYG{n}{bandsLow}\PYG{p}{[}\PYG{l+m+mi}{1}\PYG{p}{:}\PYG{p}{]} \PYG{o}{=} \PYG{n}{bandsUp}\PYG{p}{[}\PYG{p}{:}\PYG{o}{\PYGZhy{}}\PYG{l+m+mi}{1}\PYG{p}{]} \PYG{o}{=} \PYG{l+m+mf}{0.5} \PYG{o}{*} \PYG{p}{(}\PYG{n}{wavelengths}\PYG{p}{[}\PYG{l+m+mi}{1}\PYG{p}{:}\PYG{p}{]} \PYG{o}{+} \PYG{n}{wavelengths}\PYG{p}{[}\PYG{p}{:}\PYG{o}{\PYGZhy{}}\PYG{l+m+mi}{1}\PYG{p}{]}\PYG{p}{)}
\PYG{g+gp}{\PYGZgt{}\PYGZgt{}\PYGZgt{} }\PYG{n}{bandsLow}\PYG{p}{[}\PYG{l+m+mi}{0}\PYG{p}{]} \PYG{o}{=} \PYG{l+m+mf}{1.5} \PYG{o}{*} \PYG{n}{wavelengths}\PYG{p}{[}\PYG{l+m+mi}{0}\PYG{p}{]} \PYG{o}{\PYGZhy{}} \PYG{l+m+mf}{0.5} \PYG{o}{*} \PYG{n}{wavelengths}\PYG{p}{[}\PYG{l+m+mi}{1}\PYG{p}{]}
\PYG{g+gp}{\PYGZgt{}\PYGZgt{}\PYGZgt{} }\PYG{n}{bandsUp}\PYG{p}{[}\PYG{o}{\PYGZhy{}}\PYG{l+m+mi}{1}\PYG{p}{]} \PYG{o}{=}  \PYG{l+m+mf}{1.5} \PYG{o}{*} \PYG{n}{wavelengths}\PYG{p}{[}\PYG{o}{\PYGZhy{}}\PYG{l+m+mi}{1}\PYG{p}{]} \PYG{o}{\PYGZhy{}} \PYG{l+m+mf}{0.5} \PYG{o}{*} \PYG{n}{wavelengths}\PYG{p}{[}\PYG{o}{\PYGZhy{}}\PYG{l+m+mi}{2}\PYG{p}{]}
\PYG{g+gp}{\PYGZgt{}\PYGZgt{}\PYGZgt{} }\PYG{n}{surfTemp}\PYG{o}{=} \PYG{l+m+mi}{300}
\PYG{g+gp}{\PYGZgt{}\PYGZgt{}\PYGZgt{} }\PYG{n}{surfAlb}\PYG{o}{=} \PYG{n}{np}\PYG{o}{.}\PYG{n}{ones}\PYG{p}{(}\PYG{n}{nWavelengths}\PYG{p}{)} \PYG{o}{*} \PYG{l+m+mf}{0.5}
\PYG{g+gp}{\PYGZgt{}\PYGZgt{}\PYGZgt{} }\PYG{n}{d}\PYG{o}{.}\PYG{n}{makeAtmosphereFrom1D}\PYG{p}{(}\PYG{p}{\PYGZob{}}
\PYG{g+gp}{... }        \PYG{l+s+s2}{\PYGZdq{}}\PYG{l+s+s2}{nLevel}\PYG{l+s+s2}{\PYGZdq{}}\PYG{p}{:} \PYG{n}{nLevel}\PYG{p}{,}
\PYG{g+gp}{... }        \PYG{l+s+s2}{\PYGZdq{}}\PYG{l+s+s2}{nCoeff}\PYG{l+s+s2}{\PYGZdq{}}\PYG{p}{:} \PYG{n}{nCoeff}\PYG{p}{,}
\PYG{g+gp}{... }        \PYG{l+s+s2}{\PYGZdq{}}\PYG{l+s+s2}{nWavelengths}\PYG{l+s+s2}{\PYGZdq{}}\PYG{p}{:} \PYG{n}{nWavelengths}\PYG{p}{,}
\PYG{g+gp}{... }        \PYG{l+s+s2}{\PYGZdq{}}\PYG{l+s+s2}{weights}\PYG{l+s+s2}{\PYGZdq{}}\PYG{p}{:} \PYG{n}{weights}\PYG{p}{,}
\PYG{g+gp}{... }        \PYG{l+s+s2}{\PYGZdq{}}\PYG{l+s+s2}{altitude (m)}\PYG{l+s+s2}{\PYGZdq{}}\PYG{p}{:} \PYG{n}{altitudes}\PYG{p}{,}
\PYG{g+gp}{... }        \PYG{l+s+s2}{\PYGZdq{}}\PYG{l+s+s2}{temperature (K)}\PYG{l+s+s2}{\PYGZdq{}}\PYG{p}{:} \PYG{n}{temperatures}\PYG{p}{,}
\PYG{g+gp}{... }        \PYG{l+s+s2}{\PYGZdq{}}\PYG{l+s+s2}{scattering (m\PYGZhy{}1)}\PYG{l+s+s2}{\PYGZdq{}}\PYG{p}{:} \PYG{n}{scatt}\PYG{p}{,}
\PYG{g+gp}{... }        \PYG{l+s+s2}{\PYGZdq{}}\PYG{l+s+s2}{absorption (m\PYGZhy{}1)}\PYG{l+s+s2}{\PYGZdq{}}\PYG{p}{:} \PYG{n}{absor}\PYG{p}{,}
\PYG{g+gp}{... }        \PYG{l+s+s2}{\PYGZdq{}}\PYG{l+s+s2}{assymetry}\PYG{l+s+s2}{\PYGZdq{}}\PYG{p}{:} \PYG{n}{asymm}\PYG{p}{,}
\PYG{g+gp}{... }        \PYG{l+s+s2}{\PYGZdq{}}\PYG{l+s+s2}{wavelength}\PYG{l+s+s2}{\PYGZdq{}}\PYG{p}{:} \PYG{n}{wavelengths}\PYG{p}{,}
\PYG{g+gp}{... }        \PYG{l+s+s2}{\PYGZdq{}}\PYG{l+s+s2}{band low}\PYG{l+s+s2}{\PYGZdq{}}\PYG{p}{:} \PYG{n}{bandsLow}\PYG{p}{,}
\PYG{g+gp}{... }        \PYG{l+s+s2}{\PYGZdq{}}\PYG{l+s+s2}{band up}\PYG{l+s+s2}{\PYGZdq{}}\PYG{p}{:} \PYG{n}{bandsUp}
\PYG{g+gp}{... }        \PYG{p}{\PYGZcb{}}\PYG{p}{)}
\PYG{g+go}{Mesh generator. Ntheta = 30, Nphi = 50, Nz = 50, r\PYGZus{}min = 1000000.0, r\PYGZus{}max = 1500000.0}
\PYG{g+go}{Generating points...}
\PYG{g+go}{Generating nodes...}
\PYG{g+go}{Hexahedron \PYGZam{} Octahedron generation completed. N\PYGZus{}Hexahedron = 4802, N\PYGZus{}Octahedron = 64827}
\PYG{g+go}{Generating Tetrahedrons...}
\PYG{g+go}{Assigning data to nodes ...}
\PYG{g+gp}{\PYGZgt{}\PYGZgt{}\PYGZgt{} }\PYG{n}{d}\PYG{o}{.}\PYG{n}{makeGroundFrom1D}\PYG{p}{(}\PYG{n}{surfTemp}\PYG{p}{,} \PYG{p}{\PYGZob{}}
\PYG{g+gp}{... }        \PYG{l+s+s2}{\PYGZdq{}}\PYG{l+s+s2}{kind}\PYG{l+s+s2}{\PYGZdq{}}\PYG{p}{:} \PYG{l+s+s2}{\PYGZdq{}}\PYG{l+s+s2}{lambertian}\PYG{l+s+s2}{\PYGZdq{}}\PYG{p}{,}
\PYG{g+gp}{... }        \PYG{l+s+s2}{\PYGZdq{}}\PYG{l+s+s2}{albedo}\PYG{l+s+s2}{\PYGZdq{}}\PYG{p}{:} \PYG{n}{surfAlb}\PYG{p}{,}
\PYG{g+gp}{... }        \PYG{l+s+s2}{\PYGZdq{}}\PYG{l+s+s2}{bands}\PYG{l+s+s2}{\PYGZdq{}}\PYG{p}{:} \PYG{n}{np}\PYG{o}{.}\PYG{n}{array}\PYG{p}{(}\PYG{p}{[}\PYG{n}{bandsLow}\PYG{p}{,} \PYG{n}{bandsUp}\PYG{p}{]}\PYG{p}{)}\PYG{o}{.}\PYG{n}{T}
\PYG{g+gp}{... }        \PYG{p}{\PYGZcb{}}\PYG{p}{)}
\PYG{g+go}{Mesh generator. Ntheta = 30, Nphi = 50, R = 1000000.0}
\PYG{g+go}{Generating points...}
\PYG{g+go}{Generating nodes...}
\PYG{g+go}{triangles,rectangles generation completed. N\PYGZus{}triangles = 98, N\PYGZus{}rectangles = 1323}
\PYG{g+go}{Generating triangles...}
\PYG{g+go}{generating bin...}
\end{sphinxVerbatim}

\sphinxAtStartPar
Once the data have been provided, the method \sphinxcode{\sphinxupquote{htrdrPy.Data.writeInputs}}
will generate the input file in an \sphinxtitleref{input\_\{name\}} folder.

\begin{sphinxVerbatim}[commandchars=\\\{\}]
\PYG{g+gp}{\PYGZgt{}\PYGZgt{}\PYGZgt{} }\PYG{n}{d}\PYG{o}{.}\PYG{n}{writeInputs}\PYG{p}{(}\PYG{p}{)}
\PYG{g+go}{generating surface mesh bin file...}
\PYG{g+go}{Nnodes = 5586 Ncells = 2744 dim\PYGZus{}node = 3 dim\PYGZus{}cell = 3}
\PYG{g+go}{bin generation completed.}
\PYG{g+go}{generating surface properties bin file...}
\PYG{g+go}{bin generation completed.}
\PYG{g+go}{generating atmosphere mesh bin file...}
\PYG{g+go}{Nnodes = 547428 Ncells = 338541 dim\PYGZus{}node = 3 dim\PYGZus{}cell = 4}
\PYG{g+go}{bin generation completed.}
\PYG{g+go}{generating gas temperature bin file...}
\PYG{g+go}{bin generation completed.}
\PYG{g+go}{generating gas properties bin file...}
\PYG{g+go}{100\PYGZpc{}|\PYGZus{}\PYGZus{}\PYGZus{}\PYGZus{}\PYGZus{}\PYGZus{}\PYGZus{}\PYGZus{}\PYGZus{}\PYGZus{}\PYGZus{}\PYGZus{}\PYGZus{}\PYGZus{}\PYGZus{}\PYGZus{}\PYGZus{}\PYGZus{}\PYGZus{}\PYGZus{}\PYGZus{}\PYGZus{}\PYGZus{}\PYGZus{}\PYGZus{}\PYGZus{}\PYGZus{}\PYGZus{}\PYGZus{}\PYGZus{}\PYGZus{}\PYGZus{}\PYGZus{}\PYGZus{}\PYGZus{}\PYGZus{}\PYGZus{}\PYGZus{}\PYGZus{}\PYGZus{}\PYGZus{}\PYGZus{}\PYGZus{}\PYGZus{}\PYGZus{}\PYGZus{}\PYGZus{}\PYGZus{}\PYGZus{}\PYGZus{}\PYGZus{}\PYGZus{}\PYGZus{}\PYGZus{}\PYGZus{}\PYGZus{}\PYGZus{}\PYGZus{}\PYGZus{}\PYGZus{}\PYGZus{}\PYGZus{}\PYGZus{}\PYGZus{}\PYGZus{}\PYGZus{}\PYGZus{}\PYGZus{}\PYGZus{}\PYGZus{}\PYGZus{}\PYGZus{}\PYGZus{}\PYGZus{}\PYGZus{}\PYGZus{}\PYGZus{}\PYGZus{}\PYGZus{}\PYGZus{}\PYGZus{}\PYGZus{}\PYGZus{}\PYGZus{}\PYGZus{}\PYGZus{}\PYGZus{}\PYGZus{}\PYGZus{}\PYGZus{}\PYGZus{}\PYGZus{}\PYGZus{}\PYGZus{}\PYGZus{}\PYGZus{}\PYGZus{}\PYGZus{}\PYGZus{}\PYGZus{}\PYGZus{}\PYGZus{}\PYGZus{}\PYGZus{}\PYGZus{}\PYGZus{}\PYGZus{}\PYGZus{}\PYGZus{}\PYGZus{}\PYGZus{}\PYGZus{}\PYGZus{}\PYGZus{}\PYGZus{}\PYGZus{}\PYGZus{}\PYGZus{}\PYGZus{}\PYGZus{}\PYGZus{}\PYGZus{}\PYGZus{}\PYGZus{}\PYGZus{}\PYGZus{}\PYGZus{}\PYGZus{}\PYGZus{}\PYGZus{}\PYGZus{}\PYGZus{}\PYGZus{}\PYGZus{}\PYGZus{}\PYGZus{}\PYGZus{}\PYGZus{}\PYGZus{}\PYGZus{}\PYGZus{}\PYGZus{}\PYGZus{}\PYGZus{}\PYGZus{}\PYGZus{}\PYGZus{}\PYGZus{}\PYGZus{}\PYGZus{}\PYGZus{}\PYGZus{}\PYGZus{}| 20/20 [00:11\PYGZlt{}00:00,  1.69it/s]}
\PYG{g+go}{bin generation completed.}
\PYG{g+go}{generating haze properties bin file...}
\PYG{g+go}{100\PYGZpc{}|\PYGZus{}\PYGZus{}\PYGZus{}\PYGZus{}\PYGZus{}\PYGZus{}\PYGZus{}\PYGZus{}\PYGZus{}\PYGZus{}\PYGZus{}\PYGZus{}\PYGZus{}\PYGZus{}\PYGZus{}\PYGZus{}\PYGZus{}\PYGZus{}\PYGZus{}\PYGZus{}\PYGZus{}\PYGZus{}\PYGZus{}\PYGZus{}\PYGZus{}\PYGZus{}\PYGZus{}\PYGZus{}\PYGZus{}\PYGZus{}\PYGZus{}\PYGZus{}\PYGZus{}\PYGZus{}\PYGZus{}\PYGZus{}\PYGZus{}\PYGZus{}\PYGZus{}\PYGZus{}\PYGZus{}\PYGZus{}\PYGZus{}\PYGZus{}\PYGZus{}\PYGZus{}\PYGZus{}\PYGZus{}\PYGZus{}\PYGZus{}\PYGZus{}\PYGZus{}\PYGZus{}\PYGZus{}\PYGZus{}\PYGZus{}\PYGZus{}\PYGZus{}\PYGZus{}\PYGZus{}\PYGZus{}\PYGZus{}\PYGZus{}\PYGZus{}\PYGZus{}\PYGZus{}\PYGZus{}\PYGZus{}\PYGZus{}\PYGZus{}\PYGZus{}\PYGZus{}\PYGZus{}\PYGZus{}\PYGZus{}\PYGZus{}\PYGZus{}\PYGZus{}\PYGZus{}\PYGZus{}\PYGZus{}\PYGZus{}\PYGZus{}\PYGZus{}\PYGZus{}\PYGZus{}\PYGZus{}\PYGZus{}\PYGZus{}\PYGZus{}\PYGZus{}\PYGZus{}\PYGZus{}\PYGZus{}\PYGZus{}\PYGZus{}\PYGZus{}\PYGZus{}\PYGZus{}\PYGZus{}\PYGZus{}\PYGZus{}\PYGZus{}\PYGZus{}\PYGZus{}\PYGZus{}\PYGZus{}\PYGZus{}\PYGZus{}\PYGZus{}\PYGZus{}\PYGZus{}\PYGZus{}\PYGZus{}\PYGZus{}\PYGZus{}\PYGZus{}\PYGZus{}\PYGZus{}\PYGZus{}\PYGZus{}\PYGZus{}\PYGZus{}\PYGZus{}\PYGZus{}\PYGZus{}\PYGZus{}\PYGZus{}\PYGZus{}\PYGZus{}\PYGZus{}\PYGZus{}\PYGZus{}\PYGZus{}\PYGZus{}\PYGZus{}\PYGZus{}\PYGZus{}\PYGZus{}\PYGZus{}\PYGZus{}\PYGZus{}\PYGZus{}\PYGZus{}\PYGZus{}\PYGZus{}\PYGZus{}\PYGZus{}\PYGZus{}\PYGZus{}\PYGZus{}\PYGZus{}\PYGZus{}| 20/20 [00:04\PYGZlt{}00:00,  4.95it/s]}
\PYG{g+go}{bin generation completed.}
\PYG{g+go}{generating haze phase function bin file...}
\PYG{g+go}{bin generation completed.}
\end{sphinxVerbatim}
\begin{quote}\begin{description}
\sphinxlineitem{Parameters}\begin{itemize}
\item {} 
\sphinxAtStartPar
\sphinxstyleliteralstrong{\sphinxupquote{radius}} (\sphinxstyleliteralemphasis{\sphinxupquote{float}}) \textendash{} Radius of the planet {[}m{]}.

\item {} 
\sphinxAtStartPar
\sphinxstyleliteralstrong{\sphinxupquote{nTheta}} (\sphinxstyleliteralemphasis{\sphinxupquote{int}}\sphinxstyleliteralemphasis{\sphinxupquote{, }}\sphinxstyleliteralemphasis{\sphinxupquote{optional}}\sphinxstyleliteralemphasis{\sphinxupquote{, }}\sphinxstyleliteralemphasis{\sphinxupquote{not requested if meshes loaded from file}}) \textendash{} Number of latitude points in the range {[}0°,360°{]} to be used.

\item {} 
\sphinxAtStartPar
\sphinxstyleliteralstrong{\sphinxupquote{nPhi}} (\sphinxstyleliteralemphasis{\sphinxupquote{int}}\sphinxstyleliteralemphasis{\sphinxupquote{, }}\sphinxstyleliteralemphasis{\sphinxupquote{optional}}\sphinxstyleliteralemphasis{\sphinxupquote{, }}\sphinxstyleliteralemphasis{\sphinxupquote{not requested if meshes loaded from file}}) \textendash{} Number of latitude points in the range {[}\sphinxhyphen{}180°,180°{]}.

\item {} 
\sphinxAtStartPar
\sphinxstyleliteralstrong{\sphinxupquote{gravity}} (\sphinxstyleliteralemphasis{\sphinxupquote{float}}\sphinxstyleliteralemphasis{\sphinxupquote{, }}\sphinxstyleliteralemphasis{\sphinxupquote{optional}}\sphinxstyleliteralemphasis{\sphinxupquote{,}}) \textendash{} Gravity of the planet {[}m/s2{]}. Requested only for LMDZ\sphinxhyphen{}PCM
input/output files to extract the altitude grid from the
geopotential. Alternatively, the user can provide the planet mass.

\item {} 
\sphinxAtStartPar
\sphinxstyleliteralstrong{\sphinxupquote{mass}} (\sphinxstyleliteralemphasis{\sphinxupquote{float}}\sphinxstyleliteralemphasis{\sphinxupquote{, }}\sphinxstyleliteralemphasis{\sphinxupquote{optional}}\sphinxstyleliteralemphasis{\sphinxupquote{,}}) \textendash{} Mass of the planet {[}kg{]}. Requested only for LMDZ\sphinxhyphen{}PCM
input/output files to extract the altitude grid from the
geopotential. Alternatively, the user can provide the planet gravity.

\item {} 
\sphinxAtStartPar
\sphinxstyleliteralstrong{\sphinxupquote{name}} (str, optional, default uses a counter of the number of istances of \sphinxcode{\sphinxupquote{htrdrPy.Data}}) \textendash{} Name for the dataset, which will be used to name the input folders.

\end{itemize}

\end{description}\end{quote}
\index{cleanInputs() (htrdrPy.data.Data method)@\spxentry{cleanInputs()}\spxextra{htrdrPy.data.Data method}}

\begin{fulllineitems}
\phantomsection\label{\detokenize{htrdrPy.data:htrdrPy.data.Data.cleanInputs}}
\pysigstartsignatures
\pysiglinewithargsret
{\sphinxbfcode{\sphinxupquote{cleanInputs}}}
{}
{}
\pysigstopsignatures
\sphinxAtStartPar
Remove the inputs\_\{name\} folder.

\end{fulllineitems}

\index{makeAtmosphereFrom1D() (htrdrPy.data.Data method)@\spxentry{makeAtmosphereFrom1D()}\spxextra{htrdrPy.data.Data method}}

\begin{fulllineitems}
\phantomsection\label{\detokenize{htrdrPy.data:htrdrPy.data.Data.makeAtmosphereFrom1D}}
\pysigstartsignatures
\pysiglinewithargsret
{\sphinxbfcode{\sphinxupquote{makeAtmosphereFrom1D}}}
{\sphinxparam{\DUrole{n}{data}}}
{}
\pysigstopsignatures
\sphinxAtStartPar
Generate a spherical atmosphere from single column data. The data
are provided at the levels, i.e. at the interface between layers.
\begin{quote}\begin{description}
\sphinxlineitem{Parameters}
\sphinxAtStartPar
\sphinxstyleliteralstrong{\sphinxupquote{data}} (\sphinxstyleliteralemphasis{\sphinxupquote{dict}}) \textendash{} 
\sphinxAtStartPar
Dictionnary with the following items:
\begin{itemize}
\item {} \begin{description}
\sphinxlineitem{”nLevel”}{[}int{]}
\sphinxAtStartPar
Number of levels.

\end{description}

\item {} \begin{description}
\sphinxlineitem{”nCoeff”: int}
\sphinxAtStartPar
Number of quadrature points for the k\sphinxhyphen{}coeff.

\end{description}

\item {} \begin{description}
\sphinxlineitem{”nWavelengths”:  int}
\sphinxAtStartPar
Number of wavelengths.

\end{description}

\item {} \begin{description}
\sphinxlineitem{”nAngle”: int, optional}
\sphinxAtStartPar
Number of angles in the phase functions (not required if
using the builtin Henyey\sphinxhyphen{}Greenstein phase function).

\end{description}

\item {} \begin{description}
\sphinxlineitem{”weights”: \sphinxcode{\sphinxupquote{numpy.ndarray}}}
\sphinxAtStartPar
The weigths of the nCoeff quadrature points
(shape=(nWavelengths, nCoeff)).

\end{description}

\item {} \begin{description}
\sphinxlineitem{”altitude (m)”: \sphinxcode{\sphinxupquote{numpy.ndarray}}}
\sphinxAtStartPar
Array of altitudes (shape=(nLevel), {[}m{]}).

\end{description}

\item {} \begin{description}
\sphinxlineitem{”temperature (K)”: \sphinxcode{\sphinxupquote{numpy.ndarray}}}
\sphinxAtStartPar
Array of temperatures (shape=(nLevel), {[}K{]}).

\end{description}

\item {} \begin{description}
\sphinxlineitem{”scattering (m\sphinxhyphen{}1)”: \sphinxcode{\sphinxupquote{numpy.ndarray}}}
\sphinxAtStartPar
Array of scattering coefficients
(shape=(nWavelengths, nLevel, nCoeff), {[}m\sphinxhyphen{}1{]}).

\end{description}

\item {} \begin{description}
\sphinxlineitem{”absorption (m\sphinxhyphen{}1)”: \sphinxcode{\sphinxupquote{numpy.ndarray}}}
\sphinxAtStartPar
Array of absorption coefficients
(shape=(nWavelengths, nLevel, nCoeff), {[}m\sphinxhyphen{}1{]}).

\end{description}

\item {} \begin{description}
\sphinxlineitem{”angles (°)”: \sphinxcode{\sphinxupquote{numpy.ndarray}}, optional}
\sphinxAtStartPar
Array of angle values for the phase function
(shape=(nAngle), float {[}°{]}). Only required when providing
the \sphinxcode{\sphinxupquote{phaseFunc}}.

\end{description}

\item {} \begin{description}
\sphinxlineitem{”phaseFunc”: \sphinxcode{\sphinxupquote{numpy.ndarray}}, optional}
\sphinxAtStartPar
Array of discrete phase functions (shape = (nWavelengths,
nLevel, nAngle, nCoeff)). Alternatively, the user can
provide an array of asymmetry parameter.

\end{description}

\item {} \begin{description}
\sphinxlineitem{”asymmetry”: \sphinxcode{\sphinxupquote{numpy.ndarray}}, optional}
\sphinxAtStartPar
Array of asymmetry parameter (shape=(nWavelengths, nLevel,
nCoeff)).  Alternatively, the user can provide the discrete
phase function.

\end{description}

\item {} \begin{description}
\sphinxlineitem{”wavelength”: \sphinxcode{\sphinxupquote{numpy.ndarray}}, optional}
\sphinxAtStartPar
Array of wavelength (shape = (nWavelengths), {[}m{]}). The
values corresponds to the band center. The value is used of
the phase function.

\end{description}

\item {} \begin{description}
\sphinxlineitem{”band low”:  \sphinxcode{\sphinxupquote{numpy.ndarray}}}
\sphinxAtStartPar
Array of wavelength (shape = (nWavelengths), {[}m{]}). The
values corresponds to the lower boundary of the band.

\end{description}

\item {} \begin{description}
\sphinxlineitem{”band up”:  \sphinxcode{\sphinxupquote{numpy.ndarray}}}
\sphinxAtStartPar
Array of wavelength (shape = (nWavelengths), {[}m{]}). The
values corresponds to the lower boundary of the band.

\end{description}

\end{itemize}


\end{description}\end{quote}
\subsubsection*{Notes}

\sphinxAtStartPar
The nTheta and nPhi parameters are used to define the resolution of the
atmosphere mesh and are therefore mandatory.
\subsubsection*{Examples}

\begin{sphinxVerbatim}[commandchars=\\\{\}]
\PYG{g+gp}{\PYGZgt{}\PYGZgt{}\PYGZgt{} }\PYG{n}{d} \PYG{o}{=} \PYG{n}{htrdrPy}\PYG{o}{.}\PYG{n}{Data}\PYG{p}{(}\PYG{n}{radius}\PYG{o}{=}\PYG{l+m+mf}{1e6}\PYG{p}{,} \PYG{n}{nTheta}\PYG{o}{=}\PYG{l+m+mi}{30}\PYG{p}{,} \PYG{n}{nPhi}\PYG{o}{=}\PYG{l+m+mi}{50}\PYG{p}{,} \PYG{n}{name}\PYG{o}{=}\PYG{l+s+s2}{\PYGZdq{}}\PYG{l+s+s2}{Planet}\PYG{l+s+s2}{\PYGZdq{}}\PYG{p}{)}
\PYG{g+gp}{\PYGZgt{}\PYGZgt{}\PYGZgt{} }\PYG{n}{nLevel} \PYG{o}{=} \PYG{l+m+mi}{50}
\PYG{g+gp}{\PYGZgt{}\PYGZgt{}\PYGZgt{} }\PYG{n}{nCoeff} \PYG{o}{=} \PYG{l+m+mi}{4}
\PYG{g+gp}{\PYGZgt{}\PYGZgt{}\PYGZgt{} }\PYG{n}{nWavelengths} \PYG{o}{=} \PYG{l+m+mi}{20}
\PYG{g+gp}{\PYGZgt{}\PYGZgt{}\PYGZgt{} }\PYG{n}{weights} \PYG{o}{=} \PYG{n}{np}\PYG{o}{.}\PYG{n}{array}\PYG{p}{(}\PYG{n}{nWavelengths} \PYG{o}{*} \PYG{p}{[}\PYG{l+m+mf}{0.2}\PYG{p}{,} \PYG{l+m+mf}{0.3}\PYG{p}{,} \PYG{l+m+mf}{0.3}\PYG{p}{,} \PYG{l+m+mf}{0.2}\PYG{p}{]}\PYG{p}{)}\PYG{o}{.}\PYG{n}{reshape}\PYG{p}{(}\PYG{n}{nWavelengths}\PYG{p}{,} \PYG{n}{nCoeff}\PYG{p}{)}
\PYG{g+gp}{\PYGZgt{}\PYGZgt{}\PYGZgt{} }\PYG{n}{altitudes} \PYG{o}{=} \PYG{n}{np}\PYG{o}{.}\PYG{n}{linspace}\PYG{p}{(}\PYG{l+m+mi}{0}\PYG{p}{,} \PYG{l+m+mf}{5e5}\PYG{p}{,} \PYG{n}{nLevel}\PYG{p}{)}
\PYG{g+gp}{\PYGZgt{}\PYGZgt{}\PYGZgt{} }\PYG{n}{temperatures} \PYG{o}{=} \PYG{n}{np}\PYG{o}{.}\PYG{n}{linspace}\PYG{p}{(}\PYG{l+m+mi}{300}\PYG{p}{,} \PYG{l+m+mi}{500}\PYG{p}{,} \PYG{n}{nLevel}\PYG{p}{)}
\PYG{g+gp}{\PYGZgt{}\PYGZgt{}\PYGZgt{} }\PYG{n}{scatt} \PYG{o}{=} \PYG{n}{np}\PYG{o}{.}\PYG{n}{linspace}\PYG{p}{(}\PYG{l+m+mf}{1e\PYGZhy{}8}\PYG{p}{,} \PYG{l+m+mf}{1e\PYGZhy{}2}\PYG{p}{,}
\PYG{g+gp}{... }    \PYG{n}{nLevel}\PYG{o}{*}\PYG{n}{nCoeff}\PYG{o}{*}\PYG{n}{nWavelengths}\PYG{p}{)}\PYG{o}{.}\PYG{n}{reshape}\PYG{p}{(}\PYG{p}{(}\PYG{n}{nWavelengths}\PYG{p}{,} \PYG{n}{nLevel}\PYG{p}{,} \PYG{n}{nCoeff}\PYG{p}{)}\PYG{p}{)}
\PYG{g+gp}{\PYGZgt{}\PYGZgt{}\PYGZgt{} }\PYG{n}{absor} \PYG{o}{=} \PYG{n}{np}\PYG{o}{.}\PYG{n}{linspace}\PYG{p}{(}\PYG{l+m+mf}{1e\PYGZhy{}5}\PYG{p}{,} \PYG{l+m+mf}{1e\PYGZhy{}1}\PYG{p}{,}
\PYG{g+gp}{... }    \PYG{n}{nLevel}\PYG{o}{*}\PYG{n}{nCoeff}\PYG{o}{*}\PYG{n}{nWavelengths}\PYG{p}{)}\PYG{o}{.}\PYG{n}{reshape}\PYG{p}{(}\PYG{p}{(}\PYG{n}{nWavelengths}\PYG{p}{,} \PYG{n}{nLevel}\PYG{p}{,} \PYG{n}{nCoeff}\PYG{p}{)}\PYG{p}{)}
\PYG{g+gp}{\PYGZgt{}\PYGZgt{}\PYGZgt{} }\PYG{n}{asymm} \PYG{o}{=} \PYG{n}{np}\PYG{o}{.}\PYG{n}{linspace}\PYG{p}{(}\PYG{l+m+mi}{0}\PYG{p}{,} \PYG{l+m+mi}{1}\PYG{p}{,}
\PYG{g+gp}{... }    \PYG{n}{nLevel}\PYG{o}{*}\PYG{n}{nCoeff}\PYG{o}{*}\PYG{n}{nWavelengths}\PYG{p}{)}\PYG{o}{.}\PYG{n}{reshape}\PYG{p}{(}\PYG{p}{(}\PYG{n}{nWavelengths}\PYG{p}{,} \PYG{n}{nLevel}\PYG{p}{,} \PYG{n}{nCoeff}\PYG{p}{)}\PYG{p}{)}
\PYG{g+gp}{\PYGZgt{}\PYGZgt{}\PYGZgt{} }\PYG{n}{wavelengths} \PYG{o}{=} \PYG{n}{np}\PYG{o}{.}\PYG{n}{linspace}\PYG{p}{(}\PYG{l+m+mf}{2e7}\PYG{p}{,} \PYG{l+m+mf}{9e7}\PYG{p}{,} \PYG{n}{nWavelengths}\PYG{p}{)}
\PYG{g+gp}{\PYGZgt{}\PYGZgt{}\PYGZgt{} }\PYG{n}{bandsLow} \PYG{o}{=} \PYG{n}{np}\PYG{o}{.}\PYG{n}{zeros}\PYG{p}{(}\PYG{n}{nWavelengths}\PYG{p}{)}
\PYG{g+gp}{\PYGZgt{}\PYGZgt{}\PYGZgt{} }\PYG{n}{bandsUp} \PYG{o}{=} \PYG{n}{np}\PYG{o}{.}\PYG{n}{zeros}\PYG{p}{(}\PYG{n}{nWavelengths}\PYG{p}{)}
\PYG{g+gp}{\PYGZgt{}\PYGZgt{}\PYGZgt{} }\PYG{n}{bandsLow}\PYG{p}{[}\PYG{l+m+mi}{1}\PYG{p}{:}\PYG{p}{]} \PYG{o}{=} \PYG{n}{bandsUp}\PYG{p}{[}\PYG{p}{:}\PYG{o}{\PYGZhy{}}\PYG{l+m+mi}{1}\PYG{p}{]} \PYG{o}{=} \PYG{l+m+mf}{0.5} \PYG{o}{*} \PYG{p}{(}\PYG{n}{wavelengths}\PYG{p}{[}\PYG{l+m+mi}{1}\PYG{p}{:}\PYG{p}{]} \PYG{o}{+} \PYG{n}{wavelengths}\PYG{p}{[}\PYG{p}{:}\PYG{o}{\PYGZhy{}}\PYG{l+m+mi}{1}\PYG{p}{]}\PYG{p}{)}
\PYG{g+gp}{\PYGZgt{}\PYGZgt{}\PYGZgt{} }\PYG{n}{bandsLow}\PYG{p}{[}\PYG{l+m+mi}{0}\PYG{p}{]} \PYG{o}{=} \PYG{l+m+mf}{1.5} \PYG{o}{*} \PYG{n}{wavelengths}\PYG{p}{[}\PYG{l+m+mi}{0}\PYG{p}{]} \PYG{o}{\PYGZhy{}} \PYG{l+m+mf}{0.5} \PYG{o}{*} \PYG{n}{wavelengths}\PYG{p}{[}\PYG{l+m+mi}{1}\PYG{p}{]}
\PYG{g+gp}{\PYGZgt{}\PYGZgt{}\PYGZgt{} }\PYG{n}{bandsUp}\PYG{p}{[}\PYG{o}{\PYGZhy{}}\PYG{l+m+mi}{1}\PYG{p}{]} \PYG{o}{=}  \PYG{l+m+mf}{1.5} \PYG{o}{*} \PYG{n}{wavelengths}\PYG{p}{[}\PYG{o}{\PYGZhy{}}\PYG{l+m+mi}{1}\PYG{p}{]} \PYG{o}{\PYGZhy{}} \PYG{l+m+mf}{0.5} \PYG{o}{*} \PYG{n}{wavelengths}\PYG{p}{[}\PYG{o}{\PYGZhy{}}\PYG{l+m+mi}{2}\PYG{p}{]}
\PYG{g+gp}{\PYGZgt{}\PYGZgt{}\PYGZgt{} }\PYG{n}{d}\PYG{o}{.}\PYG{n}{makeAtmosphereFrom1D}\PYG{p}{(}\PYG{p}{\PYGZob{}}
\PYG{g+gp}{... }        \PYG{l+s+s2}{\PYGZdq{}}\PYG{l+s+s2}{nLevel}\PYG{l+s+s2}{\PYGZdq{}}\PYG{p}{:} \PYG{n}{nLevel}\PYG{p}{,}
\PYG{g+gp}{... }        \PYG{l+s+s2}{\PYGZdq{}}\PYG{l+s+s2}{nCoeff}\PYG{l+s+s2}{\PYGZdq{}}\PYG{p}{:} \PYG{n}{nCoeff}\PYG{p}{,}
\PYG{g+gp}{... }        \PYG{l+s+s2}{\PYGZdq{}}\PYG{l+s+s2}{nWavelengths}\PYG{l+s+s2}{\PYGZdq{}}\PYG{p}{:} \PYG{n}{nWavelengths}\PYG{p}{,}
\PYG{g+gp}{... }        \PYG{l+s+s2}{\PYGZdq{}}\PYG{l+s+s2}{weights}\PYG{l+s+s2}{\PYGZdq{}}\PYG{p}{:} \PYG{n}{weights}\PYG{p}{,}
\PYG{g+gp}{... }        \PYG{l+s+s2}{\PYGZdq{}}\PYG{l+s+s2}{altitude (m)}\PYG{l+s+s2}{\PYGZdq{}}\PYG{p}{:} \PYG{n}{altitudes}\PYG{p}{,}
\PYG{g+gp}{... }        \PYG{l+s+s2}{\PYGZdq{}}\PYG{l+s+s2}{temperature (K)}\PYG{l+s+s2}{\PYGZdq{}}\PYG{p}{:} \PYG{n}{temperatures}\PYG{p}{,}
\PYG{g+gp}{... }        \PYG{l+s+s2}{\PYGZdq{}}\PYG{l+s+s2}{scattering (m\PYGZhy{}1)}\PYG{l+s+s2}{\PYGZdq{}}\PYG{p}{:} \PYG{n}{scatt}\PYG{p}{,}
\PYG{g+gp}{... }        \PYG{l+s+s2}{\PYGZdq{}}\PYG{l+s+s2}{absorption (m\PYGZhy{}1)}\PYG{l+s+s2}{\PYGZdq{}}\PYG{p}{:} \PYG{n}{absor}\PYG{p}{,}
\PYG{g+gp}{... }        \PYG{l+s+s2}{\PYGZdq{}}\PYG{l+s+s2}{asymmetry}\PYG{l+s+s2}{\PYGZdq{}}\PYG{p}{:} \PYG{n}{asymm}\PYG{p}{,}
\PYG{g+gp}{... }        \PYG{l+s+s2}{\PYGZdq{}}\PYG{l+s+s2}{wavelength}\PYG{l+s+s2}{\PYGZdq{}}\PYG{p}{:} \PYG{n}{wavelengths}\PYG{p}{,}
\PYG{g+gp}{... }        \PYG{l+s+s2}{\PYGZdq{}}\PYG{l+s+s2}{band low}\PYG{l+s+s2}{\PYGZdq{}}\PYG{p}{:} \PYG{n}{bandsLow}\PYG{p}{,}
\PYG{g+gp}{... }        \PYG{l+s+s2}{\PYGZdq{}}\PYG{l+s+s2}{band up}\PYG{l+s+s2}{\PYGZdq{}}\PYG{p}{:} \PYG{n}{bandsUp}
\PYG{g+gp}{... }        \PYG{p}{\PYGZcb{}}\PYG{p}{)}
\PYG{g+go}{Mesh generator. Ntheta = 30, Nphi = 50, Nz = 50, r\PYGZus{}min = 1000000.0, r\PYGZus{}max = 1500000.0}
\PYG{g+go}{Generating points...}
\PYG{g+go}{Generating nodes...}
\PYG{g+go}{Hexahedron \PYGZam{} Octahedron generation completed. N\PYGZus{}Hexahedron = 4802, N\PYGZus{}Octahedron = 64827}
\PYG{g+go}{Generating Tetrahedrons...}
\PYG{g+go}{Assigning data to nodes ...}
\end{sphinxVerbatim}

\end{fulllineitems}

\index{makeAtmosphereFrom1D\_PP() (htrdrPy.data.Data method)@\spxentry{makeAtmosphereFrom1D\_PP()}\spxextra{htrdrPy.data.Data method}}

\begin{fulllineitems}
\phantomsection\label{\detokenize{htrdrPy.data:htrdrPy.data.Data.makeAtmosphereFrom1D_PP}}
\pysigstartsignatures
\pysiglinewithargsret
{\sphinxbfcode{\sphinxupquote{makeAtmosphereFrom1D\_PP}}}
{\sphinxparam{\DUrole{n}{data}}}
{}
\pysigstopsignatures
\sphinxAtStartPar
Generate a plan parallel atmosphere from single column data. The data
are provided at the levels, i.e. at the interface between layers.
\begin{quote}\begin{description}
\sphinxlineitem{Parameters}
\sphinxAtStartPar
\sphinxstyleliteralstrong{\sphinxupquote{data}} (\sphinxstyleliteralemphasis{\sphinxupquote{dict}}) \textendash{} 
\sphinxAtStartPar
Dictionnary with the following items:
\begin{itemize}
\item {} \begin{description}
\sphinxlineitem{”nLevel”}{[}int{]}
\sphinxAtStartPar
Number of levels.

\end{description}

\item {} \begin{description}
\sphinxlineitem{”nCoeff”: int}
\sphinxAtStartPar
Number of quadrature points for the k\sphinxhyphen{}coeff.

\end{description}

\item {} \begin{description}
\sphinxlineitem{”nWavelengths”:  int}
\sphinxAtStartPar
Number of wavelengths.

\end{description}

\item {} \begin{description}
\sphinxlineitem{”nAngle”: int, optional}
\sphinxAtStartPar
Number of angles in the phase functions (not required if
using the builtin Henyey\sphinxhyphen{}Greenstein phase function).

\end{description}

\item {} \begin{description}
\sphinxlineitem{”weights”: \sphinxcode{\sphinxupquote{numpy.ndarray}}}
\sphinxAtStartPar
The weigths of the nCoeff quadrature points
(shape=(nWavelengths, nCoeff)).

\end{description}

\item {} \begin{description}
\sphinxlineitem{”altitude (m)”: \sphinxcode{\sphinxupquote{numpy.ndarray}}}
\sphinxAtStartPar
Array of altitudes (shape=(nLevel), {[}m{]}).

\end{description}

\item {} \begin{description}
\sphinxlineitem{”temperature (K)”: \sphinxcode{\sphinxupquote{numpy.ndarray}}}
\sphinxAtStartPar
Array of temperatures (shape=(nLevel), {[}K{]}).

\end{description}

\item {} \begin{description}
\sphinxlineitem{”scattering (m\sphinxhyphen{}1)”: \sphinxcode{\sphinxupquote{numpy.ndarray}}}
\sphinxAtStartPar
Array of scattering coefficients
(shape=(nWavelengths, nLevel, nCoeff), {[}m\sphinxhyphen{}1{]}).

\end{description}

\item {} \begin{description}
\sphinxlineitem{”absorption (m\sphinxhyphen{}1)”: \sphinxcode{\sphinxupquote{numpy.ndarray}}}
\sphinxAtStartPar
Array of absorption coefficients
(shape=(nWavelengths, nLevel, nCoeff), {[}m\sphinxhyphen{}1{]}).

\end{description}

\item {} \begin{description}
\sphinxlineitem{”angles (°)”: \sphinxcode{\sphinxupquote{numpy.ndarray}}, optional}
\sphinxAtStartPar
Array of angle values for the phase function
(shape=(nAngle), float {[}°{]}). Only required when providing
the \sphinxcode{\sphinxupquote{phaseFunc}}.

\end{description}

\item {} \begin{description}
\sphinxlineitem{”phaseFunc”: \sphinxcode{\sphinxupquote{numpy.ndarray}}, optional}
\sphinxAtStartPar
Array of discrete phase functions (shape = (nWavelengths,
nLevel, nAngle, nCoeff)). Alternatively, the user can
provide an array of asymmetry parameter.

\end{description}

\item {} \begin{description}
\sphinxlineitem{”asymmetry”: \sphinxcode{\sphinxupquote{numpy.ndarray}}, optional}
\sphinxAtStartPar
Array of asymmetry parameter (shape=(nWavelengths, nLevel,
nCoeff)).  Alternatively, the user can provide the discrete
phase function.

\end{description}

\item {} \begin{description}
\sphinxlineitem{”wavelength”: \sphinxcode{\sphinxupquote{numpy.ndarray}}, optional}
\sphinxAtStartPar
Array of wavelength (shape = (nWavelengths), {[}m{]}). The
values corresponds to the band center. The value is used of
the phase function.

\end{description}

\item {} \begin{description}
\sphinxlineitem{”band low”:  \sphinxcode{\sphinxupquote{numpy.ndarray}}}
\sphinxAtStartPar
Array of wavelength (shape = (nWavelengths), {[}m{]}). The
values corresponds to the lower boundary of the band.

\end{description}

\item {} \begin{description}
\sphinxlineitem{”band up”:  \sphinxcode{\sphinxupquote{numpy.ndarray}}}
\sphinxAtStartPar
Array of wavelength (shape = (nWavelengths), {[}m{]}). The
values corresponds to the lower boundary of the band.

\end{description}

\end{itemize}


\end{description}\end{quote}
\subsubsection*{Notes}

\sphinxAtStartPar
In plan\sphinxhyphen{}parallel mode, the \sphinxcode{\sphinxupquote{radius}} parameter passed at the
initialisation of the instance is used as the horizontal expansion of
the ground. Make sure to use a large enough value. Also, the nTheta and
nPhi parameters are not used.
\subsubsection*{Examples}

\begin{sphinxVerbatim}[commandchars=\\\{\}]
\PYG{g+gp}{\PYGZgt{}\PYGZgt{}\PYGZgt{} }\PYG{n}{d} \PYG{o}{=} \PYG{n}{htrdrPy}\PYG{o}{.}\PYG{n}{Data}\PYG{p}{(}\PYG{n}{radius}\PYG{o}{=}\PYG{l+m+mf}{1e6}\PYG{p}{,} \PYG{n}{nTheta}\PYG{o}{=}\PYG{l+m+mi}{30}\PYG{p}{,} \PYG{n}{nPhi}\PYG{o}{=}\PYG{l+m+mi}{50}\PYG{p}{,} \PYG{n}{name}\PYG{o}{=}\PYG{l+s+s2}{\PYGZdq{}}\PYG{l+s+s2}{Planet}\PYG{l+s+s2}{\PYGZdq{}}\PYG{p}{)}
\PYG{g+gp}{\PYGZgt{}\PYGZgt{}\PYGZgt{} }\PYG{n}{nLevel} \PYG{o}{=} \PYG{l+m+mi}{50}
\PYG{g+gp}{\PYGZgt{}\PYGZgt{}\PYGZgt{} }\PYG{n}{nCoeff} \PYG{o}{=} \PYG{l+m+mi}{4}
\PYG{g+gp}{\PYGZgt{}\PYGZgt{}\PYGZgt{} }\PYG{n}{nWavelengths} \PYG{o}{=} \PYG{l+m+mi}{20}
\PYG{g+gp}{\PYGZgt{}\PYGZgt{}\PYGZgt{} }\PYG{n}{weights} \PYG{o}{=} \PYG{n}{np}\PYG{o}{.}\PYG{n}{array}\PYG{p}{(}\PYG{n}{nWavelengths} \PYG{o}{*} \PYG{p}{[}\PYG{l+m+mf}{0.2}\PYG{p}{,} \PYG{l+m+mf}{0.3}\PYG{p}{,} \PYG{l+m+mf}{0.3}\PYG{p}{,} \PYG{l+m+mf}{0.2}\PYG{p}{]}\PYG{p}{)}\PYG{o}{.}\PYG{n}{reshape}\PYG{p}{(}\PYG{n}{nWavelengths}\PYG{p}{,} \PYG{n}{nCoeff}\PYG{p}{)}
\PYG{g+gp}{\PYGZgt{}\PYGZgt{}\PYGZgt{} }\PYG{n}{altitudes} \PYG{o}{=} \PYG{n}{np}\PYG{o}{.}\PYG{n}{linspace}\PYG{p}{(}\PYG{l+m+mi}{0}\PYG{p}{,} \PYG{l+m+mf}{5e5}\PYG{p}{,} \PYG{n}{nLevel}\PYG{p}{)}
\PYG{g+gp}{\PYGZgt{}\PYGZgt{}\PYGZgt{} }\PYG{n}{temperatures} \PYG{o}{=} \PYG{n}{np}\PYG{o}{.}\PYG{n}{linspace}\PYG{p}{(}\PYG{l+m+mi}{300}\PYG{p}{,} \PYG{l+m+mi}{500}\PYG{p}{,} \PYG{n}{nLevel}\PYG{p}{)}
\PYG{g+gp}{\PYGZgt{}\PYGZgt{}\PYGZgt{} }\PYG{n}{scatt} \PYG{o}{=} \PYG{n}{np}\PYG{o}{.}\PYG{n}{linspace}\PYG{p}{(}\PYG{l+m+mf}{1e\PYGZhy{}8}\PYG{p}{,} \PYG{l+m+mf}{1e\PYGZhy{}2}\PYG{p}{,}
\PYG{g+gp}{... }    \PYG{n}{nLevel}\PYG{o}{*}\PYG{n}{nCoeff}\PYG{o}{*}\PYG{n}{nWavelengths}\PYG{p}{)}\PYG{o}{.}\PYG{n}{reshape}\PYG{p}{(}\PYG{p}{(}\PYG{n}{nWavelengths}\PYG{p}{,} \PYG{n}{nLevel}\PYG{p}{,} \PYG{n}{nCoeff}\PYG{p}{)}\PYG{p}{)}
\PYG{g+gp}{\PYGZgt{}\PYGZgt{}\PYGZgt{} }\PYG{n}{absor} \PYG{o}{=} \PYG{n}{np}\PYG{o}{.}\PYG{n}{linspace}\PYG{p}{(}\PYG{l+m+mf}{1e\PYGZhy{}5}\PYG{p}{,} \PYG{l+m+mf}{1e\PYGZhy{}1}\PYG{p}{,}
\PYG{g+gp}{... }    \PYG{n}{nLevel}\PYG{o}{*}\PYG{n}{nCoeff}\PYG{o}{*}\PYG{n}{nWavelengths}\PYG{p}{)}\PYG{o}{.}\PYG{n}{reshape}\PYG{p}{(}\PYG{p}{(}\PYG{n}{nWavelengths}\PYG{p}{,} \PYG{n}{nLevel}\PYG{p}{,} \PYG{n}{nCoeff}\PYG{p}{)}\PYG{p}{)}
\PYG{g+gp}{\PYGZgt{}\PYGZgt{}\PYGZgt{} }\PYG{n}{asymm} \PYG{o}{=} \PYG{n}{np}\PYG{o}{.}\PYG{n}{linspace}\PYG{p}{(}\PYG{l+m+mi}{0}\PYG{p}{,} \PYG{l+m+mi}{1}\PYG{p}{,}
\PYG{g+gp}{... }    \PYG{n}{nLevel}\PYG{o}{*}\PYG{n}{nCoeff}\PYG{o}{*}\PYG{n}{nWavelengths}\PYG{p}{)}\PYG{o}{.}\PYG{n}{reshape}\PYG{p}{(}\PYG{p}{(}\PYG{n}{nWavelengths}\PYG{p}{,} \PYG{n}{nLevel}\PYG{p}{,} \PYG{n}{nCoeff}\PYG{p}{)}\PYG{p}{)}
\PYG{g+gp}{\PYGZgt{}\PYGZgt{}\PYGZgt{} }\PYG{n}{wavelengths} \PYG{o}{=} \PYG{n}{np}\PYG{o}{.}\PYG{n}{linspace}\PYG{p}{(}\PYG{l+m+mf}{2e7}\PYG{p}{,} \PYG{l+m+mf}{9e7}\PYG{p}{,} \PYG{n}{nWavelengths}\PYG{p}{)}
\PYG{g+gp}{\PYGZgt{}\PYGZgt{}\PYGZgt{} }\PYG{n}{bandsLow} \PYG{o}{=} \PYG{n}{np}\PYG{o}{.}\PYG{n}{zeros}\PYG{p}{(}\PYG{n}{nWavelengths}\PYG{p}{)}
\PYG{g+gp}{\PYGZgt{}\PYGZgt{}\PYGZgt{} }\PYG{n}{bandsUp} \PYG{o}{=} \PYG{n}{np}\PYG{o}{.}\PYG{n}{zeros}\PYG{p}{(}\PYG{n}{nWavelengths}\PYG{p}{)}
\PYG{g+gp}{\PYGZgt{}\PYGZgt{}\PYGZgt{} }\PYG{n}{bandsLow}\PYG{p}{[}\PYG{l+m+mi}{1}\PYG{p}{:}\PYG{p}{]} \PYG{o}{=} \PYG{n}{bandsUp}\PYG{p}{[}\PYG{p}{:}\PYG{o}{\PYGZhy{}}\PYG{l+m+mi}{1}\PYG{p}{]} \PYG{o}{=} \PYG{l+m+mf}{0.5} \PYG{o}{*} \PYG{p}{(}\PYG{n}{wavelengths}\PYG{p}{[}\PYG{l+m+mi}{1}\PYG{p}{:}\PYG{p}{]} \PYG{o}{+} \PYG{n}{wavelengths}\PYG{p}{[}\PYG{p}{:}\PYG{o}{\PYGZhy{}}\PYG{l+m+mi}{1}\PYG{p}{]}\PYG{p}{)}
\PYG{g+gp}{\PYGZgt{}\PYGZgt{}\PYGZgt{} }\PYG{n}{bandsLow}\PYG{p}{[}\PYG{l+m+mi}{0}\PYG{p}{]} \PYG{o}{=} \PYG{l+m+mf}{1.5} \PYG{o}{*} \PYG{n}{wavelengths}\PYG{p}{[}\PYG{l+m+mi}{0}\PYG{p}{]} \PYG{o}{\PYGZhy{}} \PYG{l+m+mf}{0.5} \PYG{o}{*} \PYG{n}{wavelengths}\PYG{p}{[}\PYG{l+m+mi}{1}\PYG{p}{]}
\PYG{g+gp}{\PYGZgt{}\PYGZgt{}\PYGZgt{} }\PYG{n}{bandsUp}\PYG{p}{[}\PYG{o}{\PYGZhy{}}\PYG{l+m+mi}{1}\PYG{p}{]} \PYG{o}{=}  \PYG{l+m+mf}{1.5} \PYG{o}{*} \PYG{n}{wavelengths}\PYG{p}{[}\PYG{o}{\PYGZhy{}}\PYG{l+m+mi}{1}\PYG{p}{]} \PYG{o}{\PYGZhy{}} \PYG{l+m+mf}{0.5} \PYG{o}{*} \PYG{n}{wavelengths}\PYG{p}{[}\PYG{o}{\PYGZhy{}}\PYG{l+m+mi}{2}\PYG{p}{]}
\PYG{g+gp}{\PYGZgt{}\PYGZgt{}\PYGZgt{} }\PYG{n}{d}\PYG{o}{.}\PYG{n}{makeAtmosphereFrom1D\PYGZus{}PP}\PYG{p}{(}\PYG{p}{\PYGZob{}}
\PYG{g+gp}{... }        \PYG{l+s+s2}{\PYGZdq{}}\PYG{l+s+s2}{nLevel}\PYG{l+s+s2}{\PYGZdq{}}\PYG{p}{:} \PYG{n}{nLevel}\PYG{p}{,}
\PYG{g+gp}{... }        \PYG{l+s+s2}{\PYGZdq{}}\PYG{l+s+s2}{nCoeff}\PYG{l+s+s2}{\PYGZdq{}}\PYG{p}{:} \PYG{n}{nCoeff}\PYG{p}{,}
\PYG{g+gp}{... }        \PYG{l+s+s2}{\PYGZdq{}}\PYG{l+s+s2}{nWavelengths}\PYG{l+s+s2}{\PYGZdq{}}\PYG{p}{:} \PYG{n}{nWavelengths}\PYG{p}{,}
\PYG{g+gp}{... }        \PYG{l+s+s2}{\PYGZdq{}}\PYG{l+s+s2}{weights}\PYG{l+s+s2}{\PYGZdq{}}\PYG{p}{:} \PYG{n}{weights}\PYG{p}{,}
\PYG{g+gp}{... }        \PYG{l+s+s2}{\PYGZdq{}}\PYG{l+s+s2}{altitude (m)}\PYG{l+s+s2}{\PYGZdq{}}\PYG{p}{:} \PYG{n}{altitudes}\PYG{p}{,}
\PYG{g+gp}{... }        \PYG{l+s+s2}{\PYGZdq{}}\PYG{l+s+s2}{temperature (K)}\PYG{l+s+s2}{\PYGZdq{}}\PYG{p}{:} \PYG{n}{temperatures}\PYG{p}{,}
\PYG{g+gp}{... }        \PYG{l+s+s2}{\PYGZdq{}}\PYG{l+s+s2}{scattering (m\PYGZhy{}1)}\PYG{l+s+s2}{\PYGZdq{}}\PYG{p}{:} \PYG{n}{scatt}\PYG{p}{,}
\PYG{g+gp}{... }        \PYG{l+s+s2}{\PYGZdq{}}\PYG{l+s+s2}{absorption (m\PYGZhy{}1)}\PYG{l+s+s2}{\PYGZdq{}}\PYG{p}{:} \PYG{n}{absor}\PYG{p}{,}
\PYG{g+gp}{... }        \PYG{l+s+s2}{\PYGZdq{}}\PYG{l+s+s2}{asymmetry}\PYG{l+s+s2}{\PYGZdq{}}\PYG{p}{:} \PYG{n}{asymm}\PYG{p}{,}
\PYG{g+gp}{... }        \PYG{l+s+s2}{\PYGZdq{}}\PYG{l+s+s2}{wavelength}\PYG{l+s+s2}{\PYGZdq{}}\PYG{p}{:} \PYG{n}{wavelengths}\PYG{p}{,}
\PYG{g+gp}{... }        \PYG{l+s+s2}{\PYGZdq{}}\PYG{l+s+s2}{band low}\PYG{l+s+s2}{\PYGZdq{}}\PYG{p}{:} \PYG{n}{bandsLow}\PYG{p}{,}
\PYG{g+gp}{... }        \PYG{l+s+s2}{\PYGZdq{}}\PYG{l+s+s2}{band up}\PYG{l+s+s2}{\PYGZdq{}}\PYG{p}{:} \PYG{n}{bandsUp}
\PYG{g+gp}{... }        \PYG{p}{\PYGZcb{}}\PYG{p}{)}
\PYG{g+go}{Assigning data to nodes ...}
\end{sphinxVerbatim}

\end{fulllineitems}

\index{makeAtmosphereFrom2D() (htrdrPy.data.Data method)@\spxentry{makeAtmosphereFrom2D()}\spxextra{htrdrPy.data.Data method}}

\begin{fulllineitems}
\phantomsection\label{\detokenize{htrdrPy.data:htrdrPy.data.Data.makeAtmosphereFrom2D}}
\pysigstartsignatures
\pysiglinewithargsret
{\sphinxbfcode{\sphinxupquote{makeAtmosphereFrom2D}}}
{\sphinxparam{\DUrole{n}{data}}}
{}
\pysigstopsignatures
\sphinxAtStartPar
Generate a spherical atmosphere from single column data. The data
are provided at the levels, i.e. at the interface between layers.
\begin{quote}\begin{description}
\sphinxlineitem{Parameters}
\sphinxAtStartPar
\sphinxstyleliteralstrong{\sphinxupquote{data}} (\sphinxstyleliteralemphasis{\sphinxupquote{dict}}) \textendash{} 
\sphinxAtStartPar
Dictionnary with the following items:
\begin{itemize}
\item {} \begin{description}
\sphinxlineitem{”nLevel”}{[}int{]}
\sphinxAtStartPar
Number of levels.

\end{description}

\item {} \begin{description}
\sphinxlineitem{”nLat”}{[}int{]}
\sphinxAtStartPar
Number of latitudes.

\end{description}

\item {} \begin{description}
\sphinxlineitem{”nCoeff”: int}
\sphinxAtStartPar
Number of quadrature points for the k\sphinxhyphen{}coeff.

\end{description}

\item {} \begin{description}
\sphinxlineitem{”nWavelengths”:  int}
\sphinxAtStartPar
Number of wavelengths.

\end{description}

\item {} \begin{description}
\sphinxlineitem{”nAngle”: int, optional}
\sphinxAtStartPar
Number of angles in the phase functions (not required if
using the builtin Henyey\sphinxhyphen{}Greenstein phase function).

\end{description}

\item {} \begin{description}
\sphinxlineitem{”weights”: \sphinxcode{\sphinxupquote{numpy.ndarray}}}
\sphinxAtStartPar
The weigths of the nCoeff quadrature points
(shape=(nWavelengths, nCoeff)).

\end{description}

\item {} \begin{description}
\sphinxlineitem{”altitude (m)”: \sphinxcode{\sphinxupquote{numpy.ndarray}}}
\sphinxAtStartPar
Array of altitudes (shape=(nLevel, nLat), {[}m{]}).

\end{description}

\item {} \begin{description}
\sphinxlineitem{”latitude (°)”}{[}\sphinxcode{\sphinxupquote{numpy.ndarray}}{]}
\sphinxAtStartPar
List of latitudes (shape=(nLat), {[}°{]}).

\end{description}

\item {} \begin{description}
\sphinxlineitem{”temperature (K)”: \sphinxcode{\sphinxupquote{numpy.ndarray}}}
\sphinxAtStartPar
Array of temperatures (shape=(nLevel, nLat), {[}K{]}).

\end{description}

\item {} \begin{description}
\sphinxlineitem{”scattering (m\sphinxhyphen{}1)”: \sphinxcode{\sphinxupquote{numpy.ndarray}}}
\sphinxAtStartPar
Array of scattering coefficients
(shape=(nWavelengths, nLevel, nLat, nCoeff), {[}m\sphinxhyphen{}1{]}).

\end{description}

\item {} \begin{description}
\sphinxlineitem{”absorption (m\sphinxhyphen{}1)”: \sphinxcode{\sphinxupquote{numpy.ndarray}}}
\sphinxAtStartPar
Array of absorption coefficients
(shape=(nWavelengths, nLevel, nLat, nCoeff), {[}m\sphinxhyphen{}1{]}).

\end{description}

\item {} \begin{description}
\sphinxlineitem{”angles (°)”: \sphinxcode{\sphinxupquote{numpy.ndarray}}, optional}
\sphinxAtStartPar
Array of angle values for the phase function
(shape=(nAngle), float {[}°{]}). Only required when providing
the \sphinxcode{\sphinxupquote{phaseFunc}}.

\end{description}

\item {} \begin{description}
\sphinxlineitem{”phaseFunc”: \sphinxcode{\sphinxupquote{numpy.ndarray}}, optional}
\sphinxAtStartPar
Array of discrete phase functions (shape = (nWavelengths,
nLevel, nLat, nAngle, nCoeff)). Alternatively, the user can
provide an array of asymmetry parameter.

\end{description}

\item {} \begin{description}
\sphinxlineitem{”asymmetry”: \sphinxcode{\sphinxupquote{numpy.ndarray}}, optional}
\sphinxAtStartPar
Array of asymmetry parameter (shape=(nWavelengths, nLevel,
nLat, nCoeff)).  Alternatively, the user can provide the
discrete phase function.

\end{description}

\item {} \begin{description}
\sphinxlineitem{”wavelength”: \sphinxcode{\sphinxupquote{numpy.ndarray}}, optional}
\sphinxAtStartPar
Array of wavelength (shape = (nWavelengths), {[}m{]}). The
values corresponds to the band center. The value is used of
the phase function.

\end{description}

\item {} \begin{description}
\sphinxlineitem{”band low”:  \sphinxcode{\sphinxupquote{numpy.ndarray}}}
\sphinxAtStartPar
Array of wavelength (shape = (nWavelengths), {[}m{]}). The
values corresponds to the lower boundary of the band.

\end{description}

\item {} \begin{description}
\sphinxlineitem{”band up”:  \sphinxcode{\sphinxupquote{numpy.ndarray}}}
\sphinxAtStartPar
Array of wavelength (shape = (nWavelengths), {[}m{]}). The
values corresponds to the lower boundary of the band.

\end{description}

\end{itemize}


\end{description}\end{quote}
\subsubsection*{Notes}

\sphinxAtStartPar
The nPhi parameter provided at the initialisation of the
instance is used to define the longitudinal resolution of the
atmospheric mesh and is therefore manatory. The nTheta parameter is
obtained from the length of the \sphinxcode{\sphinxupquote{latitude}} provided.
\subsubsection*{Examples}

\begin{sphinxVerbatim}[commandchars=\\\{\}]
\PYG{g+gp}{\PYGZgt{}\PYGZgt{}\PYGZgt{} }\PYG{n}{d} \PYG{o}{=} \PYG{n}{htrdrPy}\PYG{o}{.}\PYG{n}{Data}\PYG{p}{(}\PYG{n}{radius}\PYG{o}{=}\PYG{l+m+mf}{1e6}\PYG{p}{,} \PYG{n}{nPhi}\PYG{o}{=}\PYG{l+m+mi}{50}\PYG{p}{,} \PYG{n}{name}\PYG{o}{=}\PYG{l+s+s2}{\PYGZdq{}}\PYG{l+s+s2}{Planet}\PYG{l+s+s2}{\PYGZdq{}}\PYG{p}{)}
\PYG{g+gp}{\PYGZgt{}\PYGZgt{}\PYGZgt{} }\PYG{n}{nLevel} \PYG{o}{=} \PYG{l+m+mi}{50}
\PYG{g+gp}{\PYGZgt{}\PYGZgt{}\PYGZgt{} }\PYG{n}{nLat} \PYG{o}{=} \PYG{l+m+mi}{30}
\PYG{g+gp}{\PYGZgt{}\PYGZgt{}\PYGZgt{} }\PYG{n}{nCoeff} \PYG{o}{=} \PYG{l+m+mi}{4}
\PYG{g+gp}{\PYGZgt{}\PYGZgt{}\PYGZgt{} }\PYG{n}{nWavelengths} \PYG{o}{=} \PYG{l+m+mi}{20}
\PYG{g+gp}{\PYGZgt{}\PYGZgt{}\PYGZgt{} }\PYG{n}{weights} \PYG{o}{=} \PYG{n}{np}\PYG{o}{.}\PYG{n}{array}\PYG{p}{(}\PYG{n}{nWavelengths} \PYG{o}{*} \PYG{p}{[}\PYG{l+m+mf}{0.2}\PYG{p}{,} \PYG{l+m+mf}{0.3}\PYG{p}{,} \PYG{l+m+mf}{0.3}\PYG{p}{,} \PYG{l+m+mf}{0.2}\PYG{p}{]}\PYG{p}{)}\PYG{o}{.}\PYG{n}{reshape}\PYG{p}{(}\PYG{n}{nWavelengths}\PYG{p}{,} \PYG{n}{nCoeff}\PYG{p}{)}
\PYG{g+gp}{\PYGZgt{}\PYGZgt{}\PYGZgt{} }\PYG{n}{altitudes} \PYG{o}{=} \PYG{n}{np}\PYG{o}{.}\PYG{n}{tile}\PYG{p}{(}\PYG{n}{np}\PYG{o}{.}\PYG{n}{linspace}\PYG{p}{(}\PYG{l+m+mi}{0}\PYG{p}{,} \PYG{l+m+mf}{5e5}\PYG{p}{,} \PYG{n}{nLevel}\PYG{p}{)}\PYG{p}{,} \PYG{p}{(}\PYG{n}{nLat}\PYG{p}{,} \PYG{l+m+mi}{1}\PYG{p}{)}\PYG{p}{)}\PYG{o}{.}\PYG{n}{T}
\PYG{g+gp}{\PYGZgt{}\PYGZgt{}\PYGZgt{} }\PYG{n}{latitudes} \PYG{o}{=} \PYG{n}{np}\PYG{o}{.}\PYG{n}{linspace}\PYG{p}{(}\PYG{o}{\PYGZhy{}}\PYG{l+m+mi}{90}\PYG{p}{,} \PYG{l+m+mi}{90}\PYG{p}{,} \PYG{n}{nLat}\PYG{p}{)}
\PYG{g+gp}{\PYGZgt{}\PYGZgt{}\PYGZgt{} }\PYG{n}{temperatures} \PYG{o}{=} \PYG{n}{np}\PYG{o}{.}\PYG{n}{linspace}\PYG{p}{(}\PYG{l+m+mi}{300}\PYG{p}{,} \PYG{l+m+mi}{500}\PYG{p}{,} \PYG{n}{nLevel}\PYG{o}{*}\PYG{n}{nLat}\PYG{p}{)}\PYG{o}{.}\PYG{n}{reshape}\PYG{p}{(}\PYG{n}{nLevel}\PYG{p}{,}\PYG{n}{nLat}\PYG{p}{)}
\PYG{g+gp}{\PYGZgt{}\PYGZgt{}\PYGZgt{} }\PYG{n}{scatt} \PYG{o}{=} \PYG{n}{np}\PYG{o}{.}\PYG{n}{linspace}\PYG{p}{(}\PYG{l+m+mf}{1e\PYGZhy{}8}\PYG{p}{,} \PYG{l+m+mf}{1e\PYGZhy{}2}\PYG{p}{,}
\PYG{g+gp}{... }    \PYG{n}{nLevel}\PYG{o}{*}\PYG{n}{nLat}\PYG{o}{*}\PYG{n}{nCoeff}\PYG{o}{*}\PYG{n}{nWavelengths}\PYG{p}{)}\PYG{o}{.}\PYG{n}{reshape}\PYG{p}{(}\PYG{p}{(}\PYG{n}{nWavelengths}\PYG{p}{,} \PYG{n}{nLevel}\PYG{p}{,}
\PYG{g+gp}{... }                                                \PYG{n}{nLat}\PYG{p}{,} \PYG{n}{nCoeff}\PYG{p}{)}\PYG{p}{)}
\PYG{g+gp}{\PYGZgt{}\PYGZgt{}\PYGZgt{} }\PYG{n}{absor} \PYG{o}{=} \PYG{n}{np}\PYG{o}{.}\PYG{n}{linspace}\PYG{p}{(}\PYG{l+m+mf}{1e\PYGZhy{}5}\PYG{p}{,} \PYG{l+m+mf}{1e\PYGZhy{}1}\PYG{p}{,}
\PYG{g+gp}{... }    \PYG{n}{nLevel}\PYG{o}{*}\PYG{n}{nLat}\PYG{o}{*}\PYG{n}{nCoeff}\PYG{o}{*}\PYG{n}{nWavelengths}\PYG{p}{)}\PYG{o}{.}\PYG{n}{reshape}\PYG{p}{(}\PYG{p}{(}\PYG{n}{nWavelengths}\PYG{p}{,} \PYG{n}{nLevel}\PYG{p}{,}
\PYG{g+gp}{... }                                                \PYG{n}{nLat}\PYG{p}{,} \PYG{n}{nCoeff}\PYG{p}{)}\PYG{p}{)}
\PYG{g+gp}{\PYGZgt{}\PYGZgt{}\PYGZgt{} }\PYG{n}{asymm} \PYG{o}{=} \PYG{n}{np}\PYG{o}{.}\PYG{n}{linspace}\PYG{p}{(}\PYG{l+m+mi}{0}\PYG{p}{,} \PYG{l+m+mi}{1}\PYG{p}{,}
\PYG{g+gp}{... }    \PYG{n}{nLevel}\PYG{o}{*}\PYG{n}{nLat}\PYG{o}{*}\PYG{n}{nCoeff}\PYG{o}{*}\PYG{n}{nWavelengths}\PYG{p}{)}\PYG{o}{.}\PYG{n}{reshape}\PYG{p}{(}\PYG{p}{(}\PYG{n}{nWavelengths}\PYG{p}{,} \PYG{n}{nLevel}\PYG{p}{,}
\PYG{g+gp}{... }                                                \PYG{n}{nLat}\PYG{p}{,} \PYG{n}{nCoeff}\PYG{p}{)}\PYG{p}{)}
\PYG{g+gp}{\PYGZgt{}\PYGZgt{}\PYGZgt{} }\PYG{n}{wavelengths} \PYG{o}{=} \PYG{n}{np}\PYG{o}{.}\PYG{n}{linspace}\PYG{p}{(}\PYG{l+m+mf}{2e7}\PYG{p}{,} \PYG{l+m+mf}{9e7}\PYG{p}{,} \PYG{n}{nWavelengths}\PYG{p}{)}
\PYG{g+gp}{\PYGZgt{}\PYGZgt{}\PYGZgt{} }\PYG{n}{bandsLow} \PYG{o}{=} \PYG{n}{np}\PYG{o}{.}\PYG{n}{zeros}\PYG{p}{(}\PYG{n}{nWavelengths}\PYG{p}{)}
\PYG{g+gp}{\PYGZgt{}\PYGZgt{}\PYGZgt{} }\PYG{n}{bandsUp} \PYG{o}{=} \PYG{n}{np}\PYG{o}{.}\PYG{n}{zeros}\PYG{p}{(}\PYG{n}{nWavelengths}\PYG{p}{)}
\PYG{g+gp}{\PYGZgt{}\PYGZgt{}\PYGZgt{} }\PYG{n}{bandsLow}\PYG{p}{[}\PYG{l+m+mi}{1}\PYG{p}{:}\PYG{p}{]} \PYG{o}{=} \PYG{n}{bandsUp}\PYG{p}{[}\PYG{p}{:}\PYG{o}{\PYGZhy{}}\PYG{l+m+mi}{1}\PYG{p}{]} \PYG{o}{=} \PYG{l+m+mf}{0.5} \PYG{o}{*} \PYG{p}{(}\PYG{n}{wavelengths}\PYG{p}{[}\PYG{l+m+mi}{1}\PYG{p}{:}\PYG{p}{]} \PYG{o}{+} \PYG{n}{wavelengths}\PYG{p}{[}\PYG{p}{:}\PYG{o}{\PYGZhy{}}\PYG{l+m+mi}{1}\PYG{p}{]}\PYG{p}{)}
\PYG{g+gp}{\PYGZgt{}\PYGZgt{}\PYGZgt{} }\PYG{n}{bandsLow}\PYG{p}{[}\PYG{l+m+mi}{0}\PYG{p}{]} \PYG{o}{=} \PYG{l+m+mf}{1.5} \PYG{o}{*} \PYG{n}{wavelengths}\PYG{p}{[}\PYG{l+m+mi}{0}\PYG{p}{]} \PYG{o}{\PYGZhy{}} \PYG{l+m+mf}{0.5} \PYG{o}{*} \PYG{n}{wavelengths}\PYG{p}{[}\PYG{l+m+mi}{1}\PYG{p}{]}
\PYG{g+gp}{\PYGZgt{}\PYGZgt{}\PYGZgt{} }\PYG{n}{bandsUp}\PYG{p}{[}\PYG{o}{\PYGZhy{}}\PYG{l+m+mi}{1}\PYG{p}{]} \PYG{o}{=}  \PYG{l+m+mf}{1.5} \PYG{o}{*} \PYG{n}{wavelengths}\PYG{p}{[}\PYG{o}{\PYGZhy{}}\PYG{l+m+mi}{1}\PYG{p}{]} \PYG{o}{\PYGZhy{}} \PYG{l+m+mf}{0.5} \PYG{o}{*} \PYG{n}{wavelengths}\PYG{p}{[}\PYG{o}{\PYGZhy{}}\PYG{l+m+mi}{2}\PYG{p}{]}
\PYG{g+gp}{\PYGZgt{}\PYGZgt{}\PYGZgt{} }\PYG{n}{d}\PYG{o}{.}\PYG{n}{makeAtmosphereFrom2D}\PYG{p}{(}\PYG{p}{\PYGZob{}}
\PYG{g+gp}{... }        \PYG{l+s+s2}{\PYGZdq{}}\PYG{l+s+s2}{nLevel}\PYG{l+s+s2}{\PYGZdq{}}\PYG{p}{:} \PYG{n}{nLevel}\PYG{p}{,}
\PYG{g+gp}{... }        \PYG{l+s+s2}{\PYGZdq{}}\PYG{l+s+s2}{nLat}\PYG{l+s+s2}{\PYGZdq{}}\PYG{p}{:} \PYG{n}{nLat}\PYG{p}{,}
\PYG{g+gp}{... }        \PYG{l+s+s2}{\PYGZdq{}}\PYG{l+s+s2}{nCoeff}\PYG{l+s+s2}{\PYGZdq{}}\PYG{p}{:} \PYG{n}{nCoeff}\PYG{p}{,}
\PYG{g+gp}{... }        \PYG{l+s+s2}{\PYGZdq{}}\PYG{l+s+s2}{nWavelengths}\PYG{l+s+s2}{\PYGZdq{}}\PYG{p}{:} \PYG{n}{nWavelengths}\PYG{p}{,}
\PYG{g+gp}{... }        \PYG{l+s+s2}{\PYGZdq{}}\PYG{l+s+s2}{weights}\PYG{l+s+s2}{\PYGZdq{}}\PYG{p}{:} \PYG{n}{weights}\PYG{p}{,}
\PYG{g+gp}{... }        \PYG{l+s+s2}{\PYGZdq{}}\PYG{l+s+s2}{altitude (m)}\PYG{l+s+s2}{\PYGZdq{}}\PYG{p}{:} \PYG{n}{altitudes}\PYG{p}{,}
\PYG{g+gp}{... }        \PYG{l+s+s2}{\PYGZdq{}}\PYG{l+s+s2}{latitude (°)}\PYG{l+s+s2}{\PYGZdq{}}\PYG{p}{:} \PYG{n}{latitudes}\PYG{p}{,}
\PYG{g+gp}{... }        \PYG{l+s+s2}{\PYGZdq{}}\PYG{l+s+s2}{temperature (K)}\PYG{l+s+s2}{\PYGZdq{}}\PYG{p}{:} \PYG{n}{temperatures}\PYG{p}{,}
\PYG{g+gp}{... }        \PYG{l+s+s2}{\PYGZdq{}}\PYG{l+s+s2}{scattering (m\PYGZhy{}1)}\PYG{l+s+s2}{\PYGZdq{}}\PYG{p}{:} \PYG{n}{scatt}\PYG{p}{,}
\PYG{g+gp}{... }        \PYG{l+s+s2}{\PYGZdq{}}\PYG{l+s+s2}{absorption (m\PYGZhy{}1)}\PYG{l+s+s2}{\PYGZdq{}}\PYG{p}{:} \PYG{n}{absor}\PYG{p}{,}
\PYG{g+gp}{... }        \PYG{l+s+s2}{\PYGZdq{}}\PYG{l+s+s2}{asymmetry}\PYG{l+s+s2}{\PYGZdq{}}\PYG{p}{:} \PYG{n}{asymm}\PYG{p}{,}
\PYG{g+gp}{... }        \PYG{l+s+s2}{\PYGZdq{}}\PYG{l+s+s2}{wavelength}\PYG{l+s+s2}{\PYGZdq{}}\PYG{p}{:} \PYG{n}{wavelengths}\PYG{p}{,}
\PYG{g+gp}{... }        \PYG{l+s+s2}{\PYGZdq{}}\PYG{l+s+s2}{band low}\PYG{l+s+s2}{\PYGZdq{}}\PYG{p}{:} \PYG{n}{bandsLow}\PYG{p}{,}
\PYG{g+gp}{... }        \PYG{l+s+s2}{\PYGZdq{}}\PYG{l+s+s2}{band up}\PYG{l+s+s2}{\PYGZdq{}}\PYG{p}{:} \PYG{n}{bandsUp}
\PYG{g+gp}{... }        \PYG{p}{\PYGZcb{}}\PYG{p}{)}
\end{sphinxVerbatim}

\end{fulllineitems}

\index{makeAtmosphereFrom3D() (htrdrPy.data.Data method)@\spxentry{makeAtmosphereFrom3D()}\spxextra{htrdrPy.data.Data method}}

\begin{fulllineitems}
\phantomsection\label{\detokenize{htrdrPy.data:htrdrPy.data.Data.makeAtmosphereFrom3D}}
\pysigstartsignatures
\pysiglinewithargsret
{\sphinxbfcode{\sphinxupquote{makeAtmosphereFrom3D}}}
{\sphinxparam{\DUrole{n}{data}}}
{}
\pysigstopsignatures
\sphinxAtStartPar
Generate a spherical atmosphere from single column data. The data
are provided at the levels, i.e. at the interface between layers.
\begin{quote}\begin{description}
\sphinxlineitem{Parameters}
\sphinxAtStartPar
\sphinxstyleliteralstrong{\sphinxupquote{data}} (\sphinxstyleliteralemphasis{\sphinxupquote{dict}}) \textendash{} 
\sphinxAtStartPar
Dictionnary with the following items:
\begin{itemize}
\item {} \begin{description}
\sphinxlineitem{”nLevel”}{[}int{]}
\sphinxAtStartPar
Number of levels.

\end{description}

\item {} \begin{description}
\sphinxlineitem{”nLat”}{[}int{]}
\sphinxAtStartPar
Number of latitudes.

\end{description}

\item {} \begin{description}
\sphinxlineitem{”nLon”}{[}int{]}
\sphinxAtStartPar
Number of longitudes.

\end{description}

\item {} \begin{description}
\sphinxlineitem{”nCoeff”: int}
\sphinxAtStartPar
Number of quadrature points for the k\sphinxhyphen{}coeff.

\end{description}

\item {} \begin{description}
\sphinxlineitem{”nWavelengths”:  int}
\sphinxAtStartPar
Number of wavelengths.

\end{description}

\item {} \begin{description}
\sphinxlineitem{”nAngle”: int, optional}
\sphinxAtStartPar
Number of angles in the phase functions (not required if
using the builtin Henyey\sphinxhyphen{}Greenstein phase function).

\end{description}

\item {} \begin{description}
\sphinxlineitem{”weights”: \sphinxcode{\sphinxupquote{numpy.ndarray}}}
\sphinxAtStartPar
The weigths of the nCoeff quadrature points
(shape=(nWavelengths, nCoeff)).

\end{description}

\item {} \begin{description}
\sphinxlineitem{”altitude (m)”: \sphinxcode{\sphinxupquote{numpy.ndarray}}}
\sphinxAtStartPar
Array of altitudes (shape=(nLevel, nLat, nLon), {[}m{]}).

\end{description}

\item {} \begin{description}
\sphinxlineitem{”latitude (°)”}{[}\sphinxcode{\sphinxupquote{numpy.ndarray}}{]}
\sphinxAtStartPar
List of latitudes (shape=(nLat), {[}°{]}).

\end{description}

\item {} \begin{description}
\sphinxlineitem{”longitude (°)”}{[}\sphinxcode{\sphinxupquote{numpy.ndarray}}{]}
\sphinxAtStartPar
List of longitudes (shape=(nLon), {[}°{]}).

\end{description}

\item {} \begin{description}
\sphinxlineitem{”temperature (K)”: \sphinxcode{\sphinxupquote{numpy.ndarray}}}
\sphinxAtStartPar
Array of temperatures (shape=(nLevel, nLat, nLon), {[}K{]}).

\end{description}

\item {} \begin{description}
\sphinxlineitem{”scattering (m\sphinxhyphen{}1)”: \sphinxcode{\sphinxupquote{numpy.ndarray}}}
\sphinxAtStartPar
Array of scattering coefficients
(shape=(nWavelengths, nLevel, nLat, nLon, nCoeff), {[}m\sphinxhyphen{}1{]}).

\end{description}

\item {} \begin{description}
\sphinxlineitem{”absorption (m\sphinxhyphen{}1)”: \sphinxcode{\sphinxupquote{numpy.ndarray}}}
\sphinxAtStartPar
Array of absorption coefficients
(shape=(nWavelengths, nLevel, nLat, nLon, nCoeff), {[}m\sphinxhyphen{}1{]}).

\end{description}

\item {} \begin{description}
\sphinxlineitem{”angles (°)”: \sphinxcode{\sphinxupquote{numpy.ndarray}}, optional}
\sphinxAtStartPar
Array of angle values for the phase function
(shape=(nAngle), float {[}°{]}). Only required when providing
the \sphinxcode{\sphinxupquote{phaseFunc}}.

\end{description}

\item {} \begin{description}
\sphinxlineitem{”phaseFunc”: \sphinxcode{\sphinxupquote{numpy.ndarray}}, optional}
\sphinxAtStartPar
Array of discrete phase functions (shape = (nWavelengths,
nLevel, nLat, nLon, nAngle, nCoeff)). Alternatively, the
user can provide an array of asymmetry parameter.

\end{description}

\item {} \begin{description}
\sphinxlineitem{”asymmetry”: \sphinxcode{\sphinxupquote{numpy.ndarray}}, optional}
\sphinxAtStartPar
Array of asymmetry parameter (shape=(nWavelengths, nLevel,
nLat, nLon, nCoeff)).  Alternatively, the user can provide
the discrete phase function.

\end{description}

\item {} \begin{description}
\sphinxlineitem{”wavelength”: \sphinxcode{\sphinxupquote{numpy.ndarray}}, optional}
\sphinxAtStartPar
Array of wavelength (shape = (nWavelengths), {[}m{]}). The
values corresponds to the band center. The value is used of
the phase function.

\end{description}

\item {} \begin{description}
\sphinxlineitem{”band low”:  \sphinxcode{\sphinxupquote{numpy.ndarray}}}
\sphinxAtStartPar
Array of wavelength (shape = (nWavelengths), {[}m{]}). The
values corresponds to the lower boundary of the band.

\end{description}

\item {} \begin{description}
\sphinxlineitem{”band up”:  \sphinxcode{\sphinxupquote{numpy.ndarray}}}
\sphinxAtStartPar
Array of wavelength (shape = (nWavelengths), {[}m{]}). The
values corresponds to the lower boundary of the band.

\end{description}

\end{itemize}


\end{description}\end{quote}
\subsubsection*{Notes}

\sphinxAtStartPar
The nTheta and nPhi parameters provided at the initialisation of the
instance are not used and instead are defined from the length of the
\sphinxcode{\sphinxupquote{latitude (°)}} and \sphinxcode{\sphinxupquote{longitude (°)}}, respectively.
\subsubsection*{Examples}

\begin{sphinxVerbatim}[commandchars=\\\{\}]
\PYG{g+gp}{\PYGZgt{}\PYGZgt{}\PYGZgt{} }\PYG{n}{d} \PYG{o}{=} \PYG{n}{htrdrPy}\PYG{o}{.}\PYG{n}{Data}\PYG{p}{(}\PYG{n}{radius}\PYG{o}{=}\PYG{l+m+mf}{1e6}\PYG{p}{,} \PYG{n}{name}\PYG{o}{=}\PYG{l+s+s2}{\PYGZdq{}}\PYG{l+s+s2}{Planet}\PYG{l+s+s2}{\PYGZdq{}}\PYG{p}{)}
\PYG{g+gp}{\PYGZgt{}\PYGZgt{}\PYGZgt{} }\PYG{n}{nLevel} \PYG{o}{=} \PYG{l+m+mi}{50}
\PYG{g+gp}{\PYGZgt{}\PYGZgt{}\PYGZgt{} }\PYG{n}{nLat} \PYG{o}{=} \PYG{l+m+mi}{30}
\PYG{g+gp}{\PYGZgt{}\PYGZgt{}\PYGZgt{} }\PYG{n}{nLon} \PYG{o}{=} \PYG{l+m+mi}{50}
\PYG{g+gp}{\PYGZgt{}\PYGZgt{}\PYGZgt{} }\PYG{n}{nCoeff} \PYG{o}{=} \PYG{l+m+mi}{4}
\PYG{g+gp}{\PYGZgt{}\PYGZgt{}\PYGZgt{} }\PYG{n}{nWavelengths} \PYG{o}{=} \PYG{l+m+mi}{20}
\PYG{g+gp}{\PYGZgt{}\PYGZgt{}\PYGZgt{} }\PYG{n}{weights} \PYG{o}{=} \PYG{n}{np}\PYG{o}{.}\PYG{n}{array}\PYG{p}{(}\PYG{n}{nWavelengths} \PYG{o}{*} \PYG{p}{[}\PYG{l+m+mf}{0.2}\PYG{p}{,} \PYG{l+m+mf}{0.3}\PYG{p}{,} \PYG{l+m+mf}{0.3}\PYG{p}{,} \PYG{l+m+mf}{0.2}\PYG{p}{]}\PYG{p}{)}\PYG{o}{.}\PYG{n}{reshape}\PYG{p}{(}\PYG{n}{nWavelengths}\PYG{p}{,} \PYG{n}{nCoeff}\PYG{p}{)}
\PYG{g+gp}{\PYGZgt{}\PYGZgt{}\PYGZgt{} }\PYG{n}{altitudes} \PYG{o}{=} \PYG{n}{np}\PYG{o}{.}\PYG{n}{moveaxis}\PYG{p}{(}\PYG{n}{np}\PYG{o}{.}\PYG{n}{tile}\PYG{p}{(}\PYG{n}{np}\PYG{o}{.}\PYG{n}{linspace}\PYG{p}{(}\PYG{l+m+mi}{0}\PYG{p}{,} \PYG{l+m+mf}{5e5}\PYG{p}{,} \PYG{n}{nLevel}\PYG{p}{)}\PYG{p}{,}
\PYG{g+gp}{... }                                \PYG{p}{(}\PYG{n}{nLat}\PYG{p}{,} \PYG{n}{nLon}\PYG{p}{,} \PYG{l+m+mi}{1}\PYG{p}{)}\PYG{p}{)}\PYG{p}{,} \PYG{o}{\PYGZhy{}}\PYG{l+m+mi}{1}\PYG{p}{,} \PYG{l+m+mi}{0}\PYG{p}{)}
\PYG{g+gp}{\PYGZgt{}\PYGZgt{}\PYGZgt{} }\PYG{n}{latitudes} \PYG{o}{=} \PYG{n}{np}\PYG{o}{.}\PYG{n}{linspace}\PYG{p}{(}\PYG{o}{\PYGZhy{}}\PYG{l+m+mi}{90}\PYG{p}{,} \PYG{l+m+mi}{90}\PYG{p}{,} \PYG{n}{nLat}\PYG{p}{)}
\PYG{g+gp}{\PYGZgt{}\PYGZgt{}\PYGZgt{} }\PYG{n}{longitudes} \PYG{o}{=} \PYG{n}{np}\PYG{o}{.}\PYG{n}{linspace}\PYG{p}{(}\PYG{o}{\PYGZhy{}}\PYG{l+m+mi}{180}\PYG{p}{,} \PYG{l+m+mi}{180}\PYG{p}{,} \PYG{n}{nLon}\PYG{p}{)}
\PYG{g+gp}{\PYGZgt{}\PYGZgt{}\PYGZgt{} }\PYG{n}{temperatures} \PYG{o}{=} \PYG{n}{np}\PYG{o}{.}\PYG{n}{linspace}\PYG{p}{(}\PYG{l+m+mi}{300}\PYG{p}{,} \PYG{l+m+mi}{500}\PYG{p}{,} \PYG{n}{nLevel}\PYG{o}{*}\PYG{n}{nLat}\PYG{o}{*}\PYG{n}{nLon}\PYG{p}{)}\PYG{o}{.}\PYG{n}{reshape}\PYG{p}{(}\PYG{n}{nLevel}\PYG{p}{,}\PYG{n}{nLat}\PYG{p}{,} \PYG{n}{nLon}\PYG{p}{)}
\PYG{g+gp}{\PYGZgt{}\PYGZgt{}\PYGZgt{} }\PYG{n}{scatt} \PYG{o}{=} \PYG{n}{np}\PYG{o}{.}\PYG{n}{linspace}\PYG{p}{(}\PYG{l+m+mf}{1e\PYGZhy{}8}\PYG{p}{,} \PYG{l+m+mf}{1e\PYGZhy{}2}\PYG{p}{,}
\PYG{g+gp}{... }    \PYG{n}{nLevel}\PYG{o}{*}\PYG{n}{nLat}\PYG{o}{*}\PYG{n}{nLon}\PYG{o}{*}\PYG{n}{nCoeff}\PYG{o}{*}\PYG{n}{nWavelengths}\PYG{p}{)}\PYG{o}{.}\PYG{n}{reshape}\PYG{p}{(}\PYG{p}{(}\PYG{n}{nWavelengths}\PYG{p}{,} \PYG{n}{nLevel}\PYG{p}{,}
\PYG{g+gp}{... }                                                \PYG{n}{nLat}\PYG{p}{,} \PYG{n}{nLon}\PYG{p}{,} \PYG{n}{nCoeff}\PYG{p}{)}\PYG{p}{)}
\PYG{g+gp}{\PYGZgt{}\PYGZgt{}\PYGZgt{} }\PYG{n}{absor} \PYG{o}{=} \PYG{n}{np}\PYG{o}{.}\PYG{n}{linspace}\PYG{p}{(}\PYG{l+m+mf}{1e\PYGZhy{}5}\PYG{p}{,} \PYG{l+m+mf}{1e\PYGZhy{}1}\PYG{p}{,}
\PYG{g+gp}{... }    \PYG{n}{nLevel}\PYG{o}{*}\PYG{n}{nLat}\PYG{o}{*}\PYG{n}{nLon}\PYG{o}{*}\PYG{n}{nCoeff}\PYG{o}{*}\PYG{n}{nWavelengths}\PYG{p}{)}\PYG{o}{.}\PYG{n}{reshape}\PYG{p}{(}\PYG{p}{(}\PYG{n}{nWavelengths}\PYG{p}{,} \PYG{n}{nLevel}\PYG{p}{,}
\PYG{g+gp}{... }                                                \PYG{n}{nLat}\PYG{p}{,} \PYG{n}{nLon}\PYG{p}{,} \PYG{n}{nCoeff}\PYG{p}{)}\PYG{p}{)}
\PYG{g+gp}{\PYGZgt{}\PYGZgt{}\PYGZgt{} }\PYG{n}{asymm} \PYG{o}{=} \PYG{n}{np}\PYG{o}{.}\PYG{n}{linspace}\PYG{p}{(}\PYG{l+m+mi}{0}\PYG{p}{,} \PYG{l+m+mi}{1}\PYG{p}{,}
\PYG{g+gp}{... }    \PYG{n}{nLevel}\PYG{o}{*}\PYG{n}{nLat}\PYG{o}{*}\PYG{n}{nLon}\PYG{o}{*}\PYG{n}{nCoeff}\PYG{o}{*}\PYG{n}{nWavelengths}\PYG{p}{)}\PYG{o}{.}\PYG{n}{reshape}\PYG{p}{(}\PYG{p}{(}\PYG{n}{nWavelengths}\PYG{p}{,} \PYG{n}{nLevel}\PYG{p}{,}
\PYG{g+gp}{... }                                                \PYG{n}{nLat}\PYG{p}{,} \PYG{n}{nLon}\PYG{p}{,} \PYG{n}{nCoeff}\PYG{p}{)}\PYG{p}{)}
\PYG{g+gp}{\PYGZgt{}\PYGZgt{}\PYGZgt{} }\PYG{n}{wavelengths} \PYG{o}{=} \PYG{n}{np}\PYG{o}{.}\PYG{n}{linspace}\PYG{p}{(}\PYG{l+m+mf}{2e7}\PYG{p}{,} \PYG{l+m+mf}{9e7}\PYG{p}{,} \PYG{n}{nWavelengths}\PYG{p}{)}
\PYG{g+gp}{\PYGZgt{}\PYGZgt{}\PYGZgt{} }\PYG{n}{bandsLow} \PYG{o}{=} \PYG{n}{np}\PYG{o}{.}\PYG{n}{zeros}\PYG{p}{(}\PYG{n}{nWavelengths}\PYG{p}{)}
\PYG{g+gp}{\PYGZgt{}\PYGZgt{}\PYGZgt{} }\PYG{n}{bandsUp} \PYG{o}{=} \PYG{n}{np}\PYG{o}{.}\PYG{n}{zeros}\PYG{p}{(}\PYG{n}{nWavelengths}\PYG{p}{)}
\PYG{g+gp}{\PYGZgt{}\PYGZgt{}\PYGZgt{} }\PYG{n}{bandsLow}\PYG{p}{[}\PYG{l+m+mi}{1}\PYG{p}{:}\PYG{p}{]} \PYG{o}{=} \PYG{n}{bandsUp}\PYG{p}{[}\PYG{p}{:}\PYG{o}{\PYGZhy{}}\PYG{l+m+mi}{1}\PYG{p}{]} \PYG{o}{=} \PYG{l+m+mf}{0.5} \PYG{o}{*} \PYG{p}{(}\PYG{n}{wavelengths}\PYG{p}{[}\PYG{l+m+mi}{1}\PYG{p}{:}\PYG{p}{]} \PYG{o}{+} \PYG{n}{wavelengths}\PYG{p}{[}\PYG{p}{:}\PYG{o}{\PYGZhy{}}\PYG{l+m+mi}{1}\PYG{p}{]}\PYG{p}{)}
\PYG{g+gp}{\PYGZgt{}\PYGZgt{}\PYGZgt{} }\PYG{n}{bandsLow}\PYG{p}{[}\PYG{l+m+mi}{0}\PYG{p}{]} \PYG{o}{=} \PYG{l+m+mf}{1.5} \PYG{o}{*} \PYG{n}{wavelengths}\PYG{p}{[}\PYG{l+m+mi}{0}\PYG{p}{]} \PYG{o}{\PYGZhy{}} \PYG{l+m+mf}{0.5} \PYG{o}{*} \PYG{n}{wavelengths}\PYG{p}{[}\PYG{l+m+mi}{1}\PYG{p}{]}
\PYG{g+gp}{\PYGZgt{}\PYGZgt{}\PYGZgt{} }\PYG{n}{bandsUp}\PYG{p}{[}\PYG{o}{\PYGZhy{}}\PYG{l+m+mi}{1}\PYG{p}{]} \PYG{o}{=}  \PYG{l+m+mf}{1.5} \PYG{o}{*} \PYG{n}{wavelengths}\PYG{p}{[}\PYG{o}{\PYGZhy{}}\PYG{l+m+mi}{1}\PYG{p}{]} \PYG{o}{\PYGZhy{}} \PYG{l+m+mf}{0.5} \PYG{o}{*} \PYG{n}{wavelengths}\PYG{p}{[}\PYG{o}{\PYGZhy{}}\PYG{l+m+mi}{2}\PYG{p}{]}
\PYG{g+gp}{\PYGZgt{}\PYGZgt{}\PYGZgt{} }\PYG{n}{d}\PYG{o}{.}\PYG{n}{makeAtmosphereFrom3D}\PYG{p}{(}\PYG{p}{\PYGZob{}}
\PYG{g+gp}{... }        \PYG{l+s+s2}{\PYGZdq{}}\PYG{l+s+s2}{nLevel}\PYG{l+s+s2}{\PYGZdq{}}\PYG{p}{:} \PYG{n}{nLevel}\PYG{p}{,}
\PYG{g+gp}{... }        \PYG{l+s+s2}{\PYGZdq{}}\PYG{l+s+s2}{nLat}\PYG{l+s+s2}{\PYGZdq{}}\PYG{p}{:} \PYG{n}{nLat}\PYG{p}{,}
\PYG{g+gp}{... }        \PYG{l+s+s2}{\PYGZdq{}}\PYG{l+s+s2}{nLon}\PYG{l+s+s2}{\PYGZdq{}}\PYG{p}{:} \PYG{n}{nLon}\PYG{p}{,}
\PYG{g+gp}{... }        \PYG{l+s+s2}{\PYGZdq{}}\PYG{l+s+s2}{nCoeff}\PYG{l+s+s2}{\PYGZdq{}}\PYG{p}{:} \PYG{n}{nCoeff}\PYG{p}{,}
\PYG{g+gp}{... }        \PYG{l+s+s2}{\PYGZdq{}}\PYG{l+s+s2}{nWavelengths}\PYG{l+s+s2}{\PYGZdq{}}\PYG{p}{:} \PYG{n}{nWavelengths}\PYG{p}{,}
\PYG{g+gp}{... }        \PYG{l+s+s2}{\PYGZdq{}}\PYG{l+s+s2}{weights}\PYG{l+s+s2}{\PYGZdq{}}\PYG{p}{:} \PYG{n}{weights}\PYG{p}{,}
\PYG{g+gp}{... }        \PYG{l+s+s2}{\PYGZdq{}}\PYG{l+s+s2}{altitude (m)}\PYG{l+s+s2}{\PYGZdq{}}\PYG{p}{:} \PYG{n}{altitudes}\PYG{p}{,}
\PYG{g+gp}{... }        \PYG{l+s+s2}{\PYGZdq{}}\PYG{l+s+s2}{latitude (°)}\PYG{l+s+s2}{\PYGZdq{}}\PYG{p}{:} \PYG{n}{latitudes}\PYG{p}{,}
\PYG{g+gp}{... }        \PYG{l+s+s2}{\PYGZdq{}}\PYG{l+s+s2}{longitude (°)}\PYG{l+s+s2}{\PYGZdq{}}\PYG{p}{:} \PYG{n}{longitudes}\PYG{p}{,}
\PYG{g+gp}{... }        \PYG{l+s+s2}{\PYGZdq{}}\PYG{l+s+s2}{temperature (K)}\PYG{l+s+s2}{\PYGZdq{}}\PYG{p}{:} \PYG{n}{temperatures}\PYG{p}{,}
\PYG{g+gp}{... }        \PYG{l+s+s2}{\PYGZdq{}}\PYG{l+s+s2}{scattering (m\PYGZhy{}1)}\PYG{l+s+s2}{\PYGZdq{}}\PYG{p}{:} \PYG{n}{scatt}\PYG{p}{,}
\PYG{g+gp}{... }        \PYG{l+s+s2}{\PYGZdq{}}\PYG{l+s+s2}{absorption (m\PYGZhy{}1)}\PYG{l+s+s2}{\PYGZdq{}}\PYG{p}{:} \PYG{n}{absor}\PYG{p}{,}
\PYG{g+gp}{... }        \PYG{l+s+s2}{\PYGZdq{}}\PYG{l+s+s2}{asymmetry}\PYG{l+s+s2}{\PYGZdq{}}\PYG{p}{:} \PYG{n}{asymm}\PYG{p}{,}
\PYG{g+gp}{... }        \PYG{l+s+s2}{\PYGZdq{}}\PYG{l+s+s2}{wavelength}\PYG{l+s+s2}{\PYGZdq{}}\PYG{p}{:} \PYG{n}{wavelengths}\PYG{p}{,}
\PYG{g+gp}{... }        \PYG{l+s+s2}{\PYGZdq{}}\PYG{l+s+s2}{band low}\PYG{l+s+s2}{\PYGZdq{}}\PYG{p}{:} \PYG{n}{bandsLow}\PYG{p}{,}
\PYG{g+gp}{... }        \PYG{l+s+s2}{\PYGZdq{}}\PYG{l+s+s2}{band up}\PYG{l+s+s2}{\PYGZdq{}}\PYG{p}{:} \PYG{n}{bandsUp}
\PYG{g+gp}{... }        \PYG{p}{\PYGZcb{}}\PYG{p}{)}
\end{sphinxVerbatim}

\end{fulllineitems}

\index{makeFromLMDZ() (htrdrPy.data.Data method)@\spxentry{makeFromLMDZ()}\spxextra{htrdrPy.data.Data method}}

\begin{fulllineitems}
\phantomsection\label{\detokenize{htrdrPy.data:htrdrPy.data.Data.makeFromLMDZ}}
\pysigstartsignatures
\pysiglinewithargsret
{\sphinxbfcode{\sphinxupquote{makeFromLMDZ}}}
{\sphinxparam{\DUrole{n}{LMDZinput}}\sphinxparamcomma \sphinxparam{\DUrole{n}{LMDZouput}}\sphinxparamcomma \sphinxparam{\DUrole{n}{weights=None}}\sphinxparamcomma \sphinxparam{\DUrole{n}{keys=\{\textquotesingle{}altitude\textquotesingle{}: \textquotesingle{}altitude\textquotesingle{}}}\sphinxparamcomma \sphinxparam{\DUrole{n}{\textquotesingle{}assym\textquotesingle{}: \textquotesingle{}gv\textquotesingle{}}}\sphinxparamcomma \sphinxparam{\DUrole{n}{\textquotesingle{}extinction\textquotesingle{}: \textquotesingle{}kv\textquotesingle{}}}\sphinxparamcomma \sphinxparam{\DUrole{n}{\textquotesingle{}geopotential\textquotesingle{}: \textquotesingle{}pphi\textquotesingle{}}}\sphinxparamcomma \sphinxparam{\DUrole{n}{\textquotesingle{}geopotentialSurf\textquotesingle{}: \textquotesingle{}pphis\textquotesingle{}}}\sphinxparamcomma \sphinxparam{\DUrole{n}{\textquotesingle{}latitude\textquotesingle{}: \textquotesingle{}lat\textquotesingle{}}}\sphinxparamcomma \sphinxparam{\DUrole{n}{\textquotesingle{}longitude\textquotesingle{}: \textquotesingle{}lon\textquotesingle{}}}\sphinxparamcomma \sphinxparam{\DUrole{n}{\textquotesingle{}pressure\textquotesingle{}: \textquotesingle{}p\textquotesingle{}}}\sphinxparamcomma \sphinxparam{\DUrole{n}{\textquotesingle{}ssa\textquotesingle{}: \textquotesingle{}wv\textquotesingle{}}}\sphinxparamcomma \sphinxparam{\DUrole{n}{\textquotesingle{}temp\textquotesingle{}: \textquotesingle{}temp\textquotesingle{}}}\sphinxparamcomma \sphinxparam{\DUrole{n}{\textquotesingle{}tsurf\textquotesingle{}: \textquotesingle{}tsurf\textquotesingle{}}}\sphinxparamcomma \sphinxparam{\DUrole{n}{\textquotesingle{}wavelength\textquotesingle{}: \textquotesingle{}wavelength\_vi\textquotesingle{}\}}}\sphinxparamcomma \sphinxparam{\DUrole{n}{wavelength=None}}\sphinxparamcomma \sphinxparam{\DUrole{n}{time=\sphinxhyphen{}1}}\sphinxparamcomma \sphinxparam{\DUrole{n}{hg=False}}\sphinxparamcomma \sphinxparam{\DUrole{n}{nAngle=181}}\sphinxparamcomma \sphinxparam{\DUrole{n}{phaseFunc=\textless{}function Data.\textless{}lambda\textgreater{}\textgreater{}}}}
{}
\pysigstopsignatures
\sphinxAtStartPar
Generate an heterogeneous sphere from LMDZ output and input files.
\begin{quote}\begin{description}
\sphinxlineitem{Parameters}\begin{itemize}
\item {} 
\sphinxAtStartPar
\sphinxstyleliteralstrong{\sphinxupquote{LMDZinput}} (\sphinxstyleliteralemphasis{\sphinxupquote{str}}) \textendash{} Path to the netCDF file containing surface information, to be read.

\item {} 
\sphinxAtStartPar
\sphinxstyleliteralstrong{\sphinxupquote{LMDZouput}} (\sphinxstyleliteralemphasis{\sphinxupquote{str}}) \textendash{} Path to the netCDF file containing atmosphere information, to be
read.

\item {} 
\sphinxAtStartPar
\sphinxstyleliteralstrong{\sphinxupquote{weights}} (\sphinxcode{\sphinxupquote{numpy.ndarray}}, optional) \textendash{} Array containing the weights of the Gaussian quadrature points for
correlated\sphinxhyphen{}k data (shape=(nWeight)). If not provided, assumes no
correlated\sphinxhyphen{}k are used. It is not required if wavelengths are
separted (see wavelengths).

\item {} 
\sphinxAtStartPar
\sphinxstyleliteralstrong{\sphinxupquote{keys}} (\sphinxstyleliteralemphasis{\sphinxupquote{\{str : str\}}}\sphinxstyleliteralemphasis{\sphinxupquote{, }}\sphinxstyleliteralemphasis{\sphinxupquote{default htrdrPy.keysLMDZtitan\_vi}}) \textendash{} 
\sphinxAtStartPar
Dictionnary of correspondance between keys used in this method and
keys from the input/output file. It must contain the following keys:
\begin{itemize}
\item {} \begin{description}
\sphinxlineitem{’tsurf’}{[}default ‘tsurf’{]}
\sphinxAtStartPar
Key for the surface temperature.

\end{description}

\item {} \begin{description}
\sphinxlineitem{’temp’}{[}default ‘temp’{]}
\sphinxAtStartPar
Key for the atmospheric temperature.

\end{description}

\item {} \begin{description}
\sphinxlineitem{’wavelength’}{[}default ‘wavelength\_vi’{]}
\sphinxAtStartPar
Key for the wavelengths.

\end{description}

\item {} \begin{description}
\sphinxlineitem{’ssa’: default ‘wv’}
\sphinxAtStartPar
Key for the single scatering albedo.

\end{description}

\item {} \begin{description}
\sphinxlineitem{’extinction’}{[}default ‘kv’{]}
\sphinxAtStartPar
Key for the extinction coefficeint.

\end{description}

\item {} \begin{description}
\sphinxlineitem{’assym’}{[}default ‘gv’{]}
\sphinxAtStartPar
Key for the assymetry parameter.

\end{description}

\item {} \begin{description}
\sphinxlineitem{’geopotential’}{[}default ‘pphi’{]}
\sphinxAtStartPar
Key for the geopotential.

\end{description}

\item {} \begin{description}
\sphinxlineitem{’altitude’}{[}default ‘altitude’{]}
\sphinxAtStartPar
Key for the altitude.

\end{description}

\item {} \begin{description}
\sphinxlineitem{’latitude’}{[}default ‘lat’{]}
\sphinxAtStartPar
Key for the latitude.

\end{description}

\item {} \begin{description}
\sphinxlineitem{’longitude’}{[}default ‘lon’{]}
\sphinxAtStartPar
Key for the longitude.

\end{description}

\item {} \begin{description}
\sphinxlineitem{’pressure’}{[}default ‘p’{]}
\sphinxAtStartPar
Key for the pressure.

\end{description}

\end{itemize}


\item {} 
\sphinxAtStartPar
\sphinxstyleliteralstrong{\sphinxupquote{wavelengths}} (\sphinxstyleliteralemphasis{\sphinxupquote{\{str : dict\}}}\sphinxstyleliteralemphasis{\sphinxupquote{, }}\sphinxstyleliteralemphasis{\sphinxupquote{optional}}) \textendash{} 
\sphinxAtStartPar
Dictionnary containing the wavelength bands data. It must be
composed of 1 sub\sphinxhyphen{}dictionnary for every band.
\begin{itemize}
\item {} \begin{description}
\sphinxlineitem{’index’}{[}dict{]}
\sphinxAtStartPar
Dictionnary containing the data for a specific band
(‘index’ is the index refering to the band in the GCM output
file, e.g. ‘v\_23’ for the last visible band in Titan GCM).
The sub\sphinxhyphen{}dictionnary must contain the following items:
\begin{itemize}
\item {} \begin{description}
\sphinxlineitem{’wavelength’}{[}float{]}
\sphinxAtStartPar
Central wavelength of the band (in m).

\end{description}

\item {} \begin{description}
\sphinxlineitem{’low’}{[}float{]}
\sphinxAtStartPar
Lower bound of the band (in m).

\end{description}

\item {} \begin{description}
\sphinxlineitem{’up’}{[}float{]}
\sphinxAtStartPar
Upper bound of the band (in m).

\end{description}

\item {} \begin{description}
\sphinxlineitem{’weights’}{[}\sphinxcode{\sphinxupquote{numpy.ndarray}}, optional{]}
\sphinxAtStartPar
The weights associated to the different k\sphinxhyphen{}correlated
coefficents (shape=(nWeights)). If no k\sphinxhyphen{}coeff are used,
omit the key.

\end{description}

\end{itemize}

\end{description}

\end{itemize}


\item {} 
\sphinxAtStartPar
\sphinxstyleliteralstrong{\sphinxupquote{time}} (\sphinxstyleliteralemphasis{\sphinxupquote{int}}\sphinxstyleliteralemphasis{\sphinxupquote{,  }}\sphinxstyleliteralemphasis{\sphinxupquote{default \sphinxhyphen{}1}}) \textendash{} Time index to use, default is last.

\item {} 
\sphinxAtStartPar
\sphinxstyleliteralstrong{\sphinxupquote{hg}} (\sphinxstyleliteralemphasis{\sphinxupquote{bool}}\sphinxstyleliteralemphasis{\sphinxupquote{, }}\sphinxstyleliteralemphasis{\sphinxupquote{default False}}) \textendash{} Whether or not to use the Henyey\sphinxhyphen{}Greenstein phase function built in
htrdr

\item {} 
\sphinxAtStartPar
\sphinxstyleliteralstrong{\sphinxupquote{nAngle}} (\sphinxstyleliteralemphasis{\sphinxupquote{int}}\sphinxstyleliteralemphasis{\sphinxupquote{, }}\sphinxstyleliteralemphasis{\sphinxupquote{optional}}\sphinxstyleliteralemphasis{\sphinxupquote{, }}\sphinxstyleliteralemphasis{\sphinxupquote{default 181}}) \textendash{} Number of angles to use in the discrete phase functions. (If
\sphinxcode{\sphinxupquote{hg=True}} nAngle is omitted).

\item {} 
\sphinxAtStartPar
\sphinxstyleliteralstrong{\sphinxupquote{phaseFunc}} (\sphinxstyleliteralemphasis{\sphinxupquote{func}}\sphinxstyleliteralemphasis{\sphinxupquote{, }}\sphinxstyleliteralemphasis{\sphinxupquote{optional}}\sphinxstyleliteralemphasis{\sphinxupquote{, }}\sphinxstyleliteralemphasis{\sphinxupquote{default = 1 + g cos}}\sphinxstyleliteralemphasis{\sphinxupquote{(}}\sphinxstyleliteralemphasis{\sphinxupquote{theta}}\sphinxstyleliteralemphasis{\sphinxupquote{)}}) \textendash{} Function to calculate the discrete phase function. The function must
take in first argument the asymetry parameter and in second argument
the angle (in rad). If \sphinxcode{\sphinxupquote{hg=True}} phaseFunc is omitted.

\end{itemize}

\end{description}\end{quote}
\subsubsection*{Notes}

\sphinxAtStartPar
If the surface temperature is not provided in the output file, it will
be read from the input file.

\end{fulllineitems}

\index{makeGroundFrom1D() (htrdrPy.data.Data method)@\spxentry{makeGroundFrom1D()}\spxextra{htrdrPy.data.Data method}}

\begin{fulllineitems}
\phantomsection\label{\detokenize{htrdrPy.data:htrdrPy.data.Data.makeGroundFrom1D}}
\pysigstartsignatures
\pysiglinewithargsret
{\sphinxbfcode{\sphinxupquote{makeGroundFrom1D}}}
{\sphinxparam{\DUrole{n}{surfaceTemperature}}\sphinxparamcomma \sphinxparam{\DUrole{n}{brdf}}}
{}
\pysigstopsignatures
\sphinxAtStartPar
Generates a spherical ground considering uniform temperature and
optical properties.
\begin{quote}\begin{description}
\sphinxlineitem{Parameters}\begin{itemize}
\item {} 
\sphinxAtStartPar
\sphinxstyleliteralstrong{\sphinxupquote{surfaceTemperature}} (\sphinxstyleliteralemphasis{\sphinxupquote{float}}) \textendash{} Temperature of the surface {[}K{]}.

\item {} 
\sphinxAtStartPar
\sphinxstyleliteralstrong{\sphinxupquote{brdf}} (\sphinxstyleliteralemphasis{\sphinxupquote{dict}}) \textendash{} 
\sphinxAtStartPar
Surface reflexion properties with the following items:
\begin{itemize}
\item {} \begin{description}
\sphinxlineitem{”kind”}{[}\{“lambertian”, “specular”\}){]}
\sphinxAtStartPar
Kind of brdf function to use.

\end{description}

\item {} \begin{description}
\sphinxlineitem{”albedo”}{[}\sphinxcode{\sphinxupquote{numpy.ndarray}}{]}
\sphinxAtStartPar
Wavelength dependent surface albedos (shape=(nWavelength)) .

\end{description}

\item {} \begin{description}
\sphinxlineitem{”wavelengths”}{[}\sphinxcode{\sphinxupquote{numpy.ndarray}}, optional{]}
\sphinxAtStartPar
Wavelengths where the albedo is defined
(shape=(nWavelength), {[}m{]}). Alternatively, the user can
specify the bands with the “bands” keyword.

\end{description}

\item {} \begin{description}
\sphinxlineitem{”bands”}{[}\sphinxcode{\sphinxupquote{numpy.ndarray}}, optional{]}
\sphinxAtStartPar
Wavelengths bands where the albedo is defined
(shape=(nWavelength,2), {[}m{]}). The values corresponds
to the bands limits.

\end{description}

\end{itemize}


\end{itemize}

\end{description}\end{quote}
\subsubsection*{Notes}

\sphinxAtStartPar
The nTheta and nPhi parameters provided at the initialisation of the
instance are used to define the resolution of the ground mesh and
are therefore manatory.
\subsubsection*{Examples}

\begin{sphinxVerbatim}[commandchars=\\\{\}]
\PYG{g+gp}{\PYGZgt{}\PYGZgt{}\PYGZgt{} }\PYG{n}{d} \PYG{o}{=} \PYG{n}{htrdrPy}\PYG{o}{.}\PYG{n}{Data}\PYG{p}{(}\PYG{n}{radius}\PYG{o}{=}\PYG{l+m+mf}{1e6}\PYG{p}{,} \PYG{n}{nTheta}\PYG{o}{=}\PYG{l+m+mi}{30}\PYG{p}{,} \PYG{n}{nPhi}\PYG{o}{=}\PYG{l+m+mi}{50}\PYG{p}{,} \PYG{n}{name}\PYG{o}{=}\PYG{l+s+s2}{\PYGZdq{}}\PYG{l+s+s2}{Planet}\PYG{l+s+s2}{\PYGZdq{}}\PYG{p}{)}
\PYG{g+gp}{\PYGZgt{}\PYGZgt{}\PYGZgt{} }\PYG{n}{wavelengths} \PYG{o}{=} \PYG{n}{np}\PYG{o}{.}\PYG{n}{linspace}\PYG{p}{(}\PYG{l+m+mf}{2e7}\PYG{p}{,} \PYG{l+m+mf}{9e7}\PYG{p}{,} \PYG{n}{nWavelengths}\PYG{p}{)}
\PYG{g+gp}{\PYGZgt{}\PYGZgt{}\PYGZgt{} }\PYG{n}{bandsLow} \PYG{o}{=} \PYG{n}{np}\PYG{o}{.}\PYG{n}{zeros}\PYG{p}{(}\PYG{n}{nWavelengths}\PYG{p}{)}
\PYG{g+gp}{\PYGZgt{}\PYGZgt{}\PYGZgt{} }\PYG{n}{bandsUp} \PYG{o}{=} \PYG{n}{np}\PYG{o}{.}\PYG{n}{zeros}\PYG{p}{(}\PYG{n}{nWavelengths}\PYG{p}{)}
\PYG{g+gp}{\PYGZgt{}\PYGZgt{}\PYGZgt{} }\PYG{n}{bandsLow}\PYG{p}{[}\PYG{l+m+mi}{1}\PYG{p}{:}\PYG{p}{]}\PYG{p}{,} \PYG{n}{bandsUp}\PYG{p}{[}\PYG{p}{:}\PYG{o}{\PYGZhy{}}\PYG{l+m+mi}{1}\PYG{p}{]} \PYG{o}{=} \PYG{l+m+mf}{0.5} \PYG{o}{*} \PYG{p}{(}\PYG{n}{wavelengths}\PYG{p}{[}\PYG{l+m+mi}{1}\PYG{p}{:}\PYG{p}{]} \PYG{o}{+} \PYG{n}{wavelengths}\PYG{p}{[}\PYG{p}{:}\PYG{o}{\PYGZhy{}}\PYG{l+m+mi}{1}\PYG{p}{]}\PYG{p}{)}
\PYG{g+gp}{\PYGZgt{}\PYGZgt{}\PYGZgt{} }\PYG{n}{bandsLow}\PYG{p}{[}\PYG{l+m+mi}{0}\PYG{p}{]} \PYG{o}{=} \PYG{l+m+mf}{1.5} \PYG{o}{*} \PYG{n}{wavelengths}\PYG{p}{[}\PYG{l+m+mi}{0}\PYG{p}{]} \PYG{o}{\PYGZhy{}} \PYG{l+m+mf}{0.5} \PYG{o}{*} \PYG{n}{wavelengths}\PYG{p}{[}\PYG{l+m+mi}{1}\PYG{p}{]}
\PYG{g+gp}{\PYGZgt{}\PYGZgt{}\PYGZgt{} }\PYG{n}{bandsUp}\PYG{p}{[}\PYG{o}{\PYGZhy{}}\PYG{l+m+mi}{1}\PYG{p}{]} \PYG{o}{=}  \PYG{l+m+mf}{1.5} \PYG{o}{*} \PYG{n}{wavelengths}\PYG{p}{[}\PYG{o}{\PYGZhy{}}\PYG{l+m+mi}{1}\PYG{p}{]} \PYG{o}{\PYGZhy{}} \PYG{l+m+mf}{0.5} \PYG{o}{*} \PYG{n}{wavelengths}\PYG{p}{[}\PYG{o}{\PYGZhy{}}\PYG{l+m+mi}{2}\PYG{p}{]}
\PYG{g+gp}{\PYGZgt{}\PYGZgt{}\PYGZgt{} }\PYG{n}{surfTemp}\PYG{o}{=} \PYG{l+m+mi}{300}
\PYG{g+gp}{\PYGZgt{}\PYGZgt{}\PYGZgt{} }\PYG{n}{surfAlb}\PYG{o}{=} \PYG{n}{np}\PYG{o}{.}\PYG{n}{ones}\PYG{p}{(}\PYG{n}{nWavelengths}\PYG{p}{)} \PYG{o}{*} \PYG{l+m+mf}{0.5}
\PYG{g+gp}{\PYGZgt{}\PYGZgt{}\PYGZgt{} }\PYG{n}{d}\PYG{o}{.}\PYG{n}{makeGroundFrom1D}\PYG{p}{(}\PYG{n}{surfTemp}\PYG{p}{,} \PYG{p}{\PYGZob{}}
\PYG{g+gp}{... }        \PYG{l+s+s2}{\PYGZdq{}}\PYG{l+s+s2}{kind}\PYG{l+s+s2}{\PYGZdq{}}\PYG{p}{:} \PYG{l+s+s2}{\PYGZdq{}}\PYG{l+s+s2}{lambertian}\PYG{l+s+s2}{\PYGZdq{}}\PYG{p}{,}
\PYG{g+gp}{... }        \PYG{l+s+s2}{\PYGZdq{}}\PYG{l+s+s2}{albedo}\PYG{l+s+s2}{\PYGZdq{}}\PYG{p}{:} \PYG{n}{surfAlb}\PYG{p}{,}
\PYG{g+gp}{... }        \PYG{l+s+s2}{\PYGZdq{}}\PYG{l+s+s2}{bands}\PYG{l+s+s2}{\PYGZdq{}}\PYG{p}{:} \PYG{n}{np}\PYG{o}{.}\PYG{n}{array}\PYG{p}{(}\PYG{p}{[}\PYG{n}{bandsLow}\PYG{p}{,} \PYG{n}{bandsUp}\PYG{p}{]}\PYG{p}{)}\PYG{o}{.}\PYG{n}{T}
\PYG{g+gp}{... }        \PYG{p}{\PYGZcb{}}\PYG{p}{)}
\PYG{g+go}{Mesh generator. Ntheta = 30, Nphi = 50, R = 1000000.0}
\PYG{g+go}{Generating points...}
\PYG{g+go}{Generating noeds...}
\PYG{g+go}{triangles,rectangles generation completed. N\PYGZus{}triangles = 98, N\PYGZus{}rectangles = 1323}
\PYG{g+go}{Generating triangles...}
\PYG{g+go}{generating bin...}
\end{sphinxVerbatim}

\end{fulllineitems}

\index{makeGroundFrom1D\_PP() (htrdrPy.data.Data method)@\spxentry{makeGroundFrom1D\_PP()}\spxextra{htrdrPy.data.Data method}}

\begin{fulllineitems}
\phantomsection\label{\detokenize{htrdrPy.data:htrdrPy.data.Data.makeGroundFrom1D_PP}}
\pysigstartsignatures
\pysiglinewithargsret
{\sphinxbfcode{\sphinxupquote{makeGroundFrom1D\_PP}}}
{\sphinxparam{\DUrole{n}{surfaceTemperature}}\sphinxparamcomma \sphinxparam{\DUrole{n}{brdf}}}
{}
\pysigstopsignatures
\sphinxAtStartPar
Generates a plan\sphinxhyphen{}parallel ground considering uniform temperature and
optical properties.
\begin{quote}\begin{description}
\sphinxlineitem{Parameters}\begin{itemize}
\item {} 
\sphinxAtStartPar
\sphinxstyleliteralstrong{\sphinxupquote{surfaceTemperature}} (\sphinxstyleliteralemphasis{\sphinxupquote{float}}) \textendash{} Temperature of the surface {[}K{]}.

\item {} 
\sphinxAtStartPar
\sphinxstyleliteralstrong{\sphinxupquote{brdf}} (\sphinxstyleliteralemphasis{\sphinxupquote{dict}}) \textendash{} 
\sphinxAtStartPar
Surface reflexion properties with the following items:
\begin{itemize}
\item {} \begin{description}
\sphinxlineitem{”kind”}{[}\{“lambertian”, “specular”\}){]}
\sphinxAtStartPar
Kind of brdf function to use.

\end{description}

\item {} \begin{description}
\sphinxlineitem{”albedo”}{[}\sphinxcode{\sphinxupquote{numpy.ndarray}}{]}
\sphinxAtStartPar
Wavelength dependent surface albedos (shape=(nWavelength)) .

\end{description}

\item {} \begin{description}
\sphinxlineitem{”wavelengths”}{[}\sphinxcode{\sphinxupquote{numpy.ndarray}}, optional{]}
\sphinxAtStartPar
Wavelengths where the albedo is defined
(shape=(nWavelength), {[}m{]}). Alternatively, the user can
specify the bands with the “bands” keyword.

\end{description}

\item {} \begin{description}
\sphinxlineitem{”bands”}{[}\sphinxcode{\sphinxupquote{numpy.ndarray}}, optional{]}
\sphinxAtStartPar
Wavelengths bands where the albedo is defined
(shape=(nWavelength,2), {[}m{]}). The values corresponds
to the bands limits.

\end{description}

\end{itemize}


\end{itemize}

\end{description}\end{quote}
\subsubsection*{Notes}

\sphinxAtStartPar
In plan\sphinxhyphen{}parallel mode, the \sphinxcode{\sphinxupquote{radius}} parameter passed at the
initialisation of the instance is used as the horizontal expansion of
the ground. Make sure to use a large enough value. Also, the nTheta and
nPhi parameters are not used.
\subsubsection*{Examples}

\begin{sphinxVerbatim}[commandchars=\\\{\}]
\PYG{g+gp}{\PYGZgt{}\PYGZgt{}\PYGZgt{} }\PYG{n}{d} \PYG{o}{=} \PYG{n}{htrdrPy}\PYG{o}{.}\PYG{n}{Data}\PYG{p}{(}\PYG{n}{radius}\PYG{o}{=}\PYG{l+m+mf}{1e6}\PYG{p}{,} \PYG{n}{name}\PYG{o}{=}\PYG{l+s+s2}{\PYGZdq{}}\PYG{l+s+s2}{Planet}\PYG{l+s+s2}{\PYGZdq{}}\PYG{p}{)}
\PYG{g+gp}{\PYGZgt{}\PYGZgt{}\PYGZgt{} }\PYG{n}{wavelengths} \PYG{o}{=} \PYG{n}{np}\PYG{o}{.}\PYG{n}{linspace}\PYG{p}{(}\PYG{l+m+mf}{2e7}\PYG{p}{,} \PYG{l+m+mf}{9e7}\PYG{p}{,} \PYG{n}{nWavelengths}\PYG{p}{)}
\PYG{g+gp}{\PYGZgt{}\PYGZgt{}\PYGZgt{} }\PYG{n}{bandsLow} \PYG{o}{=} \PYG{n}{np}\PYG{o}{.}\PYG{n}{zeros}\PYG{p}{(}\PYG{n}{nWavelengths}\PYG{p}{)}
\PYG{g+gp}{\PYGZgt{}\PYGZgt{}\PYGZgt{} }\PYG{n}{bandsUp} \PYG{o}{=} \PYG{n}{np}\PYG{o}{.}\PYG{n}{zeros}\PYG{p}{(}\PYG{n}{nWavelengths}\PYG{p}{)}
\PYG{g+gp}{\PYGZgt{}\PYGZgt{}\PYGZgt{} }\PYG{n}{bandsLow}\PYG{p}{[}\PYG{l+m+mi}{1}\PYG{p}{:}\PYG{p}{]}\PYG{p}{,} \PYG{n}{bandsUp}\PYG{p}{[}\PYG{p}{:}\PYG{o}{\PYGZhy{}}\PYG{l+m+mi}{1}\PYG{p}{]} \PYG{o}{=} \PYG{l+m+mf}{0.5} \PYG{o}{*} \PYG{p}{(}\PYG{n}{wavelengths}\PYG{p}{[}\PYG{l+m+mi}{1}\PYG{p}{:}\PYG{p}{]} \PYG{o}{+} \PYG{n}{wavelengths}\PYG{p}{[}\PYG{p}{:}\PYG{o}{\PYGZhy{}}\PYG{l+m+mi}{1}\PYG{p}{]}\PYG{p}{)}
\PYG{g+gp}{\PYGZgt{}\PYGZgt{}\PYGZgt{} }\PYG{n}{bandsLow}\PYG{p}{[}\PYG{l+m+mi}{0}\PYG{p}{]} \PYG{o}{=} \PYG{l+m+mf}{1.5} \PYG{o}{*} \PYG{n}{wavelengths}\PYG{p}{[}\PYG{l+m+mi}{0}\PYG{p}{]} \PYG{o}{\PYGZhy{}} \PYG{l+m+mf}{0.5} \PYG{o}{*} \PYG{n}{wavelengths}\PYG{p}{[}\PYG{l+m+mi}{1}\PYG{p}{]}
\PYG{g+gp}{\PYGZgt{}\PYGZgt{}\PYGZgt{} }\PYG{n}{bandsUp}\PYG{p}{[}\PYG{o}{\PYGZhy{}}\PYG{l+m+mi}{1}\PYG{p}{]} \PYG{o}{=}  \PYG{l+m+mf}{1.5} \PYG{o}{*} \PYG{n}{wavelengths}\PYG{p}{[}\PYG{o}{\PYGZhy{}}\PYG{l+m+mi}{1}\PYG{p}{]} \PYG{o}{\PYGZhy{}} \PYG{l+m+mf}{0.5} \PYG{o}{*} \PYG{n}{wavelengths}\PYG{p}{[}\PYG{o}{\PYGZhy{}}\PYG{l+m+mi}{2}\PYG{p}{]}
\PYG{g+gp}{\PYGZgt{}\PYGZgt{}\PYGZgt{} }\PYG{n}{surfTemp}\PYG{o}{=} \PYG{l+m+mi}{300}
\PYG{g+gp}{\PYGZgt{}\PYGZgt{}\PYGZgt{} }\PYG{n}{surfAlb}\PYG{o}{=} \PYG{n}{np}\PYG{o}{.}\PYG{n}{ones}\PYG{p}{(}\PYG{n}{nWavelengths}\PYG{p}{)} \PYG{o}{*} \PYG{l+m+mf}{0.5}
\PYG{g+gp}{\PYGZgt{}\PYGZgt{}\PYGZgt{} }\PYG{n}{d}\PYG{o}{.}\PYG{n}{makeGroundFrom1D\PYGZus{}PP}\PYG{p}{(}\PYG{n}{surfTemp}\PYG{p}{,} \PYG{p}{\PYGZob{}}
\PYG{g+gp}{... }        \PYG{l+s+s2}{\PYGZdq{}}\PYG{l+s+s2}{kind}\PYG{l+s+s2}{\PYGZdq{}}\PYG{p}{:} \PYG{l+s+s2}{\PYGZdq{}}\PYG{l+s+s2}{lambertian}\PYG{l+s+s2}{\PYGZdq{}}\PYG{p}{,}
\PYG{g+gp}{... }        \PYG{l+s+s2}{\PYGZdq{}}\PYG{l+s+s2}{albedo}\PYG{l+s+s2}{\PYGZdq{}}\PYG{p}{:} \PYG{n}{surfAlb}\PYG{p}{,}
\PYG{g+gp}{... }        \PYG{l+s+s2}{\PYGZdq{}}\PYG{l+s+s2}{bands}\PYG{l+s+s2}{\PYGZdq{}}\PYG{p}{:} \PYG{n}{np}\PYG{o}{.}\PYG{n}{array}\PYG{p}{(}\PYG{p}{[}\PYG{n}{bandsLow}\PYG{p}{,} \PYG{n}{bandsUp}\PYG{p}{]}\PYG{p}{)}\PYG{o}{.}\PYG{n}{T}
\PYG{g+gp}{... }        \PYG{p}{\PYGZcb{}}\PYG{p}{)}
\PYG{g+go}{generating bin...}
\end{sphinxVerbatim}

\end{fulllineitems}

\index{makeGroundFrom2D() (htrdrPy.data.Data method)@\spxentry{makeGroundFrom2D()}\spxextra{htrdrPy.data.Data method}}

\begin{fulllineitems}
\phantomsection\label{\detokenize{htrdrPy.data:htrdrPy.data.Data.makeGroundFrom2D}}
\pysigstartsignatures
\pysiglinewithargsret
{\sphinxbfcode{\sphinxupquote{makeGroundFrom2D}}}
{\sphinxparam{\DUrole{n}{surfaceTemperature}}\sphinxparamcomma \sphinxparam{\DUrole{n}{brdf}}}
{}
\pysigstopsignatures
\sphinxAtStartPar
Generates a spherical ground considering temperature and
optical properties varying along the latitude.
\begin{quote}\begin{description}
\sphinxlineitem{Parameters}\begin{itemize}
\item {} 
\sphinxAtStartPar
\sphinxstyleliteralstrong{\sphinxupquote{surfaceTemperature}} (\sphinxcode{\sphinxupquote{numpy.ndarray}}) \textendash{} Temperature of the surface (shape=(nLat), {[}K{]}).

\item {} 
\sphinxAtStartPar
\sphinxstyleliteralstrong{\sphinxupquote{brdf}} (\sphinxstyleliteralemphasis{\sphinxupquote{dict}}) \textendash{} 
\sphinxAtStartPar
Surface reflexion properties with the following items:
\begin{itemize}
\item {} \begin{description}
\sphinxlineitem{”kind”}{[}\{“lambertian”, “specular”\}){]}
\sphinxAtStartPar
Kind of brdf function to use.

\end{description}

\item {} \begin{description}
\sphinxlineitem{”albedo”}{[}\sphinxcode{\sphinxupquote{numpy.ndarray}}{]}
\sphinxAtStartPar
Wavelength dependent surface albedos
(shape=(nWavelength, nLat)).

\end{description}

\item {} \begin{description}
\sphinxlineitem{”latitude”}{[}\sphinxcode{\sphinxupquote{numpy.ndarray}}{]}
\sphinxAtStartPar
List of latitudes (shape=(nLat), {[}°{]}).

\end{description}

\item {} \begin{description}
\sphinxlineitem{”wavelengths”}{[}\sphinxcode{\sphinxupquote{numpy.ndarray}}, optional{]}
\sphinxAtStartPar
Wavelengths where the albedo is defined
(shape=(nWavelength), {[}m{]}). Alternatively, the user can
specify the bands with the “bands” keyword.

\end{description}

\item {} \begin{description}
\sphinxlineitem{”bands”}{[}\sphinxcode{\sphinxupquote{numpy.ndarray}}, optional{]}
\sphinxAtStartPar
Wavelengths bands where the albedo is defined
(shape=(nWavelength,2), {[}m{]}). The values corresponds
to the bands limits.

\end{description}

\end{itemize}


\end{itemize}

\end{description}\end{quote}
\subsubsection*{Notes}

\sphinxAtStartPar
The nPhi parameter provided at the initialisation of the
instance is used to define the longitudinal resolution of the ground
mesh and is therefore manatory. The nTheta parameter is obtained from
the length of the \sphinxcode{\sphinxupquote{latitude}} provided.
\subsubsection*{Examples}

\begin{sphinxVerbatim}[commandchars=\\\{\}]
\PYG{g+gp}{\PYGZgt{}\PYGZgt{}\PYGZgt{} }\PYG{n}{d} \PYG{o}{=} \PYG{n}{htrdrPy}\PYG{o}{.}\PYG{n}{Data}\PYG{p}{(}\PYG{n}{radius}\PYG{o}{=}\PYG{l+m+mf}{1e6}\PYG{p}{,} \PYG{n}{nPhi}\PYG{o}{=}\PYG{l+m+mi}{50}\PYG{p}{,} \PYG{n}{name}\PYG{o}{=}\PYG{l+s+s2}{\PYGZdq{}}\PYG{l+s+s2}{Planet}\PYG{l+s+s2}{\PYGZdq{}}\PYG{p}{)}
\PYG{g+gp}{\PYGZgt{}\PYGZgt{}\PYGZgt{} }\PYG{n}{nLat} \PYG{o}{=} \PYG{l+m+mi}{30}
\PYG{g+gp}{\PYGZgt{}\PYGZgt{}\PYGZgt{} }\PYG{n}{wavelengths} \PYG{o}{=} \PYG{n}{np}\PYG{o}{.}\PYG{n}{linspace}\PYG{p}{(}\PYG{l+m+mf}{2e7}\PYG{p}{,} \PYG{l+m+mf}{9e7}\PYG{p}{,} \PYG{n}{nWavelengths}\PYG{p}{)}
\PYG{g+gp}{\PYGZgt{}\PYGZgt{}\PYGZgt{} }\PYG{n}{bandsLow} \PYG{o}{=} \PYG{n}{np}\PYG{o}{.}\PYG{n}{zeros}\PYG{p}{(}\PYG{n}{nWavelengths}\PYG{p}{)}
\PYG{g+gp}{\PYGZgt{}\PYGZgt{}\PYGZgt{} }\PYG{n}{bandsUp} \PYG{o}{=} \PYG{n}{np}\PYG{o}{.}\PYG{n}{zeros}\PYG{p}{(}\PYG{n}{nWavelengths}\PYG{p}{)}
\PYG{g+gp}{\PYGZgt{}\PYGZgt{}\PYGZgt{} }\PYG{n}{bandsLow}\PYG{p}{[}\PYG{l+m+mi}{1}\PYG{p}{:}\PYG{p}{]}\PYG{p}{,} \PYG{n}{bandsUp}\PYG{p}{[}\PYG{p}{:}\PYG{o}{\PYGZhy{}}\PYG{l+m+mi}{1}\PYG{p}{]} \PYG{o}{=} \PYG{l+m+mf}{0.5} \PYG{o}{*} \PYG{p}{(}\PYG{n}{wavelengths}\PYG{p}{[}\PYG{l+m+mi}{1}\PYG{p}{:}\PYG{p}{]} \PYG{o}{+} \PYG{n}{wavelengths}\PYG{p}{[}\PYG{p}{:}\PYG{o}{\PYGZhy{}}\PYG{l+m+mi}{1}\PYG{p}{]}\PYG{p}{)}
\PYG{g+gp}{\PYGZgt{}\PYGZgt{}\PYGZgt{} }\PYG{n}{bandsLow}\PYG{p}{[}\PYG{l+m+mi}{0}\PYG{p}{]} \PYG{o}{=} \PYG{l+m+mf}{1.5} \PYG{o}{*} \PYG{n}{wavelengths}\PYG{p}{[}\PYG{l+m+mi}{0}\PYG{p}{]} \PYG{o}{\PYGZhy{}} \PYG{l+m+mf}{0.5} \PYG{o}{*} \PYG{n}{wavelengths}\PYG{p}{[}\PYG{l+m+mi}{1}\PYG{p}{]}
\PYG{g+gp}{\PYGZgt{}\PYGZgt{}\PYGZgt{} }\PYG{n}{bandsUp}\PYG{p}{[}\PYG{o}{\PYGZhy{}}\PYG{l+m+mi}{1}\PYG{p}{]} \PYG{o}{=}  \PYG{l+m+mf}{1.5} \PYG{o}{*} \PYG{n}{wavelengths}\PYG{p}{[}\PYG{o}{\PYGZhy{}}\PYG{l+m+mi}{1}\PYG{p}{]} \PYG{o}{\PYGZhy{}} \PYG{l+m+mf}{0.5} \PYG{o}{*} \PYG{n}{wavelengths}\PYG{p}{[}\PYG{o}{\PYGZhy{}}\PYG{l+m+mi}{2}\PYG{p}{]}
\PYG{g+gp}{\PYGZgt{}\PYGZgt{}\PYGZgt{} }\PYG{n}{latitudes} \PYG{o}{=} \PYG{n}{np}\PYG{o}{.}\PYG{n}{linspace}\PYG{p}{(}\PYG{o}{\PYGZhy{}}\PYG{l+m+mi}{90}\PYG{p}{,} \PYG{l+m+mi}{90}\PYG{p}{,} \PYG{n}{nLat}\PYG{p}{)}
\PYG{g+gp}{\PYGZgt{}\PYGZgt{}\PYGZgt{} }\PYG{n}{surfTemp} \PYG{o}{=} \PYG{l+m+mi}{300} \PYG{o}{*} \PYG{n}{np}\PYG{o}{.}\PYG{n}{ones\PYGZus{}like}\PYG{p}{(}\PYG{n}{latitudes}\PYG{p}{)}
\PYG{g+gp}{\PYGZgt{}\PYGZgt{}\PYGZgt{} }\PYG{n}{surfAlb}\PYG{o}{=} \PYG{n}{np}\PYG{o}{.}\PYG{n}{ones}\PYG{p}{(}\PYG{p}{(}\PYG{n}{nWavelengths}\PYG{p}{,} \PYG{n}{nLat}\PYG{p}{)}\PYG{p}{)} \PYG{o}{*} \PYG{l+m+mf}{0.5}
\PYG{g+gp}{\PYGZgt{}\PYGZgt{}\PYGZgt{} }\PYG{n}{d}\PYG{o}{.}\PYG{n}{makeGroundFrom2D}\PYG{p}{(}\PYG{n}{surfTemp}\PYG{p}{,} \PYG{p}{\PYGZob{}}
\PYG{g+gp}{... }        \PYG{l+s+s2}{\PYGZdq{}}\PYG{l+s+s2}{kind}\PYG{l+s+s2}{\PYGZdq{}}\PYG{p}{:} \PYG{l+s+s2}{\PYGZdq{}}\PYG{l+s+s2}{lambertian}\PYG{l+s+s2}{\PYGZdq{}}\PYG{p}{,}
\PYG{g+gp}{... }        \PYG{l+s+s2}{\PYGZdq{}}\PYG{l+s+s2}{albedo}\PYG{l+s+s2}{\PYGZdq{}}\PYG{p}{:} \PYG{n}{surfAlb}\PYG{p}{,}
\PYG{g+gp}{... }        \PYG{l+s+s2}{\PYGZdq{}}\PYG{l+s+s2}{latitude}\PYG{l+s+s2}{\PYGZdq{}}\PYG{p}{:} \PYG{n}{latitudes}\PYG{p}{,}
\PYG{g+gp}{... }        \PYG{l+s+s2}{\PYGZdq{}}\PYG{l+s+s2}{bands}\PYG{l+s+s2}{\PYGZdq{}}\PYG{p}{:} \PYG{n}{np}\PYG{o}{.}\PYG{n}{array}\PYG{p}{(}\PYG{p}{[}\PYG{n}{bandsLow}\PYG{p}{,} \PYG{n}{bandsUp}\PYG{p}{]}\PYG{p}{)}\PYG{o}{.}\PYG{n}{T}
\PYG{g+gp}{... }        \PYG{p}{\PYGZcb{}}\PYG{p}{)}
\PYG{g+go}{Mesh generator. Ntheta = 30, Nphi = 50, R = 1000000.0}
\PYG{g+go}{Generating points...}
\PYG{g+go}{Generating noeds...}
\PYG{g+go}{triangles,rectangles generation completed. N\PYGZus{}triangles = 98, N\PYGZus{}rectangles = 1323}
\PYG{g+go}{Generating triangles...}
\PYG{g+go}{generating bin...}
\end{sphinxVerbatim}

\end{fulllineitems}

\index{makeGroundFrom3D() (htrdrPy.data.Data method)@\spxentry{makeGroundFrom3D()}\spxextra{htrdrPy.data.Data method}}

\begin{fulllineitems}
\phantomsection\label{\detokenize{htrdrPy.data:htrdrPy.data.Data.makeGroundFrom3D}}
\pysigstartsignatures
\pysiglinewithargsret
{\sphinxbfcode{\sphinxupquote{makeGroundFrom3D}}}
{\sphinxparam{\DUrole{n}{SurfaceTemperature}}\sphinxparamcomma \sphinxparam{\DUrole{n}{brdf}}}
{}
\pysigstopsignatures
\sphinxAtStartPar
Generates a spherical ground considering temperature and
optical properties varying along the latitude and longitude.
\begin{quote}\begin{description}
\sphinxlineitem{Parameters}\begin{itemize}
\item {} 
\sphinxAtStartPar
\sphinxstyleliteralstrong{\sphinxupquote{surfaceTemperature}} (\sphinxcode{\sphinxupquote{numpy.ndarray}}) \textendash{} Temperature of the surface (shape=(nLat, nLon), {[}K{]}).

\item {} 
\sphinxAtStartPar
\sphinxstyleliteralstrong{\sphinxupquote{brdf}} (\sphinxstyleliteralemphasis{\sphinxupquote{dict}}) \textendash{} 
\sphinxAtStartPar
Surface reflexion properties with the following items:
\begin{itemize}
\item {} \begin{description}
\sphinxlineitem{”kind”}{[}\{“lambertian”, “specular”\}){]}
\sphinxAtStartPar
Kind of brdf function to use.

\end{description}

\item {} \begin{description}
\sphinxlineitem{”albedo”}{[}\sphinxcode{\sphinxupquote{numpy.ndarray}}{]}
\sphinxAtStartPar
Wavelength dependent surface albedos
(shape=(nWavelength, nLat, nLon)).

\end{description}

\item {} \begin{description}
\sphinxlineitem{”latitude”}{[}\sphinxcode{\sphinxupquote{numpy.ndarray}}{]}
\sphinxAtStartPar
List of latitudes (shape=(nLat), {[}°{]}).

\end{description}

\item {} \begin{description}
\sphinxlineitem{”longitude”}{[}\sphinxcode{\sphinxupquote{numpy.ndarray}}{]}
\sphinxAtStartPar
List of longitudes (shape=(nLon), {[}°{]}).

\end{description}

\item {} \begin{description}
\sphinxlineitem{”wavelengths”}{[}\sphinxcode{\sphinxupquote{numpy.ndarray}}, optional{]}
\sphinxAtStartPar
Wavelengths where the albedo is defined
(shape=(nWavelength), {[}m{]}). Alternatively, the user can
specify the bands with the “bands” keyword.

\end{description}

\item {} \begin{description}
\sphinxlineitem{”bands”}{[}\sphinxcode{\sphinxupquote{numpy.ndarray}}, optional{]}
\sphinxAtStartPar
Wavelengths bands where the albedo is defined
(shape=(nWavelength,2), {[}m{]}). The values corresponds
to the bands limits.

\end{description}

\end{itemize}


\end{itemize}

\end{description}\end{quote}
\subsubsection*{Notes}

\sphinxAtStartPar
The nTheta and nPhi parameters provided at the initialisation of the
instance are not used and instead are defined from the length of the
\sphinxcode{\sphinxupquote{latitude}} and \sphinxcode{\sphinxupquote{longitude}}, respectively.
\subsubsection*{Examples}

\begin{sphinxVerbatim}[commandchars=\\\{\}]
\PYG{g+gp}{\PYGZgt{}\PYGZgt{}\PYGZgt{} }\PYG{n}{d} \PYG{o}{=} \PYG{n}{htrdrPy}\PYG{o}{.}\PYG{n}{Data}\PYG{p}{(}\PYG{n}{radius}\PYG{o}{=}\PYG{l+m+mf}{1e6}\PYG{p}{,} \PYG{n}{name}\PYG{o}{=}\PYG{l+s+s2}{\PYGZdq{}}\PYG{l+s+s2}{Planet}\PYG{l+s+s2}{\PYGZdq{}}\PYG{p}{)}
\PYG{g+gp}{\PYGZgt{}\PYGZgt{}\PYGZgt{} }\PYG{n}{nLat} \PYG{o}{=} \PYG{l+m+mi}{30}
\PYG{g+gp}{\PYGZgt{}\PYGZgt{}\PYGZgt{} }\PYG{n}{nLon} \PYG{o}{=} \PYG{l+m+mi}{50}
\PYG{g+gp}{\PYGZgt{}\PYGZgt{}\PYGZgt{} }\PYG{n}{wavelengths} \PYG{o}{=} \PYG{n}{np}\PYG{o}{.}\PYG{n}{linspace}\PYG{p}{(}\PYG{l+m+mf}{2e7}\PYG{p}{,} \PYG{l+m+mf}{9e7}\PYG{p}{,} \PYG{n}{nWavelengths}\PYG{p}{)}
\PYG{g+gp}{\PYGZgt{}\PYGZgt{}\PYGZgt{} }\PYG{n}{bandsLow} \PYG{o}{=} \PYG{n}{np}\PYG{o}{.}\PYG{n}{zeros}\PYG{p}{(}\PYG{n}{nWavelengths}\PYG{p}{)}
\PYG{g+gp}{\PYGZgt{}\PYGZgt{}\PYGZgt{} }\PYG{n}{bandsUp} \PYG{o}{=} \PYG{n}{np}\PYG{o}{.}\PYG{n}{zeros}\PYG{p}{(}\PYG{n}{nWavelengths}\PYG{p}{)}
\PYG{g+gp}{\PYGZgt{}\PYGZgt{}\PYGZgt{} }\PYG{n}{bandsLow}\PYG{p}{[}\PYG{l+m+mi}{1}\PYG{p}{:}\PYG{p}{]}\PYG{p}{,} \PYG{n}{bandsUp}\PYG{p}{[}\PYG{p}{:}\PYG{o}{\PYGZhy{}}\PYG{l+m+mi}{1}\PYG{p}{]} \PYG{o}{=} \PYG{l+m+mf}{0.5} \PYG{o}{*} \PYG{p}{(}\PYG{n}{wavelengths}\PYG{p}{[}\PYG{l+m+mi}{1}\PYG{p}{:}\PYG{p}{]} \PYG{o}{+} \PYG{n}{wavelengths}\PYG{p}{[}\PYG{p}{:}\PYG{o}{\PYGZhy{}}\PYG{l+m+mi}{1}\PYG{p}{]}\PYG{p}{)}
\PYG{g+gp}{\PYGZgt{}\PYGZgt{}\PYGZgt{} }\PYG{n}{bandsLow}\PYG{p}{[}\PYG{l+m+mi}{0}\PYG{p}{]} \PYG{o}{=} \PYG{l+m+mf}{1.5} \PYG{o}{*} \PYG{n}{wavelengths}\PYG{p}{[}\PYG{l+m+mi}{0}\PYG{p}{]} \PYG{o}{\PYGZhy{}} \PYG{l+m+mf}{0.5} \PYG{o}{*} \PYG{n}{wavelengths}\PYG{p}{[}\PYG{l+m+mi}{1}\PYG{p}{]}
\PYG{g+gp}{\PYGZgt{}\PYGZgt{}\PYGZgt{} }\PYG{n}{bandsUp}\PYG{p}{[}\PYG{o}{\PYGZhy{}}\PYG{l+m+mi}{1}\PYG{p}{]} \PYG{o}{=}  \PYG{l+m+mf}{1.5} \PYG{o}{*} \PYG{n}{wavelengths}\PYG{p}{[}\PYG{o}{\PYGZhy{}}\PYG{l+m+mi}{1}\PYG{p}{]} \PYG{o}{\PYGZhy{}} \PYG{l+m+mf}{0.5} \PYG{o}{*} \PYG{n}{wavelengths}\PYG{p}{[}\PYG{o}{\PYGZhy{}}\PYG{l+m+mi}{2}\PYG{p}{]}
\PYG{g+gp}{\PYGZgt{}\PYGZgt{}\PYGZgt{} }\PYG{n}{latitudes} \PYG{o}{=} \PYG{n}{np}\PYG{o}{.}\PYG{n}{linspace}\PYG{p}{(}\PYG{o}{\PYGZhy{}}\PYG{l+m+mi}{90}\PYG{p}{,} \PYG{l+m+mi}{90}\PYG{p}{,} \PYG{n}{nLat}\PYG{p}{)}
\PYG{g+gp}{\PYGZgt{}\PYGZgt{}\PYGZgt{} }\PYG{n}{longitudes} \PYG{o}{=} \PYG{n}{np}\PYG{o}{.}\PYG{n}{linspace}\PYG{p}{(}\PYG{o}{\PYGZhy{}}\PYG{l+m+mi}{180}\PYG{p}{,} \PYG{l+m+mi}{180}\PYG{p}{,} \PYG{n}{nLon}\PYG{p}{)}
\PYG{g+gp}{\PYGZgt{}\PYGZgt{}\PYGZgt{} }\PYG{n}{surfTemp} \PYG{o}{=} \PYG{l+m+mi}{300} \PYG{o}{*} \PYG{n}{np}\PYG{o}{.}\PYG{n}{ones}\PYG{p}{(}\PYG{p}{(}\PYG{n}{nLat}\PYG{p}{,} \PYG{n}{nLon}\PYG{p}{)}\PYG{p}{)}
\PYG{g+gp}{\PYGZgt{}\PYGZgt{}\PYGZgt{} }\PYG{n}{surfAlb}\PYG{o}{=} \PYG{n}{np}\PYG{o}{.}\PYG{n}{ones}\PYG{p}{(}\PYG{p}{(}\PYG{n}{nWavelengths}\PYG{p}{,} \PYG{n}{nLat}\PYG{p}{,} \PYG{n}{nLon}\PYG{p}{)}\PYG{p}{)} \PYG{o}{*} \PYG{l+m+mf}{0.5}
\PYG{g+gp}{\PYGZgt{}\PYGZgt{}\PYGZgt{} }\PYG{n}{d}\PYG{o}{.}\PYG{n}{makeGroundFrom3D}\PYG{p}{(}\PYG{n}{surfTemp}\PYG{p}{,} \PYG{p}{\PYGZob{}}
\PYG{g+gp}{... }        \PYG{l+s+s2}{\PYGZdq{}}\PYG{l+s+s2}{kind}\PYG{l+s+s2}{\PYGZdq{}}\PYG{p}{:} \PYG{l+s+s2}{\PYGZdq{}}\PYG{l+s+s2}{lambertian}\PYG{l+s+s2}{\PYGZdq{}}\PYG{p}{,}
\PYG{g+gp}{... }        \PYG{l+s+s2}{\PYGZdq{}}\PYG{l+s+s2}{albedo}\PYG{l+s+s2}{\PYGZdq{}}\PYG{p}{:} \PYG{n}{surfAlb}\PYG{p}{,}
\PYG{g+gp}{... }        \PYG{l+s+s2}{\PYGZdq{}}\PYG{l+s+s2}{latitude}\PYG{l+s+s2}{\PYGZdq{}}\PYG{p}{:} \PYG{n}{latitudes}\PYG{p}{,}
\PYG{g+gp}{... }        \PYG{l+s+s2}{\PYGZdq{}}\PYG{l+s+s2}{longitude}\PYG{l+s+s2}{\PYGZdq{}}\PYG{p}{:} \PYG{n}{longitudes}\PYG{p}{,}
\PYG{g+gp}{... }        \PYG{l+s+s2}{\PYGZdq{}}\PYG{l+s+s2}{bands}\PYG{l+s+s2}{\PYGZdq{}}\PYG{p}{:} \PYG{n}{np}\PYG{o}{.}\PYG{n}{array}\PYG{p}{(}\PYG{p}{[}\PYG{n}{bandsLow}\PYG{p}{,} \PYG{n}{bandsUp}\PYG{p}{]}\PYG{p}{)}\PYG{o}{.}\PYG{n}{T}
\PYG{g+gp}{... }        \PYG{p}{\PYGZcb{}}\PYG{p}{)}
\PYG{g+go}{Mesh generator. Ntheta = 30, Nphi = 50, R = 1000000.0}
\PYG{g+go}{Generating points...}
\PYG{g+go}{Generating noeds...}
\PYG{g+go}{triangles,rectangles generation completed. N\PYGZus{}triangles = 98, N\PYGZus{}rectangles = 1323}
\PYG{g+go}{Generating triangles...}
\PYG{g+go}{generating bin...}
\end{sphinxVerbatim}

\end{fulllineitems}

\index{writeInputAtmosphere() (htrdrPy.data.Data method)@\spxentry{writeInputAtmosphere()}\spxextra{htrdrPy.data.Data method}}

\begin{fulllineitems}
\phantomsection\label{\detokenize{htrdrPy.data:htrdrPy.data.Data.writeInputAtmosphere}}
\pysigstartsignatures
\pysiglinewithargsret
{\sphinxbfcode{\sphinxupquote{writeInputAtmosphere}}}
{}
{}
\pysigstopsignatures
\sphinxAtStartPar
Write the atmosphere binary input files for htrdr\sphinxhyphen{}planets

\end{fulllineitems}

\index{writeInputGround() (htrdrPy.data.Data method)@\spxentry{writeInputGround()}\spxextra{htrdrPy.data.Data method}}

\begin{fulllineitems}
\phantomsection\label{\detokenize{htrdrPy.data:htrdrPy.data.Data.writeInputGround}}
\pysigstartsignatures
\pysiglinewithargsret
{\sphinxbfcode{\sphinxupquote{writeInputGround}}}
{}
{}
\pysigstopsignatures
\sphinxAtStartPar
Write the ground binary input files for htrdr\sphinxhyphen{}planets

\end{fulllineitems}

\index{writeInputs() (htrdrPy.data.Data method)@\spxentry{writeInputs()}\spxextra{htrdrPy.data.Data method}}

\begin{fulllineitems}
\phantomsection\label{\detokenize{htrdrPy.data:htrdrPy.data.Data.writeInputs}}
\pysigstartsignatures
\pysiglinewithargsret
{\sphinxbfcode{\sphinxupquote{writeInputs}}}
{\sphinxparam{\DUrole{n}{octree\_def}\DUrole{o}{=}\DUrole{default_value}{512}}\sphinxparamcomma \sphinxparam{\DUrole{n}{opthick}\DUrole{o}{=}\DUrole{default_value}{1}}\sphinxparamcomma \sphinxparam{\DUrole{n}{nthOctree}\DUrole{o}{=}\DUrole{default_value}{8}}\sphinxparamcomma \sphinxparam{\DUrole{n}{procOctree}\DUrole{o}{=}\DUrole{default_value}{\textquotesingle{}master\textquotesingle{}}}\sphinxparamcomma \sphinxparam{\DUrole{n}{octreeFile}\DUrole{o}{=}\DUrole{default_value}{\textquotesingle{}\textquotesingle{}}}}
{}
\pysigstopsignatures
\sphinxAtStartPar
Write the binary input files for htrdr\sphinxhyphen{}planets and precalculate octrees
if the \sphinxcode{\sphinxupquote{octreeFile}} is passed.
\begin{quote}\begin{description}
\sphinxlineitem{Parameters}\begin{itemize}
\item {} 
\sphinxAtStartPar
\sphinxstyleliteralstrong{\sphinxupquote{octree\_def}} (\sphinxstyleliteralemphasis{\sphinxupquote{int}}\sphinxstyleliteralemphasis{\sphinxupquote{ or }}\sphinxstyleliteralemphasis{\sphinxupquote{str}}\sphinxstyleliteralemphasis{\sphinxupquote{, }}\sphinxstyleliteralemphasis{\sphinxupquote{default 512}}) \textendash{} Maximal defintion of the octree grid.

\item {} 
\sphinxAtStartPar
\sphinxstyleliteralstrong{\sphinxupquote{opthick}} (\sphinxstyleliteralemphasis{\sphinxupquote{float}}\sphinxstyleliteralemphasis{\sphinxupquote{ or }}\sphinxstyleliteralemphasis{\sphinxupquote{str}}\sphinxstyleliteralemphasis{\sphinxupquote{, }}\sphinxstyleliteralemphasis{\sphinxupquote{default 1}}) \textendash{} Optical thickness threshold to assess the merge of cells.

\item {} 
\sphinxAtStartPar
\sphinxstyleliteralstrong{\sphinxupquote{nthOctree}} (\sphinxstyleliteralemphasis{\sphinxupquote{int}}\sphinxstyleliteralemphasis{\sphinxupquote{, }}\sphinxstyleliteralemphasis{\sphinxupquote{default 8}}) \textendash{} Number of threads to use for octree computation.

\item {} 
\sphinxAtStartPar
\sphinxstyleliteralstrong{\sphinxupquote{procOctree}} (\sphinxstyleliteralemphasis{\sphinxupquote{\{\textquotesingle{}all\textquotesingle{}}}\sphinxstyleliteralemphasis{\sphinxupquote{, }}\sphinxstyleliteralemphasis{\sphinxupquote{\textquotesingle{}master\textquotesingle{}\}}}\sphinxstyleliteralemphasis{\sphinxupquote{, }}\sphinxstyleliteralemphasis{\sphinxupquote{default \textquotesingle{}master\textquotesingle{}}}) \textendash{} Which process must realize the octree calculation (useless if no
storage). Put to ‘all’ if the processes do not share the disk space.

\item {} 
\sphinxAtStartPar
\sphinxstyleliteralstrong{\sphinxupquote{octreeFile}} (\sphinxstyleliteralemphasis{\sphinxupquote{str}}\sphinxstyleliteralemphasis{\sphinxupquote{, }}\sphinxstyleliteralemphasis{\sphinxupquote{optional}}) \textendash{} Filename to use for the stroring of octrees. If not provided,
octrees are not stored on disk. If a file is provided, a “blank run”
of htrdr\sphinxhyphen{}planets is launched to precalculate octrees. The file is
located in the outputs\_\{name\} folder.

\end{itemize}

\end{description}\end{quote}
\subsubsection*{Examples}

\begin{sphinxVerbatim}[commandchars=\\\{\}]
\PYG{g+gp}{\PYGZgt{}\PYGZgt{}\PYGZgt{} }\PYG{n}{d} \PYG{o}{=} \PYG{n}{htrdrPy}\PYG{o}{.}\PYG{n}{Data}\PYG{p}{(}\PYG{n}{radius}\PYG{o}{=}\PYG{l+m+mf}{1e6}\PYG{p}{,} \PYG{n}{nPhi}\PYG{o}{=}\PYG{l+m+mi}{50}\PYG{p}{,} \PYG{n}{name}\PYG{o}{=}\PYG{l+s+s2}{\PYGZdq{}}\PYG{l+s+s2}{Planet}\PYG{l+s+s2}{\PYGZdq{}}\PYG{p}{)}
\PYG{g+gp}{...}
\PYG{g+gp}{\PYGZgt{}\PYGZgt{}\PYGZgt{} }\PYG{n}{d}\PYG{o}{.}\PYG{n}{writeInputs}\PYG{p}{(}\PYG{n}{octreeFile}\PYG{o}{=}\PYG{l+s+s2}{\PYGZdq{}}\PYG{l+s+s2}{octree.bin}\PYG{l+s+s2}{\PYGZdq{}}\PYG{p}{)}
\end{sphinxVerbatim}

\sphinxAtStartPar
This will generate the input files and start a very quick run of
htrdr\sphinxhyphen{}planets to generate an octree file that will be stored in
outputs\_Planet/octree.bin

\begin{sphinxVerbatim}[commandchars=\\\{\}]
\PYG{g+gp}{\PYGZgt{}\PYGZgt{}\PYGZgt{} }\PYG{n}{d}\PYG{o}{.}\PYG{n}{writeInputs}\PYG{p}{(}\PYG{n}{octree\PYGZus{}def}\PYG{o}{=}\PYG{l+m+mi}{1024}\PYG{p}{)}
\end{sphinxVerbatim}

\sphinxAtStartPar
This will generate the input files and keep the information on the
octree\_def for the future calculation of the octrees. The octrees are
not precalculated since they will only be stored in memory.

\end{fulllineitems}

\index{writeVTKfiles() (htrdrPy.data.Data method)@\spxentry{writeVTKfiles()}\spxextra{htrdrPy.data.Data method}}

\begin{fulllineitems}
\phantomsection\label{\detokenize{htrdrPy.data:htrdrPy.data.Data.writeVTKfiles}}
\pysigstartsignatures
\pysiglinewithargsret
{\sphinxbfcode{\sphinxupquote{writeVTKfiles}}}
{}
{}
\pysigstopsignatures
\sphinxAtStartPar
Write the VTK and obj files for view in paraview or other visualisation
software handling obj and VTK files.

\end{fulllineitems}

\index{writeVTKfilesAtmosphere() (htrdrPy.data.Data method)@\spxentry{writeVTKfilesAtmosphere()}\spxextra{htrdrPy.data.Data method}}

\begin{fulllineitems}
\phantomsection\label{\detokenize{htrdrPy.data:htrdrPy.data.Data.writeVTKfilesAtmosphere}}
\pysigstartsignatures
\pysiglinewithargsret
{\sphinxbfcode{\sphinxupquote{writeVTKfilesAtmosphere}}}
{}
{}
\pysigstopsignatures
\sphinxAtStartPar
Write the atmosphere VTK and obj files.

\end{fulllineitems}

\index{writeVTKfilesGround() (htrdrPy.data.Data method)@\spxentry{writeVTKfilesGround()}\spxextra{htrdrPy.data.Data method}}

\begin{fulllineitems}
\phantomsection\label{\detokenize{htrdrPy.data:htrdrPy.data.Data.writeVTKfilesGround}}
\pysigstartsignatures
\pysiglinewithargsret
{\sphinxbfcode{\sphinxupquote{writeVTKfilesGround}}}
{}
{}
\pysigstopsignatures
\sphinxAtStartPar
Write the ground VTK and obj files.

\end{fulllineitems}


\end{fulllineitems}


\sphinxstepscope


\subsection{htrdrPy.geometry module}
\label{\detokenize{htrdrPy.geometry:module-htrdrPy.geometry}}\label{\detokenize{htrdrPy.geometry:htrdrpy-geometry-module}}\label{\detokenize{htrdrPy.geometry::doc}}\index{module@\spxentry{module}!htrdrPy.geometry@\spxentry{htrdrPy.geometry}}\index{htrdrPy.geometry@\spxentry{htrdrPy.geometry}!module@\spxentry{module}}\index{Geometry (class in htrdrPy.geometry)@\spxentry{Geometry}\spxextra{class in htrdrPy.geometry}}

\begin{fulllineitems}
\phantomsection\label{\detokenize{htrdrPy.geometry:htrdrPy.geometry.Geometry}}
\pysigstartsignatures
\pysiglinewithargsret
{\sphinxbfcode{\sphinxupquote{\DUrole{k}{class}\DUrole{w}{ }}}\sphinxcode{\sphinxupquote{htrdrPy.geometry.}}\sphinxbfcode{\sphinxupquote{Geometry}}}
{\sphinxparam{\DUrole{n}{source}\DUrole{o}{=}\DUrole{default_value}{None}}\sphinxparamcomma \sphinxparam{\DUrole{n}{camera}\DUrole{o}{=}\DUrole{default_value}{None}}\sphinxparamcomma \sphinxparam{\DUrole{n}{image}\DUrole{o}{=}\DUrole{default_value}{None}}\sphinxparamcomma \sphinxparam{\DUrole{n}{volrad}\DUrole{o}{=}\DUrole{default_value}{None}}\sphinxparamcomma \sphinxparam{\DUrole{n}{case}\DUrole{o}{=}\DUrole{default_value}{None}}}
{}
\pysigstopsignatures
\sphinxAtStartPar
Bases: \sphinxcode{\sphinxupquote{object}}

\sphinxAtStartPar
\sphinxcode{\sphinxupquote{htrdrPy.Geometry}} is a class that aims at managing all information
related to the observation geometry. It handles the camera, source and image
properties as well as the mesh on which running the volumic radiative budget
calculations, if running on this mode.

\sphinxAtStartPar
The Geometry module handles the positioning, orientation and properties
of the camera and source.  The data can be provided directly when
creating an instance as distinct dictionnaries for the source, the
camera and the image, with the following keys:
\begin{quote}\begin{description}
\sphinxlineitem{Parameters}\begin{itemize}
\item {} 
\sphinxAtStartPar
\sphinxstyleliteralstrong{\sphinxupquote{source}} (\sphinxstyleliteralemphasis{\sphinxupquote{dict}}\sphinxstyleliteralemphasis{\sphinxupquote{, }}\sphinxstyleliteralemphasis{\sphinxupquote{optional}}) \textendash{} 
\sphinxAtStartPar
Dictionnary containing the information on the source, with the
following keys:
\begin{itemize}
\item {} \begin{description}
\sphinxlineitem{”longitude”}{[}float{]}
\sphinxAtStartPar
Longitude of the source {[}°{]}.

\end{description}

\item {} \begin{description}
\sphinxlineitem{”latitude”}{[}float{]}
\sphinxAtStartPar
Latitude of the source {[}°{]}.

\end{description}

\item {} \begin{description}
\sphinxlineitem{”distance” :}
\sphinxAtStartPar
Distance of the source {[}m{]}.

\end{description}

\item {} \begin{description}
\sphinxlineitem{”radius”}{[}float{]}
\sphinxAtStartPar
Radius of the source {[}m{]}.

\end{description}

\item {} \begin{description}
\sphinxlineitem{”temperature”}{[}float, optional{]}
\sphinxAtStartPar
Surface temperature of the source {[}K{]} (used to calculate
the Planck’s function)

\end{description}

\item {} \begin{description}
\sphinxlineitem{”radiance”}{[}str, optional{]}
\sphinxAtStartPar
Path to a radiance file in htrdr readable format.

\end{description}

\item {} \begin{description}
\sphinxlineitem{”spectrum”}{[}\sphinxcode{\sphinxupquote{numpy.ndarray}}, optional{]}
\sphinxAtStartPar
2\sphinxhyphen{}D array containing the spectrum (shape=(nWavelength,2)).
The first column contains the wavelength {[}nm{]} and the second
contains the radiance {[}W/m2/sr/nm{]}) at the surface of the
source.

\end{description}

\end{itemize}


\item {} 
\sphinxAtStartPar
\sphinxstyleliteralstrong{\sphinxupquote{camera}} (\sphinxstyleliteralemphasis{\sphinxupquote{dict}}\sphinxstyleliteralemphasis{\sphinxupquote{, }}\sphinxstyleliteralemphasis{\sphinxupquote{optional}}) \textendash{} 
\sphinxAtStartPar
Dictionnary containing the information on the camera, with the
following keys:
\begin{itemize}
\item {} \begin{description}
\sphinxlineitem{”position”}{[}\sphinxcode{\sphinxupquote{numpy.ndarray}}{]}
\sphinxAtStartPar
Position of the camera in cartesian coordinates (shape=(3),
{[}m{]}). The origin corresponds to the center of the observed
target.

\end{description}

\item {} \begin{description}
\sphinxlineitem{”target”}{[}\sphinxcode{\sphinxupquote{numpy.ndarray}}{]}
\sphinxAtStartPar
Position of the target in cartesian coordinates (shape=(3),
{[}m{]}). The origin corresponds to the center of the observed
target. This is NOT the line of sight but the position
vector of the target.

\end{description}

\item {} \begin{description}
\sphinxlineitem{”field of view”}{[}float{]}
\sphinxAtStartPar
Vertical field of view of the camera {[}°{]} (the horizontal
field of view is calculated via scaling by the image aspect
ratio, assuming square pixels).

\end{description}

\item {} \begin{description}
\sphinxlineitem{”roll”}{[}\sphinxcode{\sphinxupquote{numpy.ndarray}}, optional{]}
\sphinxAtStartPar
Vector setting the upward direction of the camera (shape=(3),
{[}m{]}) i.e. a vector in the pixel plane to turn the camera
around the line of sight. If not provided, it is calculated
as perpendicular to the line of sight and to z axis (or x
axis if the z axis corresponds to the line of sight).

\end{description}

\end{itemize}


\item {} 
\sphinxAtStartPar
\sphinxstyleliteralstrong{\sphinxupquote{image}} (\sphinxstyleliteralemphasis{\sphinxupquote{dict}}\sphinxstyleliteralemphasis{\sphinxupquote{, }}\sphinxstyleliteralemphasis{\sphinxupquote{optional}}) \textendash{} 
\sphinxAtStartPar
Dictionnary containing the information on the image, with the
following keys:
\begin{itemize}
\item {} \begin{description}
\sphinxlineitem{”definition”}{[}array\sphinxhyphen{}like{]}
\sphinxAtStartPar
Definition (number of pixels) of the image (shape=(2), the
first value is the horizontal number of pixel and the second
value is the vertical pixel count).

\end{description}

\item {} \begin{description}
\sphinxlineitem{”sampling”}{[}int{]}
\sphinxAtStartPar
Number of rays to sample for each pixel.

\end{description}

\end{itemize}


\end{itemize}

\end{description}\end{quote}
\index{exportGeometry() (htrdrPy.geometry.Geometry method)@\spxentry{exportGeometry()}\spxextra{htrdrPy.geometry.Geometry method}}

\begin{fulllineitems}
\phantomsection\label{\detokenize{htrdrPy.geometry:htrdrPy.geometry.Geometry.exportGeometry}}
\pysigstartsignatures
\pysiglinewithargsret
{\sphinxbfcode{\sphinxupquote{exportGeometry}}}
{}
{}
\pysigstopsignatures
\sphinxAtStartPar
Export the geometry (source, camera and image data) in a
geometry\_\{case\}.json file stored in the “geometries/” repository

\end{fulllineitems}

\index{geometryFromAPIE() (htrdrPy.geometry.Geometry method)@\spxentry{geometryFromAPIE()}\spxextra{htrdrPy.geometry.Geometry method}}

\begin{fulllineitems}
\phantomsection\label{\detokenize{htrdrPy.geometry:htrdrPy.geometry.Geometry.geometryFromAPIE}}
\pysigstartsignatures
\pysiglinewithargsret
{\sphinxbfcode{\sphinxupquote{geometryFromAPIE}}}
{\sphinxparam{\DUrole{n}{observation}}\sphinxparamcomma \sphinxparam{\DUrole{n}{distance}}\sphinxparamcomma \sphinxparam{\DUrole{n}{radius}}\sphinxparamcomma \sphinxparam{\DUrole{n}{cameraFOV}}\sphinxparamcomma \sphinxparam{\DUrole{n}{sourceDist}}\sphinxparamcomma \sphinxparam{\DUrole{n}{sourceRad}}\sphinxparamcomma \sphinxparam{\DUrole{n}{sourceTemp}\DUrole{o}{=}\DUrole{default_value}{None}}\sphinxparamcomma \sphinxparam{\DUrole{n}{radianceFile}\DUrole{o}{=}\DUrole{default_value}{None}}\sphinxparamcomma \sphinxparam{\DUrole{n}{spectrum}\DUrole{o}{=}\DUrole{default_value}{None}}}
{}
\pysigstopsignatures
\sphinxAtStartPar
Calulate the observation geometry from Azimut, Phase, Incident and
Emergent angles.
\begin{quote}\begin{description}
\sphinxlineitem{Parameters}\begin{itemize}
\item {} 
\sphinxAtStartPar
\sphinxstyleliteralstrong{\sphinxupquote{observation}} (\sphinxstyleliteralemphasis{\sphinxupquote{dict}}) \textendash{} 
\sphinxAtStartPar
Dictionnary containing the geometry of the observation with the
following items:
\begin{itemize}
\item {} \begin{description}
\sphinxlineitem{”azimut”}{[}float{]}
\sphinxAtStartPar
Angle between the projected incidence and the projected
emergence {[}°{]}.

\end{description}

\item {} \begin{description}
\sphinxlineitem{”phase”}{[}float{]}
\sphinxAtStartPar
Angle between incident rays (directly from the source) and
the line of sight {[}°{]}.

\end{description}

\item {} \begin{description}
\sphinxlineitem{”incidence”}{[}float{]}
\sphinxAtStartPar
Angle between the incident rays (directly from the source)
and the normal to the ground at the observed loaction {[}°{]}.

\end{description}

\item {} \begin{description}
\sphinxlineitem{”emergence”}{[}float{]}
\sphinxAtStartPar
Angle between the normal to the line of sight and the normal
to the ground at the observed loaction {[}°{]}.

\end{description}

\end{itemize}


\item {} 
\sphinxAtStartPar
\sphinxstyleliteralstrong{\sphinxupquote{distance}} (\sphinxstyleliteralemphasis{\sphinxupquote{float}}) \textendash{} Distance from the camera to the target {[}m{]}.

\item {} 
\sphinxAtStartPar
\sphinxstyleliteralstrong{\sphinxupquote{radius}} (\sphinxstyleliteralemphasis{\sphinxupquote{float}}) \textendash{} Radius of the planet to locate the target on the surface {[}m{]}.

\item {} 
\sphinxAtStartPar
\sphinxstyleliteralstrong{\sphinxupquote{cameraFOV}} (\sphinxstyleliteralemphasis{\sphinxupquote{float}}) \textendash{} Field of view of the camera (c.f. \sphinxcode{\sphinxupquote{htrdrPy.Geometry.setCamera}}).

\item {} 
\sphinxAtStartPar
\sphinxstyleliteralstrong{\sphinxupquote{sourceDist}} (\sphinxstyleliteralemphasis{\sphinxupquote{float}}) \textendash{} Distance of the source {[}m{]} (c.f. \sphinxcode{\sphinxupquote{htrdrPy.Geometry.setSource}}).

\item {} 
\sphinxAtStartPar
\sphinxstyleliteralstrong{\sphinxupquote{sourceRad}} (\sphinxstyleliteralemphasis{\sphinxupquote{float}}) \textendash{} Radius of the source {[}m{]} (c.f. \sphinxcode{\sphinxupquote{htrdrPy.Geometry.setSource}}).

\item {} 
\sphinxAtStartPar
\sphinxstyleliteralstrong{\sphinxupquote{spectrum}} (\sphinxstyleliteralemphasis{\sphinxupquote{float}}\sphinxstyleliteralemphasis{\sphinxupquote{, }}\sphinxstyleliteralemphasis{\sphinxupquote{optional}}) \textendash{} Spectrum of the source (c.f. \sphinxcode{\sphinxupquote{htrdrpy.geometry.setsource}}).

\item {} 
\sphinxAtStartPar
\sphinxstyleliteralstrong{\sphinxupquote{radianceFile}} (\sphinxstyleliteralemphasis{\sphinxupquote{float}}\sphinxstyleliteralemphasis{\sphinxupquote{, }}\sphinxstyleliteralemphasis{\sphinxupquote{optional}}) \textendash{} Path to the radiance file of the source (c.f.
\sphinxtitleref{htrdrpy.geometry.setsource}).

\item {} 
\sphinxAtStartPar
\sphinxstyleliteralstrong{\sphinxupquote{sourcetemp}} (\sphinxstyleliteralemphasis{\sphinxupquote{float}}\sphinxstyleliteralemphasis{\sphinxupquote{, }}\sphinxstyleliteralemphasis{\sphinxupquote{optional}}) \textendash{} Temperature of the source (c.f. \sphinxcode{\sphinxupquote{htrdrpy.geometry.setsource}}).

\end{itemize}

\end{description}\end{quote}

\end{fulllineitems}

\index{plotGeometry() (htrdrPy.geometry.Geometry method)@\spxentry{plotGeometry()}\spxextra{htrdrPy.geometry.Geometry method}}

\begin{fulllineitems}
\phantomsection\label{\detokenize{htrdrPy.geometry:htrdrPy.geometry.Geometry.plotGeometry}}
\pysigstartsignatures
\pysiglinewithargsret
{\sphinxbfcode{\sphinxupquote{plotGeometry}}}
{\sphinxparam{\DUrole{n}{ax}}\sphinxparamcomma \sphinxparam{\DUrole{n}{radius}}}
{}
\pysigstopsignatures
\sphinxAtStartPar
Plot the gometry: the observed planet, the line of sight (blue vector),
the source direction (red vector), the camera plan (black vectors) and
the field of view (green vectors).
\begin{quote}\begin{description}
\sphinxlineitem{Parameters}\begin{itemize}
\item {} 
\sphinxAtStartPar
\sphinxstyleliteralstrong{\sphinxupquote{ax}} (\sphinxstyleliteralemphasis{\sphinxupquote{matplotlib.pyplot.Axes}}) \textendash{} Axe on which draxing the plot. It must be a 3D axe.

\item {} 
\sphinxAtStartPar
\sphinxstyleliteralstrong{\sphinxupquote{radius}} (\sphinxstyleliteralemphasis{\sphinxupquote{float}}) \textendash{} Radius of the planet {[}m{]}.

\end{itemize}

\end{description}\end{quote}

\begin{sphinxadmonition}{warning}{Warning:}
\sphinxAtStartPar
matplotlib.pyplot 3d projection has some issues with vector orientation.
If they have the right direction, they may not have the right sens.
\end{sphinxadmonition}

\end{fulllineitems}

\index{setCamera() (htrdrPy.geometry.Geometry method)@\spxentry{setCamera()}\spxextra{htrdrPy.geometry.Geometry method}}

\begin{fulllineitems}
\phantomsection\label{\detokenize{htrdrPy.geometry:htrdrPy.geometry.Geometry.setCamera}}
\pysigstartsignatures
\pysiglinewithargsret
{\sphinxbfcode{\sphinxupquote{setCamera}}}
{\sphinxparam{\DUrole{n}{position}}\sphinxparamcomma \sphinxparam{\DUrole{n}{targetPosition}}\sphinxparamcomma \sphinxparam{\DUrole{n}{fieldOfView}}\sphinxparamcomma \sphinxparam{\DUrole{n}{roll}\DUrole{o}{=}\DUrole{default_value}{None}}}
{}
\pysigstopsignatures
\sphinxAtStartPar
Setup the camera position, orientation and field of view
\begin{quote}\begin{description}
\sphinxlineitem{Parameters}\begin{itemize}
\item {} 
\sphinxAtStartPar
\sphinxstyleliteralstrong{\sphinxupquote{position}} (\sphinxcode{\sphinxupquote{numpy.ndarray}}) \textendash{} Position of the camera in cartesian coordinates (shape=(3),
{[}m{]}). The origin corresponds to the center of the observed
target.

\item {} 
\sphinxAtStartPar
\sphinxstyleliteralstrong{\sphinxupquote{targetPosition}} (\sphinxcode{\sphinxupquote{numpy.ndarray}}) \textendash{} Position of the target in cartesian coordinates (shape=(3),
{[}m{]}). The origin corresponds to the center of the observed
target. This is NOT the line of sight but the position
vector of the target.

\item {} 
\sphinxAtStartPar
\sphinxstyleliteralstrong{\sphinxupquote{fieldOfView}} (\sphinxstyleliteralemphasis{\sphinxupquote{float}}) \textendash{} Vertical field of view of the camera {[}°{]} (the horizontal
field of view is calculated via scaling by the image aspect
ratio, assuming square pixels).

\item {} 
\sphinxAtStartPar
\sphinxstyleliteralstrong{\sphinxupquote{roll}} (\sphinxcode{\sphinxupquote{numpy.ndarray}}, optional) \textendash{} Vector setting the upward direction of the camera (shape=(3),
{[}m{]}) i.e. a vector in the pixel plane to turn the camera
around the line of sight. If not provided, it is calculated
as perpendicular to the line of sight and to z axis (or x
axis if the z axis corresponds to the line of sight).

\end{itemize}

\end{description}\end{quote}

\end{fulllineitems}

\index{setImage() (htrdrPy.geometry.Geometry method)@\spxentry{setImage()}\spxextra{htrdrPy.geometry.Geometry method}}

\begin{fulllineitems}
\phantomsection\label{\detokenize{htrdrPy.geometry:htrdrPy.geometry.Geometry.setImage}}
\pysigstartsignatures
\pysiglinewithargsret
{\sphinxbfcode{\sphinxupquote{setImage}}}
{\sphinxparam{\DUrole{n}{definition}}\sphinxparamcomma \sphinxparam{\DUrole{n}{sampling}}}
{}
\pysigstopsignatures
\sphinxAtStartPar
Setup the image properties
\begin{quote}\begin{description}
\sphinxlineitem{Parameters}\begin{itemize}
\item {} 
\sphinxAtStartPar
\sphinxstyleliteralstrong{\sphinxupquote{definition"}} (\sphinxstyleliteralemphasis{\sphinxupquote{array\sphinxhyphen{}like}}) \textendash{} Definition (number of pixels) of the image (shape=(2), the
first value is the horizontal number of pixel and the second
value is the vertical pixel count).

\item {} 
\sphinxAtStartPar
\sphinxstyleliteralstrong{\sphinxupquote{sampling"}} (\sphinxstyleliteralemphasis{\sphinxupquote{int}}) \textendash{} Number of rays to sample for each pixel.

\end{itemize}

\end{description}\end{quote}

\end{fulllineitems}

\index{setSource() (htrdrPy.geometry.Geometry method)@\spxentry{setSource()}\spxextra{htrdrPy.geometry.Geometry method}}

\begin{fulllineitems}
\phantomsection\label{\detokenize{htrdrPy.geometry:htrdrPy.geometry.Geometry.setSource}}
\pysigstartsignatures
\pysiglinewithargsret
{\sphinxbfcode{\sphinxupquote{setSource}}}
{\sphinxparam{\DUrole{n}{longitude}}\sphinxparamcomma \sphinxparam{\DUrole{n}{latitude}}\sphinxparamcomma \sphinxparam{\DUrole{n}{distance}}\sphinxparamcomma \sphinxparam{\DUrole{n}{radius}}\sphinxparamcomma \sphinxparam{\DUrole{n}{temperature}\DUrole{o}{=}\DUrole{default_value}{None}}\sphinxparamcomma \sphinxparam{\DUrole{n}{radianceFile}\DUrole{o}{=}\DUrole{default_value}{None}}\sphinxparamcomma \sphinxparam{\DUrole{n}{spectrum}\DUrole{o}{=}\DUrole{default_value}{None}}}
{}
\pysigstopsignatures
\sphinxAtStartPar
Setup the source properties.
\begin{quote}\begin{description}
\sphinxlineitem{Parameters}\begin{itemize}
\item {} 
\sphinxAtStartPar
\sphinxstyleliteralstrong{\sphinxupquote{longitude}} (\sphinxstyleliteralemphasis{\sphinxupquote{float}}) \textendash{} Longitude of the source {[}°{]}.

\item {} 
\sphinxAtStartPar
\sphinxstyleliteralstrong{\sphinxupquote{latitude}} (\sphinxstyleliteralemphasis{\sphinxupquote{float}}) \textendash{} Latitude of the source {[}°{]}.

\item {} 
\sphinxAtStartPar
\sphinxstyleliteralstrong{\sphinxupquote{distance}} \textendash{} Distance of the source {[}m{]}.

\item {} 
\sphinxAtStartPar
\sphinxstyleliteralstrong{\sphinxupquote{radius}} (\sphinxstyleliteralemphasis{\sphinxupquote{float}}) \textendash{} Radius of the source {[}m{]}.

\item {} 
\sphinxAtStartPar
\sphinxstyleliteralstrong{\sphinxupquote{temperature}} (\sphinxstyleliteralemphasis{\sphinxupquote{float}}\sphinxstyleliteralemphasis{\sphinxupquote{, }}\sphinxstyleliteralemphasis{\sphinxupquote{optional}}) \textendash{} Surface temperature of the source {[}K{]} (used to calculate
the Planck’s function)

\item {} 
\sphinxAtStartPar
\sphinxstyleliteralstrong{\sphinxupquote{radianceFile}} (\sphinxstyleliteralemphasis{\sphinxupquote{str}}\sphinxstyleliteralemphasis{\sphinxupquote{, }}\sphinxstyleliteralemphasis{\sphinxupquote{optional}}) \textendash{} Path to a radiance file in htrdr readable format.

\item {} 
\sphinxAtStartPar
\sphinxstyleliteralstrong{\sphinxupquote{spectrum}} (\sphinxcode{\sphinxupquote{numpy.ndarray}}, optional) \textendash{} 2\sphinxhyphen{}D array containing the spectrum (shape=(nWavelength,2)).
The first column contains the wavelength {[}nm{]} and the second
contains the radiance {[}W/m2/sr/nm{]}) at the surface of the
source.

\end{itemize}

\end{description}\end{quote}
\subsubsection*{Notes}

\sphinxAtStartPar
Whereas \sphinxcode{\sphinxupquote{temperature}}, \sphinxcode{\sphinxupquote{radianceFile}} and \sphinxcode{\sphinxupquote{spectrum}} are optional,
at least one of those must be provided. If more than one is provided,
the \sphinxcode{\sphinxupquote{spectrum}} is taken in priority, then the \sphinxcode{\sphinxupquote{radianceFile}} and
finally the \sphinxcode{\sphinxupquote{temperature}}.

\end{fulllineitems}

\index{setSpectralCumulDist() (htrdrPy.geometry.Geometry method)@\spxentry{setSpectralCumulDist()}\spxextra{htrdrPy.geometry.Geometry method}}

\begin{fulllineitems}
\phantomsection\label{\detokenize{htrdrPy.geometry:htrdrPy.geometry.Geometry.setSpectralCumulDist}}
\pysigstartsignatures
\pysiglinewithargsret
{\sphinxbfcode{\sphinxupquote{setSpectralCumulDist}}}
{\sphinxparam{\DUrole{n}{pdf}}\sphinxparamcomma \sphinxparam{\DUrole{n}{data}\DUrole{o}{=}\DUrole{default_value}{None}}}
{}
\pysigstopsignatures
\sphinxAtStartPar
Setup the probability density function to used for sampling the
wavelength and correlated\sphinxhyphen{}k coefficient.
\begin{quote}\begin{description}
\sphinxlineitem{Parameters}
\sphinxAtStartPar
\sphinxstyleliteralstrong{\sphinxupquote{pdf}} (\sphinxcode{\sphinxupquote{numpy.ndarray}}) \textendash{} Table containing the cumulative distribution used to sample the
wavelength and k\sphinxhyphen{}coefficient
(shape=(nAltitudes,nLatitudes,nLongitudes,nSpectralElements)).
nSpectralElements correspond to nWavelength * nCoeff.

\end{description}\end{quote}

\end{fulllineitems}

\index{setVolrad() (htrdrPy.geometry.Geometry method)@\spxentry{setVolrad()}\spxextra{htrdrPy.geometry.Geometry method}}

\begin{fulllineitems}
\phantomsection\label{\detokenize{htrdrPy.geometry:htrdrPy.geometry.Geometry.setVolrad}}
\pysigstartsignatures
\pysiglinewithargsret
{\sphinxbfcode{\sphinxupquote{setVolrad}}}
{\sphinxparam{\DUrole{n}{sampling}}\sphinxparamcomma \sphinxparam{\DUrole{n}{mesh}\DUrole{o}{=}\DUrole{default_value}{\textquotesingle{}origin\textquotesingle{}}}\sphinxparamcomma \sphinxparam{\DUrole{n}{args}\DUrole{o}{=}\DUrole{default_value}{None}}}
{}
\pysigstopsignatures
\sphinxAtStartPar
Setup the volumic radiative budget properties.
The volumic radiative budget mode evaluate the divergence of the flux
within each provided tetrahedron. This method handles the generation of
the mesh on which the calulation will be conducted.
\begin{quote}\begin{description}
\sphinxlineitem{Parameters}\begin{itemize}
\item {} 
\sphinxAtStartPar
\sphinxstyleliteralstrong{\sphinxupquote{sampling}} (\sphinxstyleliteralemphasis{\sphinxupquote{int}}) \textendash{} Number of ray to sample for each tetrahedron.

\item {} 
\sphinxAtStartPar
\sphinxstyleliteralstrong{\sphinxupquote{mesh}} (\sphinxstyleliteralemphasis{\sphinxupquote{\{"origin"}}\sphinxstyleliteralemphasis{\sphinxupquote{, }}\sphinxstyleliteralemphasis{\sphinxupquote{"makeColumnPP"}}\sphinxstyleliteralemphasis{\sphinxupquote{, }}\sphinxstyleliteralemphasis{\sphinxupquote{"makeFromCellCoord"}}\sphinxstyleliteralemphasis{\sphinxupquote{, }}\sphinxstyleliteralemphasis{\sphinxupquote{"makeSliceAltLat"}}\sphinxstyleliteralemphasis{\sphinxupquote{, }}\sphinxstyleliteralemphasis{\sphinxupquote{"extractFromData"\}}}\sphinxstyleliteralemphasis{\sphinxupquote{, }}\sphinxstyleliteralemphasis{\sphinxupquote{default "origin"}}) \textendash{} 
\sphinxAtStartPar
Method to build the mesh on which the radiative budget calculation
is realized.


\begin{savenotes}\sphinxattablestart
\sphinxthistablewithglobalstyle
\centering
\sphinxcapstartof{table}
\sphinxthecaptionisattop
\sphinxcaption{Building method}\label{\detokenize{htrdrPy.geometry:id1}}
\sphinxaftertopcaption
\begin{tabular}[t]{\X{25}{100}\X{75}{100}}
\sphinxtoprule
\sphinxtableatstartofbodyhook
\sphinxAtStartPar
”origin”
&
\sphinxAtStartPar
Use the original atmosphere mesh.
\\
\sphinxhline
\sphinxAtStartPar
”makeColumnPP”
&
\sphinxAtStartPar
Generate one plane\sphinxhyphen{}parallel column.
\\
\sphinxhline
\sphinxAtStartPar
”makeFromCellCoord”
&
\sphinxAtStartPar
Generate a complete sphere from a table of coordinates.
\\
\sphinxhline
\sphinxAtStartPar
”makeSliceAltLat”
&
\sphinxAtStartPar
Generate a slice in altitude / latitude from a table of
coordinates.
\\
\sphinxhline
\sphinxAtStartPar
”extractFromData”
&
\sphinxAtStartPar
Extract a list of cells from an already generated mesh.
\\
\sphinxbottomrule
\end{tabular}
\sphinxtableafterendhook\par
\sphinxattableend\end{savenotes}


\item {} 
\sphinxAtStartPar
\sphinxstyleliteralstrong{\sphinxupquote{args}} (\sphinxstyleliteralemphasis{\sphinxupquote{tuple}}) \textendash{} 
\sphinxAtStartPar
Tuple containing the arguments required by the chosen mesh
generation method.


\begin{savenotes}\sphinxattablestart
\sphinxthistablewithglobalstyle
\centering
\sphinxcapstartof{table}
\sphinxthecaptionisattop
\sphinxcaption{Arguments}\label{\detokenize{htrdrPy.geometry:id2}}
\sphinxaftertopcaption
\begin{tabular}[t]{\X{25}{100}\X{75}{100}}
\sphinxtoprule
\sphinxstyletheadfamily 
\sphinxAtStartPar
Method
&\sphinxstyletheadfamily 
\sphinxAtStartPar
Required arguments
\\
\sphinxmidrule
\sphinxtableatstartofbodyhook
\sphinxAtStartPar
”origin”
&\\
\sphinxhline
\sphinxAtStartPar
”makeColumnPP”
&\begin{description}
\sphinxlineitem{altitudes}{[}\sphinxcode{\sphinxupquote{numpy.ndarray}}{]}
\sphinxAtStartPar
Array continaing the altitudes {[}m{]}, either at the center
of the cells or at the boundaries (shape = nLevel or
nLayer).

\sphinxlineitem{hwidth}{[}float{]}
\sphinxAtStartPar
Horizontal dimension of the squared base column {[}m{]}.

\sphinxlineitem{center}{[}bool, default: True{]}
\sphinxAtStartPar
True if the altitudes represent the cell centers and
False if they correspond to the boundaries of the cells.

\end{description}
\\
\sphinxhline
\sphinxAtStartPar
”makeFromCellCoord”
&\begin{description}
\sphinxlineitem{cellCoord}{[}\sphinxcode{\sphinxupquote{numpy.ndarray}}{]}
\sphinxAtStartPar
Coordinates of the cells centers or cell interface
centers (c.f. \sphinxcode{\sphinxupquote{onLevels}},
shape=(nAltitudes,nLatitudes,nLongitudes,3), {[}m{]})

\sphinxlineitem{radius}{[}float{]}
\sphinxAtStartPar
Radius of the planet {[}m{]}.

\sphinxlineitem{poles}{[}bool, default False{]}
\sphinxAtStartPar
Whether or not the first and last latitudes correspond
to the poles in the given array of coordinates.

\sphinxlineitem{onLevels}{[}bool, default False{]}
\sphinxAtStartPar
Whether or not the altitudes provided in the coordinates
are given on the levels or in the center of the cells.

\end{description}
\\
\sphinxhline
\sphinxAtStartPar
”makeSliceAltLat”
&\begin{description}
\sphinxlineitem{cellCoord}{[}\sphinxcode{\sphinxupquote{numpy.ndarray}}{]}
\sphinxAtStartPar
Coordinates of the cells centers or cell interface
centers (c.f. \sphinxcode{\sphinxupquote{onLevels}},
shape=(nAltitudes,nLatitudes,3), {[}m{]})

\sphinxlineitem{radius}{[}float{]}
\sphinxAtStartPar
Radius of the planet {[}m{]}.

\sphinxlineitem{dLongitude}{[}float{]}
\sphinxAtStartPar
Longitude width of the slice {[}°{]}.

\sphinxlineitem{poles}{[}bool, default False{]}
\sphinxAtStartPar
Whether or not the first and last latitudes correspond
to the poles in the given array of coordinates.

\sphinxlineitem{onLevels}{[}bool, default False{]}
\sphinxAtStartPar
Whether or not the altitudes provided in the coordinates
are given on the levels or in the center of the cells.

\end{description}
\\
\sphinxhline
\sphinxAtStartPar
”extractFromData”
&\begin{description}
\sphinxlineitem{data}{[}\sphinxcode{\sphinxupquote{htrdrPy.Data}} or \sphinxcode{\sphinxupquote{htrdrPy.Geometry}}{]}
\sphinxAtStartPar
Object instance containing an already generated mesh
from which extracting the list of cells. This is
non\sphinxhyphen{}desctructive for the original \sphinxcode{\sphinxupquote{htrdrPy.Data}} or
\sphinxcode{\sphinxupquote{htrdrPy.Geometry}} and only affects the current
instance that “steals” the wanted cells.

\sphinxlineitem{cells}{[}\sphinxcode{\sphinxupquote{numpy.ndarray}}{]}
\sphinxAtStartPar
Array of the cells to be extracted (shape=(nCell,3))
with the altitude, latitude and longitude indices.

\end{description}
\\
\sphinxbottomrule
\end{tabular}
\sphinxtableafterendhook\par
\sphinxattableend\end{savenotes}


\end{itemize}

\end{description}\end{quote}

\end{fulllineitems}


\end{fulllineitems}


\sphinxstepscope


\subsection{htrdrPy.script module}
\label{\detokenize{htrdrPy.script:module-htrdrPy.script}}\label{\detokenize{htrdrPy.script:htrdrpy-script-module}}\label{\detokenize{htrdrPy.script::doc}}\index{module@\spxentry{module}!htrdrPy.script@\spxentry{htrdrPy.script}}\index{htrdrPy.script@\spxentry{htrdrPy.script}!module@\spxentry{module}}\index{Script (class in htrdrPy.script)@\spxentry{Script}\spxextra{class in htrdrPy.script}}

\begin{fulllineitems}
\phantomsection\label{\detokenize{htrdrPy.script:htrdrPy.script.Script}}
\pysigstartsignatures
\pysiglinewithargsret
{\sphinxbfcode{\sphinxupquote{\DUrole{k}{class}\DUrole{w}{ }}}\sphinxcode{\sphinxupquote{htrdrPy.script.}}\sphinxbfcode{\sphinxupquote{Script}}}
{\sphinxparam{\DUrole{n}{case}\DUrole{o}{=}\DUrole{default_value}{\textquotesingle{}\textquotesingle{}}}\sphinxparamcomma \sphinxparam{\DUrole{n}{threadFlag}\DUrole{o}{=}\DUrole{default_value}{\textquotesingle{}\textquotesingle{}}}\sphinxparamcomma \sphinxparam{\DUrole{n}{MPIcmd}\DUrole{o}{=}\DUrole{default_value}{\textquotesingle{}\textquotesingle{}}}\sphinxparamcomma \sphinxparam{\DUrole{n}{verbose}\DUrole{o}{=}\DUrole{default_value}{True}}}
{}
\pysigstopsignatures
\sphinxAtStartPar
Bases: \sphinxcode{\sphinxupquote{object}}

\sphinxAtStartPar
The \sphinxcode{\sphinxupquote{htrdrPy.Script}} module aims at creating a callable (a function)
taking as input a \sphinxcode{\sphinxupquote{htrdrPy.Data}} object.
\subsubsection*{Examples}

\sphinxAtStartPar
The first step is to create an instance of \sphinxcode{\sphinxupquote{htrdrPy.Script}}:

\begin{sphinxVerbatim}[commandchars=\\\{\}]
\PYG{g+gp}{\PYGZgt{}\PYGZgt{}\PYGZgt{} }\PYG{n}{script} \PYG{o}{=} \PYG{n}{htrdrPy}\PYG{o}{.}\PYG{n}{script}\PYG{p}{(}\PYG{n}{case}\PYG{o}{=}\PYG{l+s+s2}{\PYGZdq{}}\PYG{l+s+s2}{caseName}\PYG{l+s+s2}{\PYGZdq{}}\PYG{p}{)}
\end{sphinxVerbatim}

\sphinxAtStartPar
The second step is to define the kind of script to be executed (see the
documentation for all possibilities). If none of the predefined script suits
your use, you can use the \sphinxcode{\sphinxupquote{htrdrPy.Script.startMultipleObsGeometry}}
method.

\begin{sphinxVerbatim}[commandchars=\\\{\}]
\PYG{g+gp}{\PYGZgt{}\PYGZgt{}\PYGZgt{} }\PYG{n}{script}\PYG{o}{.}\PYG{n}{startMultipleObsGeometry}\PYG{p}{(}\PYG{o}{.}\PYG{o}{.}\PYG{o}{.}\PYG{p}{)}
\end{sphinxVerbatim}

\sphinxAtStartPar
Finally, you can call the script on an already defined \sphinxcode{\sphinxupquote{htrdrPy.Data}}
object:

\begin{sphinxVerbatim}[commandchars=\\\{\}]
\PYG{g+gp}{\PYGZgt{}\PYGZgt{}\PYGZgt{} }\PYG{n}{script}\PYG{p}{(}\PYG{n}{data}\PYG{p}{)}
\end{sphinxVerbatim}

\sphinxAtStartPar
The Script module aims at creating a callable (a function) to be called on a Data object.
\begin{quote}\begin{description}
\sphinxlineitem{Parameters}\begin{itemize}
\item {} 
\sphinxAtStartPar
\sphinxstyleliteralstrong{\sphinxupquote{case}} (\sphinxstyleliteralemphasis{\sphinxupquote{str}}\sphinxstyleliteralemphasis{\sphinxupquote{, }}\sphinxstyleliteralemphasis{\sphinxupquote{optional}}) \textendash{} String to identify the script. This is mostly usefull in case
different scripts are used on the same \sphinxcode{\sphinxupquote{htrdrPy.Data}} instance.
The output files being stored in the same folder (see
\sphinxcode{\sphinxupquote{htrdrPy.Data}} documentation), the case name is used to
differentiate the output files of the different scripts.

\item {} 
\sphinxAtStartPar
\sphinxstyleliteralstrong{\sphinxupquote{threadFlag}} (\sphinxstyleliteralemphasis{\sphinxupquote{str}}\sphinxstyleliteralemphasis{\sphinxupquote{, }}\sphinxstyleliteralemphasis{\sphinxupquote{optional}}) \textendash{} Thread option to use in the htrdr command. Should take the form “\sphinxhyphen{}t
\textless{}num\textgreater{}” where \textless{}num\textgreater{} is the number of threads to be used. If left
empty, the maximum number of threads (corresponding to the number of
virtual cores on the computer) will be used.

\item {} 
\sphinxAtStartPar
\sphinxstyleliteralstrong{\sphinxupquote{MPIcmd}} (\sphinxstyleliteralemphasis{\sphinxupquote{str}}\sphinxstyleliteralemphasis{\sphinxupquote{, }}\sphinxstyleliteralemphasis{\sphinxupquote{optional}}) \textendash{} MPI command to pass before the call to htrdr\sphinxhyphen{}planets, depending on
the MPI runner on your system. For insatnce, it could be an ‘mpirun’
command on many computers, or a ‘srun’ command on a slurm\sphinxhyphen{}based
supercalculator.

\item {} 
\sphinxAtStartPar
\sphinxstyleliteralstrong{\sphinxupquote{verbose}} (\sphinxstyleliteralemphasis{\sphinxupquote{bool}}\sphinxstyleliteralemphasis{\sphinxupquote{, }}\sphinxstyleliteralemphasis{\sphinxupquote{default True}}) \textendash{} Whether or not activate the verbose of htrdr\sphinxhyphen{}planets.

\end{itemize}

\end{description}\end{quote}
\index{bandIntegratedImage() (htrdrPy.script.Script method)@\spxentry{bandIntegratedImage()}\spxextra{htrdrPy.script.Script method}}

\begin{fulllineitems}
\phantomsection\label{\detokenize{htrdrPy.script:htrdrPy.script.Script.bandIntegratedImage}}
\pysigstartsignatures
\pysiglinewithargsret
{\sphinxbfcode{\sphinxupquote{bandIntegratedImage}}}
{\sphinxparam{\DUrole{n}{geometry}\DUrole{p}{:}\DUrole{w}{ }\DUrole{n}{{\hyperref[\detokenize{htrdrPy.geometry:htrdrPy.geometry.Geometry}]{\sphinxcrossref{Geometry}}}}}\sphinxparamcomma \sphinxparam{\DUrole{n}{kind}}\sphinxparamcomma \sphinxparam{\DUrole{n}{wavelengthLow}}\sphinxparamcomma \sphinxparam{\DUrole{n}{wavelengthUp}}}
{}
\pysigstopsignatures
\sphinxAtStartPar
Set up the script to calculate an image integrated over a given spectral
range.
\begin{quote}\begin{description}
\sphinxlineitem{Parameters}\begin{itemize}
\item {} 
\sphinxAtStartPar
\sphinxstyleliteralstrong{\sphinxupquote{geometry}} (\sphinxcode{\sphinxupquote{htrdrPy.Geometry}}) \textendash{} A \sphinxcode{\sphinxupquote{htrdrPy.Geometry}} object previously created and set up.

\item {} 
\sphinxAtStartPar
\sphinxstyleliteralstrong{\sphinxupquote{kind}} (\sphinxstyleliteralemphasis{\sphinxupquote{\{"sw"}}\sphinxstyleliteralemphasis{\sphinxupquote{, }}\sphinxstyleliteralemphasis{\sphinxupquote{"lw"\}}}) \textendash{} Type of calculation. “sw” for a calculation with an external source,
and “lw” to use the atmosphere emission as source.

\item {} 
\sphinxAtStartPar
\sphinxstyleliteralstrong{\sphinxupquote{wavelengthLow}} (\sphinxstyleliteralemphasis{\sphinxupquote{float}}) \textendash{} Lower boundary wavelength {[}m{]}.

\item {} 
\sphinxAtStartPar
\sphinxstyleliteralstrong{\sphinxupquote{wavelengthUp}} (\sphinxstyleliteralemphasis{\sphinxupquote{float}}) \textendash{} Upper boundary wavelength {[}m{]}.

\end{itemize}

\end{description}\end{quote}

\end{fulllineitems}

\index{compositeRBG() (htrdrPy.script.Script method)@\spxentry{compositeRBG()}\spxextra{htrdrPy.script.Script method}}

\begin{fulllineitems}
\phantomsection\label{\detokenize{htrdrPy.script:htrdrPy.script.Script.compositeRBG}}
\pysigstartsignatures
\pysiglinewithargsret
{\sphinxbfcode{\sphinxupquote{compositeRBG}}}
{\sphinxparam{\DUrole{n}{geometry}\DUrole{p}{:}\DUrole{w}{ }\DUrole{n}{{\hyperref[\detokenize{htrdrPy.geometry:htrdrPy.geometry.Geometry}]{\sphinxcrossref{Geometry}}}}}\sphinxparamcomma \sphinxparam{\DUrole{n}{kind}}\sphinxparamcomma \sphinxparam{\DUrole{n}{wavelengthRed}}\sphinxparamcomma \sphinxparam{\DUrole{n}{wavelengthGreen}}\sphinxparamcomma \sphinxparam{\DUrole{n}{wavelengthBlue}}}
{}
\pysigstopsignatures
\sphinxAtStartPar
Set up the script to calculate the composite image from 3 monochromatic
images.
\begin{quote}\begin{description}
\sphinxlineitem{Parameters}\begin{itemize}
\item {} 
\sphinxAtStartPar
\sphinxstyleliteralstrong{\sphinxupquote{geometry}} (\sphinxcode{\sphinxupquote{htrdrPy.Geometry}}) \textendash{} A \sphinxcode{\sphinxupquote{htrdrPy.Geometry}} object previously created and set up.

\item {} 
\sphinxAtStartPar
\sphinxstyleliteralstrong{\sphinxupquote{kind}} (\sphinxstyleliteralemphasis{\sphinxupquote{\{"sw"}}\sphinxstyleliteralemphasis{\sphinxupquote{, }}\sphinxstyleliteralemphasis{\sphinxupquote{"lw"\}}}) \textendash{} Type of calculation. “sw” for a calculation with an external source,
and “lw” to use the atmosphere emission as source.

\item {} 
\sphinxAtStartPar
\sphinxstyleliteralstrong{\sphinxupquote{wavelengthRed}} (\sphinxstyleliteralemphasis{\sphinxupquote{float}}) \textendash{} Wavelength for red chanel image {[}m{]}.

\item {} 
\sphinxAtStartPar
\sphinxstyleliteralstrong{\sphinxupquote{wavelengthGreen}} (\sphinxstyleliteralemphasis{\sphinxupquote{float}}) \textendash{} Wavelength for green chanel image {[}m{]}.

\item {} 
\sphinxAtStartPar
\sphinxstyleliteralstrong{\sphinxupquote{wavelengthBlue}} (\sphinxstyleliteralemphasis{\sphinxupquote{float}}) \textendash{} Wavelength for red chanel image {[}m{]}.

\end{itemize}

\end{description}\end{quote}

\begin{sphinxadmonition}{warning}{Warning:}
\sphinxAtStartPar
This script has not bee tested yet and may not work correctly. This
should be treated in future versions.
\end{sphinxadmonition}

\end{fulllineitems}

\index{imageRatio() (htrdrPy.script.Script method)@\spxentry{imageRatio()}\spxextra{htrdrPy.script.Script method}}

\begin{fulllineitems}
\phantomsection\label{\detokenize{htrdrPy.script:htrdrPy.script.Script.imageRatio}}
\pysigstartsignatures
\pysiglinewithargsret
{\sphinxbfcode{\sphinxupquote{imageRatio}}}
{\sphinxparam{\DUrole{n}{geometry}\DUrole{p}{:}\DUrole{w}{ }\DUrole{n}{{\hyperref[\detokenize{htrdrPy.geometry:htrdrPy.geometry.Geometry}]{\sphinxcrossref{Geometry}}}}}\sphinxparamcomma \sphinxparam{\DUrole{n}{kind}}\sphinxparamcomma \sphinxparam{\DUrole{n}{wavelengthNum}}\sphinxparamcomma \sphinxparam{\DUrole{n}{wavelengthDen}}}
{}
\pysigstopsignatures
\sphinxAtStartPar
Set up the script to calculate the ratio of 2 monochromatic images.
\begin{quote}\begin{description}
\sphinxlineitem{Parameters}\begin{itemize}
\item {} 
\sphinxAtStartPar
\sphinxstyleliteralstrong{\sphinxupquote{geometry}} (\sphinxcode{\sphinxupquote{htrdrPy.Geometry}}) \textendash{} A \sphinxcode{\sphinxupquote{htrdrPy.Geometry}} object previously created and set up.

\item {} 
\sphinxAtStartPar
\sphinxstyleliteralstrong{\sphinxupquote{kind}} (\sphinxstyleliteralemphasis{\sphinxupquote{\{"sw"}}\sphinxstyleliteralemphasis{\sphinxupquote{, }}\sphinxstyleliteralemphasis{\sphinxupquote{"lw"\}}}) \textendash{} Type of calculation. “sw” for a calculation with an external source,
and “lw” to use the atmosphere emission as source.

\item {} 
\sphinxAtStartPar
\sphinxstyleliteralstrong{\sphinxupquote{wavelengthNum}} (\sphinxstyleliteralemphasis{\sphinxupquote{float}}) \textendash{} Wavelength for numerator image {[}m{]}.

\item {} 
\sphinxAtStartPar
\sphinxstyleliteralstrong{\sphinxupquote{wavelengthDen}} (\sphinxstyleliteralemphasis{\sphinxupquote{float}}) \textendash{} Wavelength for denominator image {[}m{]}.

\end{itemize}

\end{description}\end{quote}

\end{fulllineitems}

\index{monochromaticImage() (htrdrPy.script.Script method)@\spxentry{monochromaticImage()}\spxextra{htrdrPy.script.Script method}}

\begin{fulllineitems}
\phantomsection\label{\detokenize{htrdrPy.script:htrdrPy.script.Script.monochromaticImage}}
\pysigstartsignatures
\pysiglinewithargsret
{\sphinxbfcode{\sphinxupquote{monochromaticImage}}}
{\sphinxparam{\DUrole{n}{geometry}\DUrole{p}{:}\DUrole{w}{ }\DUrole{n}{{\hyperref[\detokenize{htrdrPy.geometry:htrdrPy.geometry.Geometry}]{\sphinxcrossref{Geometry}}}}}\sphinxparamcomma \sphinxparam{\DUrole{n}{kind}}\sphinxparamcomma \sphinxparam{\DUrole{n}{wavelength}}}
{}
\pysigstopsignatures
\sphinxAtStartPar
Set up the script to calculate a monochromatic image.
\begin{quote}\begin{description}
\sphinxlineitem{Parameters}\begin{itemize}
\item {} 
\sphinxAtStartPar
\sphinxstyleliteralstrong{\sphinxupquote{geometry}} (\sphinxcode{\sphinxupquote{htrdrPy.Geometry}}) \textendash{} A \sphinxcode{\sphinxupquote{htrdrPy.Geometry}} object previously created and set up.

\item {} 
\sphinxAtStartPar
\sphinxstyleliteralstrong{\sphinxupquote{kind}} (\sphinxstyleliteralemphasis{\sphinxupquote{\{"sw"}}\sphinxstyleliteralemphasis{\sphinxupquote{, }}\sphinxstyleliteralemphasis{\sphinxupquote{"lw"\}}}) \textendash{} Type of calculation. “sw” for a calculation with an external source,
and “lw” to use the atmosphere emission as source.

\item {} 
\sphinxAtStartPar
\sphinxstyleliteralstrong{\sphinxupquote{wavelength}} (\sphinxstyleliteralemphasis{\sphinxupquote{float}}) \textendash{} Wavelength to use for the calculation {[}m{]}.

\end{itemize}

\end{description}\end{quote}

\end{fulllineitems}

\index{reflectanceSpectrum() (htrdrPy.script.Script method)@\spxentry{reflectanceSpectrum()}\spxextra{htrdrPy.script.Script method}}

\begin{fulllineitems}
\phantomsection\label{\detokenize{htrdrPy.script:htrdrPy.script.Script.reflectanceSpectrum}}
\pysigstartsignatures
\pysiglinewithargsret
{\sphinxbfcode{\sphinxupquote{reflectanceSpectrum}}}
{\sphinxparam{\DUrole{n}{geometry}\DUrole{p}{:}\DUrole{w}{ }\DUrole{n}{{\hyperref[\detokenize{htrdrPy.geometry:htrdrPy.geometry.Geometry}]{\sphinxcrossref{Geometry}}}}}\sphinxparamcomma \sphinxparam{\DUrole{n}{kind}}\sphinxparamcomma \sphinxparam{\DUrole{n}{wavelengths}}\sphinxparamcomma \sphinxparam{\DUrole{n}{bandWidths}\DUrole{o}{=}\DUrole{default_value}{None}}}
{}
\pysigstopsignatures
\sphinxAtStartPar
Start mutliple runs of htrdr\sphinxhyphen{}planets to calculate a reflectance
spectrum, producing an output for each wavelength.
\begin{quote}\begin{description}
\sphinxlineitem{Parameters}\begin{itemize}
\item {} 
\sphinxAtStartPar
\sphinxstyleliteralstrong{\sphinxupquote{geometry}} (\sphinxcode{\sphinxupquote{htrdrPy.Geometry}}) \textendash{} A \sphinxcode{\sphinxupquote{htrdrPy.Geometry}} object previously created and set up.

\item {} 
\sphinxAtStartPar
\sphinxstyleliteralstrong{\sphinxupquote{kind}} (\sphinxstyleliteralemphasis{\sphinxupquote{\{"sw"}}\sphinxstyleliteralemphasis{\sphinxupquote{, }}\sphinxstyleliteralemphasis{\sphinxupquote{"lw"\}}}) \textendash{} Type of calculation. “sw” for a calculation with an external source,
and “lw” to use the atmosphere emission as source.

\item {} 
\sphinxAtStartPar
\sphinxstyleliteralstrong{\sphinxupquote{wavelength}} (\sphinxstyleliteralemphasis{\sphinxupquote{float}}) \textendash{} Wavelength to use for the calculation {[}m{]}.

\item {} 
\sphinxAtStartPar
\sphinxstyleliteralstrong{\sphinxupquote{bandWidths}} (\sphinxcode{\sphinxupquote{numpy.ndarray}}) \textendash{} Integration band around each wavelength. If not provided, the
calculation is monochromatic (shape=(nWavelength), {[}m{]}).

\end{itemize}

\end{description}\end{quote}

\end{fulllineitems}

\index{spectrum() (htrdrPy.script.Script method)@\spxentry{spectrum()}\spxextra{htrdrPy.script.Script method}}

\begin{fulllineitems}
\phantomsection\label{\detokenize{htrdrPy.script:htrdrPy.script.Script.spectrum}}
\pysigstartsignatures
\pysiglinewithargsret
{\sphinxbfcode{\sphinxupquote{spectrum}}}
{\sphinxparam{\DUrole{n}{geometry}\DUrole{p}{:}\DUrole{w}{ }\DUrole{n}{{\hyperref[\detokenize{htrdrPy.geometry:htrdrPy.geometry.Geometry}]{\sphinxcrossref{Geometry}}}}}\sphinxparamcomma \sphinxparam{\DUrole{n}{kind}}\sphinxparamcomma \sphinxparam{\DUrole{n}{wavelengths}}\sphinxparamcomma \sphinxparam{\DUrole{n}{bandWidths}\DUrole{o}{=}\DUrole{default_value}{None}}}
{}
\pysigstopsignatures
\sphinxAtStartPar
Start mutliple runs of htrdr\sphinxhyphen{}planets to calculate a spectrum, producing
an output for each wavelength.
\begin{quote}\begin{description}
\sphinxlineitem{Parameters}\begin{itemize}
\item {} 
\sphinxAtStartPar
\sphinxstyleliteralstrong{\sphinxupquote{geometry}} (\sphinxcode{\sphinxupquote{htrdrPy.Geometry}}) \textendash{} A \sphinxcode{\sphinxupquote{htrdrPy.Geometry}} object previously created and set up.

\item {} 
\sphinxAtStartPar
\sphinxstyleliteralstrong{\sphinxupquote{kind}} (\sphinxstyleliteralemphasis{\sphinxupquote{\{"sw"}}\sphinxstyleliteralemphasis{\sphinxupquote{, }}\sphinxstyleliteralemphasis{\sphinxupquote{"lw"\}}}) \textendash{} Type of calculation. “sw” for a calculation with an external source,
and “lw” to use the atmosphere emission as source.

\item {} 
\sphinxAtStartPar
\sphinxstyleliteralstrong{\sphinxupquote{wavelength}} (\sphinxstyleliteralemphasis{\sphinxupquote{float}}) \textendash{} Wavelength to use for the calculation {[}m{]}.

\item {} 
\sphinxAtStartPar
\sphinxstyleliteralstrong{\sphinxupquote{bandWidths}} (\sphinxcode{\sphinxupquote{numpy.ndarray}}) \textendash{} Integration band around each wavelength. If not provided, the
calculation is monochromatic (shape=(nWavelength), {[}m{]}).

\end{itemize}

\end{description}\end{quote}

\end{fulllineitems}

\index{startMultipleObsGeometry() (htrdrPy.script.Script method)@\spxentry{startMultipleObsGeometry()}\spxextra{htrdrPy.script.Script method}}

\begin{fulllineitems}
\phantomsection\label{\detokenize{htrdrPy.script:htrdrPy.script.Script.startMultipleObsGeometry}}
\pysigstartsignatures
\pysiglinewithargsret
{\sphinxbfcode{\sphinxupquote{startMultipleObsGeometry}}}
{\sphinxparam{\DUrole{n}{obsList}\DUrole{p}{:}\DUrole{w}{ }\DUrole{n}{list\DUrole{p}{{[}}{\hyperref[\detokenize{htrdrPy.geometry:htrdrPy.geometry.Geometry}]{\sphinxcrossref{Geometry}}}\DUrole{p}{{]}}}}\sphinxparamcomma \sphinxparam{\DUrole{n}{wavelength}}}
{}
\pysigstopsignatures
\sphinxAtStartPar
Start multiple runs with different observation geometries but the same planet inputs
\begin{quote}\begin{description}
\sphinxlineitem{Parameters}\begin{itemize}
\item {} 
\sphinxAtStartPar
\sphinxstyleliteralstrong{\sphinxupquote{obsList}} (\sphinxstyleliteralemphasis{\sphinxupquote{array\sphinxhyphen{}like}}) \textendash{} List of \sphinxcode{\sphinxupquote{htrdrPy.Geometry}} instances.

\item {} 
\sphinxAtStartPar
\sphinxstyleliteralstrong{\sphinxupquote{wavelength}} (\sphinxstyleliteralemphasis{\sphinxupquote{dict}}) \textendash{} 
\sphinxAtStartPar
Dictionnary containing the spectral information with the following
items:
\begin{itemize}
\item {} \begin{description}
\sphinxlineitem{”type”}{[}\{“cie\_xyz”, “sw”, “lw”\}{]}
\sphinxAtStartPar
Type of calculation.

\end{description}

\item {} \begin{description}
\sphinxlineitem{”low”}{[}float{]}
\sphinxAtStartPar
Lower bound of integration band {[}m{]}.

\end{description}

\item {} \begin{description}
\sphinxlineitem{”up”}{[}float{]}
\sphinxAtStartPar
Upper bound of integration band {[}m{]} (if monochromatic
calculation, “up” = “low”).

\end{description}

\end{itemize}


\end{itemize}

\end{description}\end{quote}

\end{fulllineitems}

\index{startRadBudgetGCM() (htrdrPy.script.Script method)@\spxentry{startRadBudgetGCM()}\spxextra{htrdrPy.script.Script method}}

\begin{fulllineitems}
\phantomsection\label{\detokenize{htrdrPy.script:htrdrPy.script.Script.startRadBudgetGCM}}
\pysigstartsignatures
\pysiglinewithargsret
{\sphinxbfcode{\sphinxupquote{startRadBudgetGCM}}}
{\sphinxparam{\DUrole{n}{geometry}\DUrole{p}{:}\DUrole{w}{ }\DUrole{n}{{\hyperref[\detokenize{htrdrPy.geometry:htrdrPy.geometry.Geometry}]{\sphinxcrossref{Geometry}}}}}\sphinxparamcomma \sphinxparam{\DUrole{n}{kind}}}
{}
\pysigstopsignatures
\sphinxAtStartPar
Set up the script to calculate athe radiative budget of each GCM cell.
\begin{quote}\begin{description}
\sphinxlineitem{Parameters}\begin{itemize}
\item {} 
\sphinxAtStartPar
\sphinxstyleliteralstrong{\sphinxupquote{geometry}} (\sphinxcode{\sphinxupquote{htrdrPy.Geometry}}) \textendash{} A \sphinxcode{\sphinxupquote{htrdrPy.Geometry}} object previously created and set up.

\item {} 
\sphinxAtStartPar
\sphinxstyleliteralstrong{\sphinxupquote{kind}} (\sphinxstyleliteralemphasis{\sphinxupquote{\{"sw"}}\sphinxstyleliteralemphasis{\sphinxupquote{, }}\sphinxstyleliteralemphasis{\sphinxupquote{"lw"\}}}) \textendash{} Type of calculation. “sw” for a calculation with an external source,
and “lw” to use the atmosphere emission as source.

\end{itemize}

\end{description}\end{quote}

\end{fulllineitems}

\index{visibleImage() (htrdrPy.script.Script method)@\spxentry{visibleImage()}\spxextra{htrdrPy.script.Script method}}

\begin{fulllineitems}
\phantomsection\label{\detokenize{htrdrPy.script:htrdrPy.script.Script.visibleImage}}
\pysigstartsignatures
\pysiglinewithargsret
{\sphinxbfcode{\sphinxupquote{visibleImage}}}
{\sphinxparam{\DUrole{n}{geometry}\DUrole{p}{:}\DUrole{w}{ }\DUrole{n}{{\hyperref[\detokenize{htrdrPy.geometry:htrdrPy.geometry.Geometry}]{\sphinxcrossref{Geometry}}}}}}
{}
\pysigstopsignatures
\sphinxAtStartPar
Set up the script to calculate a visible (RGB) image.
\begin{quote}\begin{description}
\sphinxlineitem{Parameters}
\sphinxAtStartPar
\sphinxstyleliteralstrong{\sphinxupquote{geometry}} (\sphinxcode{\sphinxupquote{htrdrPy.Geometry}}) \textendash{} A \sphinxcode{\sphinxupquote{htrdrPy.Geometry}} object previously created and set up.

\end{description}\end{quote}

\end{fulllineitems}


\end{fulllineitems}

\index{loadScript() (in module htrdrPy.script)@\spxentry{loadScript()}\spxextra{in module htrdrPy.script}}

\begin{fulllineitems}
\phantomsection\label{\detokenize{htrdrPy.script:htrdrPy.script.loadScript}}
\pysigstartsignatures
\pysiglinewithargsret
{\sphinxcode{\sphinxupquote{htrdrPy.script.}}\sphinxbfcode{\sphinxupquote{loadScript}}}
{\sphinxparam{\DUrole{n}{filename}}}
{}
\pysigstopsignatures
\sphinxAtStartPar
Load a \sphinxcode{\sphinxupquote{htrdrPy.Script}} object from the binary file where it is saved
(generated after the call on a \sphinxcode{\sphinxupquote{htrdrPy.Data}} object).
\begin{quote}\begin{description}
\sphinxlineitem{Parameters}
\sphinxAtStartPar
\sphinxstyleliteralstrong{\sphinxupquote{filename}} (\sphinxstyleliteralemphasis{\sphinxupquote{str}}) \textendash{} Path to the binary file to be loaded.

\end{description}\end{quote}

\end{fulllineitems}


\sphinxstepscope


\subsection{htrdrPy.postprocess module}
\label{\detokenize{htrdrPy.postprocess:module-htrdrPy.postprocess}}\label{\detokenize{htrdrPy.postprocess:htrdrpy-postprocess-module}}\label{\detokenize{htrdrPy.postprocess::doc}}\index{module@\spxentry{module}!htrdrPy.postprocess@\spxentry{htrdrPy.postprocess}}\index{htrdrPy.postprocess@\spxentry{htrdrPy.postprocess}!module@\spxentry{module}}\index{Postprocess (class in htrdrPy.postprocess)@\spxentry{Postprocess}\spxextra{class in htrdrPy.postprocess}}

\begin{fulllineitems}
\phantomsection\label{\detokenize{htrdrPy.postprocess:htrdrPy.postprocess.Postprocess}}
\pysigstartsignatures
\pysiglinewithargsret
{\sphinxbfcode{\sphinxupquote{\DUrole{k}{class}\DUrole{w}{ }}}\sphinxcode{\sphinxupquote{htrdrPy.postprocess.}}\sphinxbfcode{\sphinxupquote{Postprocess}}}
{\sphinxparam{\DUrole{n}{script}\DUrole{p}{:}\DUrole{w}{ }\DUrole{n}{{\hyperref[\detokenize{htrdrPy.script:htrdrPy.script.Script}]{\sphinxcrossref{Script}}}}\DUrole{w}{ }\DUrole{o}{=}\DUrole{w}{ }\DUrole{default_value}{None}}\sphinxparamcomma \sphinxparam{\DUrole{n}{skip}\DUrole{o}{=}\DUrole{default_value}{False}}\sphinxparamcomma \sphinxparam{\DUrole{n}{localPath}\DUrole{o}{=}\DUrole{default_value}{False}}\sphinxparamcomma \sphinxparam{\DUrole{n}{kwargs}\DUrole{o}{=}\DUrole{default_value}{\{\}}}}
{}
\pysigstopsignatures
\sphinxAtStartPar
Bases: \sphinxcode{\sphinxupquote{object}}

\sphinxAtStartPar
The \sphinxcode{\sphinxupquote{htrdrPy.Postprocess}} module aims at providing tools to process the
outputs generated by htrdr\sphinxhyphen{}planets or auto\sphinxhyphen{}process the outputs based on the
\sphinxcode{\sphinxupquote{htrdrPy.Script}} used.
\subsubsection*{Examples}

\sphinxAtStartPar
In the case of the use of a predefined \sphinxcode{\sphinxupquote{htrdrPy.Script}} other than
\sphinxcode{\sphinxupquote{htrdrPy.Script.startMultipleObsGeometry}}, the user simply needs to
provide the \sphinxcode{\sphinxupquote{htrdrPy.Script}} used at the creatin of the
\sphinxcode{\sphinxupquote{htrdrPy.Postprocess}} instance.

\begin{sphinxVerbatim}[commandchars=\\\{\}]
\PYG{g+gp}{\PYGZgt{}\PYGZgt{}\PYGZgt{} }\PYG{n}{pp} \PYG{o}{=} \PYG{n}{htrdrPy}\PYG{o}{.}\PYG{n}{Postprocess}\PYG{p}{(}\PYG{n}{script}\PYG{p}{)}
\end{sphinxVerbatim}

\sphinxAtStartPar
This will generate the output files (‘.json’, images, …) in a
‘result\_\{name\}/’ folder, where the ‘name’ is the name of the
\sphinxcode{\sphinxupquote{htrdrPy.Data}} instance passed to the \sphinxcode{\sphinxupquote{htrdrPy.Script}}.
\begin{quote}\begin{description}
\sphinxlineitem{Parameters}\begin{itemize}
\item {} 
\sphinxAtStartPar
\sphinxstyleliteralstrong{\sphinxupquote{script}} (\sphinxcode{\sphinxupquote{htrdrPy.Script}}, optional) \textendash{} \sphinxcode{\sphinxupquote{htrdrPy.Script}} object which as been called on a \sphinxcode{\sphinxupquote{htrdrPy.Data}}
object.

\item {} 
\sphinxAtStartPar
\sphinxstyleliteralstrong{\sphinxupquote{skip}} (\sphinxstyleliteralemphasis{\sphinxupquote{bool}}\sphinxstyleliteralemphasis{\sphinxupquote{, }}\sphinxstyleliteralemphasis{\sphinxupquote{default False}}) \textendash{} Whether or not to skip the default post\sphinxhyphen{}processing operation.
Default is False.

\item {} 
\sphinxAtStartPar
\sphinxstyleliteralstrong{\sphinxupquote{localPath}} (\sphinxstyleliteralemphasis{\sphinxupquote{bool}}\sphinxstyleliteralemphasis{\sphinxupquote{, }}\sphinxstyleliteralemphasis{\sphinxupquote{default False}}) \textendash{} Whether or not to create the result directory at the root of the the
script running the post\sphinxhyphen{}processing. If True, the result directory
will be located in the same directory as the running post\sphinxhyphen{}processing
script, otherwise, it is located in the directory where
htrdr\sphinxhyphen{}planets was run.

\item {} 
\sphinxAtStartPar
\sphinxstyleliteralstrong{\sphinxupquote{kwargs}} (\sphinxstyleliteralemphasis{\sphinxupquote{dict}}) \textendash{} 
\sphinxAtStartPar
Dictionary containing the arguments for the post\sphinxhyphen{}processing routine.
You will find the required items, depending on the script used,
in the following table.


\begin{savenotes}\sphinxattablestart
\sphinxthistablewithglobalstyle
\centering
\sphinxcapstartof{table}
\sphinxthecaptionisattop
\sphinxcaption{kwargs}\label{\detokenize{htrdrPy.postprocess:id1}}
\sphinxaftertopcaption
\begin{tabular}[t]{\X{25}{100}\X{75}{100}}
\sphinxtoprule
\sphinxstyletheadfamily 
\sphinxAtStartPar
Script
&\sphinxstyletheadfamily 
\sphinxAtStartPar
Items
\\
\sphinxmidrule
\sphinxtableatstartofbodyhook\begin{itemize}
\item {} 
\sphinxAtStartPar
\sphinxcode{\sphinxupquote{htrdPy.Script.visibleImage}} or

\item {} 
\sphinxAtStartPar
\sphinxcode{\sphinxupquote{htrdPy.Script.monochromaticImage}} or

\item {} 
\sphinxAtStartPar
\sphinxcode{\sphinxupquote{htrdPy.Script.bandIntegratedImage}}

\end{itemize}
&\begin{itemize}
\item {} \begin{description}
\sphinxlineitem{”exposure”}{[}float, default 1.{]}
\sphinxAtStartPar
Exposure time {[}s{]} for the generation of the image.

\end{description}

\item {} \begin{description}
\sphinxlineitem{”cmap”}{[}str, default “inferno”{]}
\sphinxAtStartPar
Color map to be used. See htrdr documentation for all
possibilities.

\end{description}

\end{itemize}
\\
\sphinxhline
\sphinxAtStartPar
\sphinxcode{\sphinxupquote{htrdPy.Script.imageRatio}}
&\begin{itemize}
\item {} \begin{description}
\sphinxlineitem{”threshold”}{[}float, default 0.{]}
\sphinxAtStartPar
Threshold limit below which the data are considered zero
(expressed relatively to the data maximale value:
threshold * max(data)).

\end{description}

\end{itemize}
\\
\sphinxhline\begin{itemize}
\item {} 
\sphinxAtStartPar
\sphinxcode{\sphinxupquote{htrdPy.Script.spectrum}} or

\item {} 
\sphinxAtStartPar
\sphinxcode{\sphinxupquote{htrdPy.Script.reflectanceSpectrum}} or

\item {} 
\sphinxAtStartPar
\sphinxcode{\sphinxupquote{htrdPy.Script.compositeRBG}}

\end{itemize}
&\\
\sphinxhline
\sphinxAtStartPar
\sphinxcode{\sphinxupquote{htrdPy.Script.startRadBudgetGCM}}
&\begin{itemize}
\item {} \begin{description}
\sphinxlineitem{”heatCapacity”}{[}float, optional{]}
\sphinxAtStartPar
Heat capacity of the atmosphere {[}J/kg/K{]}. If not
provided, the calculation of the heating rate is not
realised and the resulting file only contains the flux
divergence.

\end{description}

\item {} \begin{description}
\sphinxlineitem{”rho”}{[}float, optional{]}
\sphinxAtStartPar
Mass volume density of the atmosphere {[}kg/m3{]}. If not
provided, the calculation of the heating rate is not
realised and the resulting file only contains the flux
divergence.

\end{description}

\end{itemize}
\\
\sphinxbottomrule
\end{tabular}
\sphinxtableafterendhook\par
\sphinxattableend\end{savenotes}


\end{itemize}

\end{description}\end{quote}
\subsubsection*{Notes}

\sphinxAtStartPar
Im most cases, the module recognize the kind of script and automatically
apply the required post\sphinxhyphen{}process function. This is not the case for
script based on “startMultipleObsGeometry”, as the post\sphinxhyphen{}processing
routine to be used is not trivial. A bunch of additional methods are
therefore provided.
\index{extractMeanRadianceFromOutput() (htrdrPy.postprocess.Postprocess method)@\spxentry{extractMeanRadianceFromOutput()}\spxextra{htrdrPy.postprocess.Postprocess method}}

\begin{fulllineitems}
\phantomsection\label{\detokenize{htrdrPy.postprocess:htrdrPy.postprocess.Postprocess.extractMeanRadianceFromOutput}}
\pysigstartsignatures
\pysiglinewithargsret
{\sphinxbfcode{\sphinxupquote{extractMeanRadianceFromOutput}}}
{\sphinxparam{\DUrole{n}{file}}\sphinxparamcomma \sphinxparam{\DUrole{n}{time}\DUrole{o}{=}\DUrole{default_value}{False}}}
{}
\pysigstopsignatures
\sphinxAtStartPar
Calculate the average radiance and standard deviation over all pixels
\begin{quote}\begin{description}
\sphinxlineitem{Parameters}
\sphinxAtStartPar
\sphinxstyleliteralstrong{\sphinxupquote{file}} (\sphinxstyleliteralemphasis{\sphinxupquote{str}}) \textendash{} Path to the htrdr output file containing the pixels information.

\sphinxlineitem{Returns}
\sphinxAtStartPar
\begin{itemize}
\item {} 
\sphinxAtStartPar
\sphinxstyleemphasis{float} \textendash{} Average radiance over the pixel grid (units are similar to htrdr
outputs).

\item {} 
\sphinxAtStartPar
\sphinxstyleemphasis{float} \textendash{} Standard deviation of the radiance over the pixel grid (units are
similar to htrdr outputs).

\item {} 
\sphinxAtStartPar
\sphinxstyleemphasis{float} \textendash{} Mean computation time per path {[}µs{]}.

\item {} 
\sphinxAtStartPar
\sphinxstyleemphasis{float} \textendash{} Standard deviation of the computation time per path {[}µs{]}.

\end{itemize}


\end{description}\end{quote}

\end{fulllineitems}

\index{extractMeanRadiances() (htrdrPy.postprocess.Postprocess method)@\spxentry{extractMeanRadiances()}\spxextra{htrdrPy.postprocess.Postprocess method}}

\begin{fulllineitems}
\phantomsection\label{\detokenize{htrdrPy.postprocess:htrdrPy.postprocess.Postprocess.extractMeanRadiances}}
\pysigstartsignatures
\pysiglinewithargsret
{\sphinxbfcode{\sphinxupquote{extractMeanRadiances}}}
{\sphinxparam{\DUrole{n}{indices}\DUrole{o}{=}\DUrole{default_value}{False}}}
{}
\pysigstopsignatures
\sphinxAtStartPar
Extract the mean radiances of all (or a subset) images generated by the
Script.
\begin{quote}\begin{description}
\sphinxlineitem{Parameters}
\sphinxAtStartPar
\sphinxstyleliteralstrong{\sphinxupquote{indices}} (\sphinxstyleliteralemphasis{\sphinxupquote{array\sphinxhyphen{}like}}) \textendash{} List containing the indices of the file to be processed. In case
multiple outputs are generated (can be the case for a spectrum, or
using \sphinxcode{\sphinxupquote{htrdrPy.Script.startMultipleGeometry}}) the file index
follows the order of execution of the runs (for a spectrum, the
order in which the wavelength have been given and for
\sphinxcode{\sphinxupquote{htrdrPy.Script.startMultipleGeometry}} the order of the list of
geometries.)

\sphinxlineitem{Returns}
\sphinxAtStartPar
Dictionnary with mean radiances and standard deviations (c.f.
\sphinxcode{\sphinxupquote{htrdrPy.Postprocess.extractMeanRadianceFromOutput}}) for all
requested output files.

\sphinxlineitem{Return type}
\sphinxAtStartPar
dict

\end{description}\end{quote}

\end{fulllineitems}

\index{getImage() (htrdrPy.postprocess.Postprocess method)@\spxentry{getImage()}\spxextra{htrdrPy.postprocess.Postprocess method}}

\begin{fulllineitems}
\phantomsection\label{\detokenize{htrdrPy.postprocess:htrdrPy.postprocess.Postprocess.getImage}}
\pysigstartsignatures
\pysiglinewithargsret
{\sphinxbfcode{\sphinxupquote{getImage}}}
{\sphinxparam{\DUrole{n}{file}}}
{}
\pysigstopsignatures
\sphinxAtStartPar
Recover the image data from a htrdr output file.
\begin{quote}\begin{description}
\sphinxlineitem{Parameters}
\sphinxAtStartPar
\sphinxstyleliteralstrong{\sphinxupquote{file}} (\sphinxstyleliteralemphasis{\sphinxupquote{str}}) \textendash{} Path to the htrdr output file.

\sphinxlineitem{Returns}
\sphinxAtStartPar
\begin{itemize}
\item {} 
\sphinxAtStartPar
\sphinxcode{\sphinxupquote{numpy.ndarray}} \textendash{} Radiances associated to each pixel (shape=(nPixelX,nPixelY)).

\item {} 
\sphinxAtStartPar
\sphinxcode{\sphinxupquote{numpy.ndarray}} \textendash{} Standard deviation associated to each pixel
(shape=(nPixelX,nPixelY)).

\end{itemize}


\end{description}\end{quote}

\end{fulllineitems}

\index{getImages() (htrdrPy.postprocess.Postprocess method)@\spxentry{getImages()}\spxextra{htrdrPy.postprocess.Postprocess method}}

\begin{fulllineitems}
\phantomsection\label{\detokenize{htrdrPy.postprocess:htrdrPy.postprocess.Postprocess.getImages}}
\pysigstartsignatures
\pysiglinewithargsret
{\sphinxbfcode{\sphinxupquote{getImages}}}
{}
{}
\pysigstopsignatures
\sphinxAtStartPar
Recover the images data from all htrdr output files generated by the
Script.
\begin{quote}\begin{description}
\sphinxlineitem{Returns}
\sphinxAtStartPar
Dictionnary with radiances and standard deviations map for all
output files.

\sphinxlineitem{Return type}
\sphinxAtStartPar
dict

\end{description}\end{quote}

\end{fulllineitems}

\index{processImages() (htrdrPy.postprocess.Postprocess method)@\spxentry{processImages()}\spxextra{htrdrPy.postprocess.Postprocess method}}

\begin{fulllineitems}
\phantomsection\label{\detokenize{htrdrPy.postprocess:htrdrPy.postprocess.Postprocess.processImages}}
\pysigstartsignatures
\pysiglinewithargsret
{\sphinxbfcode{\sphinxupquote{processImages}}}
{\sphinxparam{\DUrole{n}{exposure}\DUrole{o}{=}\DUrole{default_value}{1}}\sphinxparamcomma \sphinxparam{\DUrole{n}{cmap}\DUrole{o}{=}\DUrole{default_value}{\textquotesingle{}inferno\textquotesingle{}}}}
{}
\pysigstopsignatures
\sphinxAtStartPar
Generates the image corresponding to all output files.
\begin{quote}\begin{description}
\sphinxlineitem{Parameters}\begin{itemize}
\item {} 
\sphinxAtStartPar
\sphinxstyleliteralstrong{\sphinxupquote{exposure}} (\sphinxstyleliteralemphasis{\sphinxupquote{float}}\sphinxstyleliteralemphasis{\sphinxupquote{, }}\sphinxstyleliteralemphasis{\sphinxupquote{default 1.}}) \textendash{} Exposure time {[}s{]} for the generation of the image.

\item {} 
\sphinxAtStartPar
\sphinxstyleliteralstrong{\sphinxupquote{cmap}} (\sphinxstyleliteralemphasis{\sphinxupquote{str}}\sphinxstyleliteralemphasis{\sphinxupquote{, }}\sphinxstyleliteralemphasis{\sphinxupquote{default "inferno"}}) \textendash{} Color map to be used. See htrdr documentation for all
possibilities.

\end{itemize}

\end{description}\end{quote}

\end{fulllineitems}

\index{processSingleArrayObsSW() (htrdrPy.postprocess.Postprocess method)@\spxentry{processSingleArrayObsSW()}\spxextra{htrdrPy.postprocess.Postprocess method}}

\begin{fulllineitems}
\phantomsection\label{\detokenize{htrdrPy.postprocess:htrdrPy.postprocess.Postprocess.processSingleArrayObsSW}}
\pysigstartsignatures
\pysiglinewithargsret
{\sphinxbfcode{\sphinxupquote{processSingleArrayObsSW}}}
{}
{}
\pysigstopsignatures
\sphinxAtStartPar
Processes the previously generated files in the case of single array
(one of the image definition is 1).
\begin{quote}\begin{description}
\sphinxlineitem{Returns}
\sphinxAtStartPar
Dictionnary containing 2\sphinxhyphen{}D \sphinxcode{\sphinxupquote{numpy.ndarray}} (shape=(nPixel,2)) for
each output file. The first column of the arrays are radiances
(units are similar to htrdr outputs) and the second columns are the
associated standard deviation (units are similar to htrdr outputs).

\sphinxlineitem{Return type}
\sphinxAtStartPar
dict

\end{description}\end{quote}

\end{fulllineitems}


\end{fulllineitems}


\sphinxstepscope


\subsection{htrdrPy.helperFunctions module}
\label{\detokenize{htrdrPy.helperFunctions:module-htrdrPy.helperFunctions}}\label{\detokenize{htrdrPy.helperFunctions:htrdrpy-helperfunctions-module}}\label{\detokenize{htrdrPy.helperFunctions::doc}}\index{module@\spxentry{module}!htrdrPy.helperFunctions@\spxentry{htrdrPy.helperFunctions}}\index{htrdrPy.helperFunctions@\spxentry{htrdrPy.helperFunctions}!module@\spxentry{module}}\index{cart2sphere() (in module htrdrPy.helperFunctions)@\spxentry{cart2sphere()}\spxextra{in module htrdrPy.helperFunctions}}

\begin{fulllineitems}
\phantomsection\label{\detokenize{htrdrPy.helperFunctions:htrdrPy.helperFunctions.cart2sphere}}
\pysigstartsignatures
\pysiglinewithargsret
{\sphinxcode{\sphinxupquote{htrdrPy.helperFunctions.}}\sphinxbfcode{\sphinxupquote{cart2sphere}}}
{\sphinxparam{\DUrole{n}{vec}}}
{}
\pysigstopsignatures
\sphinxAtStartPar
Convert cartesian to spherical coordinates.
\begin{quote}\begin{description}
\sphinxlineitem{Parameters}
\sphinxAtStartPar
\sphinxstyleliteralstrong{\sphinxupquote{vec}} (\sphinxcode{\sphinxupquote{numpy.ndarray}}) \textendash{} Cartesian coordinate array (x {[}m{]}, y {[}m{]}, z {[}m{]}).

\sphinxlineitem{Returns}
\sphinxAtStartPar
Spherical coordinate array (altitude {[}m{]}, latitude {[}°{]}, longitude {[}°{]}).

\sphinxlineitem{Return type}
\sphinxAtStartPar
\sphinxcode{\sphinxupquote{numpy.ndarray}}

\end{description}\end{quote}

\end{fulllineitems}

\index{combineEstimates() (in module htrdrPy.helperFunctions)@\spxentry{combineEstimates()}\spxextra{in module htrdrPy.helperFunctions}}

\begin{fulllineitems}
\phantomsection\label{\detokenize{htrdrPy.helperFunctions:htrdrPy.helperFunctions.combineEstimates}}
\pysigstartsignatures
\pysiglinewithargsret
{\sphinxcode{\sphinxupquote{htrdrPy.helperFunctions.}}\sphinxbfcode{\sphinxupquote{combineEstimates}}}
{\sphinxparam{\DUrole{n}{sumX}}\sphinxparamcomma \sphinxparam{\DUrole{n}{sumXsquare}}\sphinxparamcomma \sphinxparam{\DUrole{n}{numbers}}}
{}
\pysigstopsignatures
\sphinxAtStartPar
Calculate the mean, the variance and the standard deviation of a set of
estimates.
\begin{quote}\begin{description}
\sphinxlineitem{Parameters}\begin{itemize}
\item {} 
\sphinxAtStartPar
\sphinxstyleliteralstrong{\sphinxupquote{sumX}} (\sphinxcode{\sphinxupquote{numpy.ndarray}}) \textendash{} Array containing the sum of the Monte Carlo weights of a list of
estimates.

\item {} 
\sphinxAtStartPar
\sphinxstyleliteralstrong{\sphinxupquote{sumXsquare}} (\sphinxcode{\sphinxupquote{numpy.ndarray}}) \textendash{} Array containing the sum of the square of the  Monte Carlo weights of a
list of estimates.

\item {} 
\sphinxAtStartPar
\sphinxstyleliteralstrong{\sphinxupquote{numbers}} (\sphinxcode{\sphinxupquote{numpy.ndarray}}) \textendash{} Array containing the number of realizations for each estimate.

\end{itemize}

\sphinxlineitem{Returns}
\sphinxAtStartPar
\begin{itemize}
\item {} 
\sphinxAtStartPar
\sphinxstyleemphasis{float} \textendash{} Mean of the combined estimates.

\item {} 
\sphinxAtStartPar
\sphinxstyleemphasis{float} \textendash{} Variance of the combined estimates.

\item {} 
\sphinxAtStartPar
\sphinxstyleemphasis{float} \textendash{} Standard deviation of the combined estimates.

\end{itemize}


\end{description}\end{quote}

\end{fulllineitems}

\index{dplanck\_dT() (in module htrdrPy.helperFunctions)@\spxentry{dplanck\_dT()}\spxextra{in module htrdrPy.helperFunctions}}

\begin{fulllineitems}
\phantomsection\label{\detokenize{htrdrPy.helperFunctions:htrdrPy.helperFunctions.dplanck_dT}}
\pysigstartsignatures
\pysiglinewithargsret
{\sphinxcode{\sphinxupquote{htrdrPy.helperFunctions.}}\sphinxbfcode{\sphinxupquote{dplanck\_dT}}}
{\sphinxparam{\DUrole{n}{T}}\sphinxparamcomma \sphinxparam{\DUrole{n}{wvl}}\sphinxparamcomma \sphinxparam{\DUrole{n}{r\_d}\DUrole{o}{=}\DUrole{default_value}{False}}}
{}
\pysigstopsignatures
\sphinxAtStartPar
Calculate the derivative of the planck emission regarding the temperature in
W/m2/sr/m/K for a surface at temperature T and at wavelengths wvl.
\begin{quote}\begin{description}
\sphinxlineitem{Parameters}\begin{itemize}
\item {} 
\sphinxAtStartPar
\sphinxstyleliteralstrong{\sphinxupquote{T}} (float or \sphinxcode{\sphinxupquote{numpy.ndarray}}) \textendash{} Temperature or N\sphinxhyphen{}D array of temperatures {[}K{]} of the emiting surface.

\item {} 
\sphinxAtStartPar
\sphinxstyleliteralstrong{\sphinxupquote{wvl}} (float or \sphinxcode{\sphinxupquote{numpy.ndarray}}) \textendash{} Wavelength or 1\sphinxhyphen{}D array of wavelengths.

\item {} 
\sphinxAtStartPar
\sphinxstyleliteralstrong{\sphinxupquote{(}}\sphinxstyleliteralstrong{\sphinxupquote{optional}} (\sphinxstyleliteralemphasis{\sphinxupquote{r\_d}}) \textendash{} Shape 2 array containing the source radius and distance, respectively.
If not provided, returns the surface radiance, if given, returns the
radiance received at that distance from the source.

\item {} 
\sphinxAtStartPar
\sphinxstyleliteralstrong{\sphinxupquote{(}}\sphinxstyleliteralstrong{\sphinxupquote{shape=}}\sphinxstyleliteralstrong{\sphinxupquote{(}}\sphinxstyleliteralstrong{\sphinxupquote{2}}\sphinxstyleliteralstrong{\sphinxupquote{)}} (\sphinxstyleliteralemphasis{\sphinxupquote{1\sphinxhyphen{}D array}}) \textendash{} Shape 2 array containing the source radius and distance, respectively.
If not provided, returns the surface radiance, if given, returns the
radiance received at that distance from the source.

\item {} 
\sphinxAtStartPar
\sphinxstyleliteralstrong{\sphinxupquote{{[}}}\sphinxstyleliteralstrong{\sphinxupquote{m}}\sphinxstyleliteralstrong{\sphinxupquote{{]}}}\sphinxstyleliteralstrong{\sphinxupquote{)}}\sphinxstyleliteralstrong{\sphinxupquote{)}} (\sphinxstyleliteralemphasis{\sphinxupquote{float}}) \textendash{} Shape 2 array containing the source radius and distance, respectively.
If not provided, returns the surface radiance, if given, returns the
radiance received at that distance from the source.

\end{itemize}

\sphinxlineitem{Returns}
\sphinxAtStartPar
Radiance either at the surface of the source (if \sphinxcode{\sphinxupquote{r\_d}} not provided)
or at the given distance from the source. The shape depends on the
shape of the parameters provided. If both the \sphinxcode{\sphinxupquote{T}} and \sphinxcode{\sphinxupquote{wvl}} are
floats, the result is a float. If \sphinxcode{\sphinxupquote{T}} is a flaot and \sphinxcode{\sphinxupquote{wvl}} an array,
the result has the length of \sphinxcode{\sphinxupquote{wvl}}. If \sphinxcode{\sphinxupquote{T}} is an array and \sphinxcode{\sphinxupquote{wvl}}
is a float, the result has the shape of \sphinxcode{\sphinxupquote{T}}. Finally, if both \sphinxcode{\sphinxupquote{T}}
and \sphinxcode{\sphinxupquote{wvl}} are arrays (\sphinxcode{\sphinxupquote{T}} has dimension N and \sphinxcode{\sphinxupquote{wvl}} has dimension
1), the result has the N+1 dimensions (the N dimensions of \sphinxcode{\sphinxupquote{T}} plus
the dimension of \sphinxcode{\sphinxupquote{wvl}}).

\sphinxlineitem{Return type}
\sphinxAtStartPar
float or M\sphinxhyphen{}D array (shape=(nWavelength), float {[}W/m2/sr/m{]}))

\end{description}\end{quote}

\end{fulllineitems}

\index{planck() (in module htrdrPy.helperFunctions)@\spxentry{planck()}\spxextra{in module htrdrPy.helperFunctions}}

\begin{fulllineitems}
\phantomsection\label{\detokenize{htrdrPy.helperFunctions:htrdrPy.helperFunctions.planck}}
\pysigstartsignatures
\pysiglinewithargsret
{\sphinxcode{\sphinxupquote{htrdrPy.helperFunctions.}}\sphinxbfcode{\sphinxupquote{planck}}}
{\sphinxparam{\DUrole{n}{T}}\sphinxparamcomma \sphinxparam{\DUrole{n}{wvl}}\sphinxparamcomma \sphinxparam{\DUrole{n}{r\_d}\DUrole{o}{=}\DUrole{default_value}{False}}}
{}
\pysigstopsignatures
\sphinxAtStartPar
Calculate planck emission in W/m2/sr/m for a surface at temperature T and at
wavelengths wvl.
\begin{quote}\begin{description}
\sphinxlineitem{Parameters}\begin{itemize}
\item {} 
\sphinxAtStartPar
\sphinxstyleliteralstrong{\sphinxupquote{T}} (float or \sphinxcode{\sphinxupquote{numpy.ndarray}}) \textendash{} Temperature or N\sphinxhyphen{}D array of temperatures {[}K{]} of the emiting surface.

\item {} 
\sphinxAtStartPar
\sphinxstyleliteralstrong{\sphinxupquote{wvl}} (float or \sphinxcode{\sphinxupquote{numpy.ndarray}}) \textendash{} Wavelength or 1\sphinxhyphen{}D array of wavelengths.

\item {} 
\sphinxAtStartPar
\sphinxstyleliteralstrong{\sphinxupquote{(}}\sphinxstyleliteralstrong{\sphinxupquote{optional}} (\sphinxstyleliteralemphasis{\sphinxupquote{r\_d}}) \textendash{} Shape 2 array containing the source radius and distance, respectively.
If not provided, returns the surface radiance, if given, returns the
radiance received at that distance from the source.

\item {} 
\sphinxAtStartPar
\sphinxstyleliteralstrong{\sphinxupquote{(}}\sphinxstyleliteralstrong{\sphinxupquote{shape=}}\sphinxstyleliteralstrong{\sphinxupquote{(}}\sphinxstyleliteralstrong{\sphinxupquote{2}}\sphinxstyleliteralstrong{\sphinxupquote{)}} (\sphinxstyleliteralemphasis{\sphinxupquote{1\sphinxhyphen{}D array}}) \textendash{} Shape 2 array containing the source radius and distance, respectively.
If not provided, returns the surface radiance, if given, returns the
radiance received at that distance from the source.

\item {} 
\sphinxAtStartPar
\sphinxstyleliteralstrong{\sphinxupquote{{[}}}\sphinxstyleliteralstrong{\sphinxupquote{m}}\sphinxstyleliteralstrong{\sphinxupquote{{]}}}\sphinxstyleliteralstrong{\sphinxupquote{)}}\sphinxstyleliteralstrong{\sphinxupquote{)}} (\sphinxstyleliteralemphasis{\sphinxupquote{float}}) \textendash{} Shape 2 array containing the source radius and distance, respectively.
If not provided, returns the surface radiance, if given, returns the
radiance received at that distance from the source.

\end{itemize}

\sphinxlineitem{Returns}
\sphinxAtStartPar
Radiance either at the surface of the source (if \sphinxcode{\sphinxupquote{r\_d}} not provided)
or at the given distance from the source. The shape depends on the
shape of the parameters provided. If both the \sphinxcode{\sphinxupquote{T}} and \sphinxcode{\sphinxupquote{wvl}} are
floats, the result is a float. If \sphinxcode{\sphinxupquote{T}} is a flaot and \sphinxcode{\sphinxupquote{wvl}} an array,
the result has the length of \sphinxcode{\sphinxupquote{wvl}}. If \sphinxcode{\sphinxupquote{T}} is an array and \sphinxcode{\sphinxupquote{wvl}}
is a float, the result has the shape of \sphinxcode{\sphinxupquote{T}}. Finally, if both \sphinxcode{\sphinxupquote{T}}
and \sphinxcode{\sphinxupquote{wvl}} are arrays (\sphinxcode{\sphinxupquote{T}} has dimension N and \sphinxcode{\sphinxupquote{wvl}} has dimension
1), the result has the N+1 dimensions (the N dimensions of \sphinxcode{\sphinxupquote{T}} plus
the dimension of \sphinxcode{\sphinxupquote{wvl}}).

\sphinxlineitem{Return type}
\sphinxAtStartPar
float or M\sphinxhyphen{}D array (shape=(nWavelength), float {[}W/m2/sr/m{]}))

\end{description}\end{quote}

\end{fulllineitems}

\index{sphere2cart() (in module htrdrPy.helperFunctions)@\spxentry{sphere2cart()}\spxextra{in module htrdrPy.helperFunctions}}

\begin{fulllineitems}
\phantomsection\label{\detokenize{htrdrPy.helperFunctions:htrdrPy.helperFunctions.sphere2cart}}
\pysigstartsignatures
\pysiglinewithargsret
{\sphinxcode{\sphinxupquote{htrdrPy.helperFunctions.}}\sphinxbfcode{\sphinxupquote{sphere2cart}}}
{\sphinxparam{\DUrole{n}{vec}}}
{}
\pysigstopsignatures
\sphinxAtStartPar
Convert spherical to cartesian coordinates.
\begin{quote}\begin{description}
\sphinxlineitem{Parameters}
\sphinxAtStartPar
\sphinxstyleliteralstrong{\sphinxupquote{vec}} (\sphinxcode{\sphinxupquote{numpy.ndarray}}) \textendash{} Spherical coordinate array (altitude {[}m{]}, latitude {[}°{]}, longitude {[}°{]}).

\sphinxlineitem{Returns}
\sphinxAtStartPar
Cartesian coordinate array (x {[}m{]}, y {[}m{]}, z {[}m{]}).

\sphinxlineitem{Return type}
\sphinxAtStartPar
\sphinxcode{\sphinxupquote{numpy.ndarray}}

\end{description}\end{quote}

\end{fulllineitems}



\renewcommand{\indexname}{Python Module Index}
\begin{sphinxtheindex}
\let\bigletter\sphinxstyleindexlettergroup
\bigletter{h}
\item\relax\sphinxstyleindexentry{htrdrPy.data}\sphinxstyleindexpageref{htrdrPy.data:\detokenize{module-htrdrPy.data}}
\item\relax\sphinxstyleindexentry{htrdrPy.geometry}\sphinxstyleindexpageref{htrdrPy.geometry:\detokenize{module-htrdrPy.geometry}}
\item\relax\sphinxstyleindexentry{htrdrPy.helperFunctions}\sphinxstyleindexpageref{htrdrPy.helperFunctions:\detokenize{module-htrdrPy.helperFunctions}}
\item\relax\sphinxstyleindexentry{htrdrPy.postprocess}\sphinxstyleindexpageref{htrdrPy.postprocess:\detokenize{module-htrdrPy.postprocess}}
\item\relax\sphinxstyleindexentry{htrdrPy.script}\sphinxstyleindexpageref{htrdrPy.script:\detokenize{module-htrdrPy.script}}
\end{sphinxtheindex}

\renewcommand{\indexname}{Index}
\printindex
\end{document}